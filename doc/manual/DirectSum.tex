% -*- Mode: LaTeX -*-
% $Id: DirectSum.tex,v 1.2 1995/02/07 23:46:56 rz Exp $
\chapter{Sums, Products and Quotients of Domains}
\label{Quotients:Chap}

Let $\frak A$ be category of domains, \eg, groups, rings, fields, etc.
Given two elements of $\frak A$, say $A$ and $B$, there are often ways
of combining them to produce another element of $\frak A$.  This
chapter discusses several of those techniques. 

\section{Direct Sums}
\label{Direct:Sum:Sec}

Given two domains $A$ and $B$, their direct sum can be viewed as the
set of all pairs of elements in $A$ and $B$, where arithmetic
operations are performed component-wise.  We denote the direct sum
of $A$ and $B$ by $A \oplus B$.  If $A$, $B$ and $C$ are domains, then
\[
A \oplus ( B \oplus C) = (A \oplus B) \oplus C = A \oplus B \oplus C,
\]
so multiple direct sums are well defined.  Direct sums are
created using the function \keylisp{get-direct-sum}:

\begin{functiondef}{get-direct-sum}{$\mbox{\em domain}_1 \ldots
\mbox{\em domain}_n$}
Creates a domain that is the direct sum of $\mbox{\em domain}_1,
\ldots, \mbox{\em domain}_n$.  If any of the $\mbox{\em domain}_i$ are
themselves direct sums, then their components are included as if they
had been explicitly specified.
\end{functiondef}

The algebraic structure of the object returned by \keylisp{get-direct-sum}
depends upon the type of the objects of which it is composed.  The
following table illustrates the options.

\begin{center}
\begin{tabular}{||c|c|c||}
\multicolumn{1}{c}{$A$} & \multicolumn{1}{c}{$B$} & 
   \multicolumn{1}{c}{$A \oplus B$} \\ \hline
Group & Group & Group \\ \hline
Ring & Ring & Ring \\ \hline
Field & Field  & Ring \\ \hline 
Group & Ring & Group \\ \hline 
Abelian & Abelian & Abelian \\ \hline 
\end{tabular} 
\end{center}

The dimension of a direct sum is the number of its components:

\begin{functiondef}{dimension}{direct-sum}
Returns the number of components of the direct sum.
\end{functiondef}

To illustrate this the direct sum we can compute the direct sum of the
real numbers, the rational numbers and the rational integers as
follows:
\begin{code}
> (setq direct-sum (get-direct-sum (get-real-numbers)
                                   (get-rational-numbers)
                                   (get-rational-integers))
R (+) Q (+) Z
\end{code}

The symbol used to indicate a direct sum when only
{\sc ascii} characters are available is (+).  This domain has
dimension 3 and it is a ring as one might expect:
\begin{code}  
> (dimension direct-sum)
3
  
> (typep direct-sum 'ring)
T
  
> (typep direct-sum 'field)
NIL
\end{code}
  
Even though some of the components of the direct sum domain are
fields, the direct sum itself is a ring.  Individual elements of the
direct sum can be extracted using the routine \keylisp{ref}:
\begin{code}  
> (ref direct-sum 0)
R
  
> (ref direct-sum 2)
Z
\end{code}

\begin{functiondef}{ref}{direct-sum i}
Returns the {\em i}-th element (zero based) of {\em direct-sum}.  {\em
direct-sum} can either be a direct sum domain or an element of a
direct sum domain.
\end{functiondef}
  
Elements of a direct sum domains are created using
the following function:

\begin{functiondef}{make-element}{direct-sum $\mbox{\em element}_1
\ldots \mbox{\em element}_n$}
 Creates an element of {\em direct-sum\/}.  The number of elements
provided must match the dimension of {\em direct-sum\/}.  Each of the
elements is coerced into the domain of their associated component of
the {\em direct-sum} before the element is created.
\end{functiondef}

The following illustrates the use of elements of a direct sum.
\begin{code}  
> (setq x (make-element direct-sum 1 2 3))
1 (+) 2 (+) 3 
  
> (+ x x) 
2 (+) 4 (+) 6
  
> (* 3 x)
3 (+) 6 (+) 9
\end{code}
  
As with direct sum domains, the dimension of an element of a direct
sum domain can be computed using dimension, and individual components
can be determined using {\sf ref}.
\begin{code}
> (dimension x)
3

> (loop for i below (dimension x)
        do (format t "~%Component ~D: ~S, domain: ~S"
                   i (ref x i) (domain-of (ref x i))))
Component 0: 1, domain R
Component 1: 2, domain Q
Component 2: 3, domain Z
\end{code}

\section{Free Modules}
\label{FreeModules:Sec}
  
$M$ is a free $R$-module if $M$ is both a free abelian group and an
$R$-module. In Weyl, elements of free modules are represented as
$n$-tuples of elements of the coefficient domain $R$. Thus, we are
only able to deal with finite dimensional free modules. If $n$ is the
rank of $M$ as a free abelian group, then as a free $R$-module, $M$ is
isomorphic to the direct sum of n copies of $R$.  Closely related to
the concept of a free module is that of a vector space. A vector space
is a free module whose coefficient domain is a field. Additional
information about operations on elements of vector spaces can be found
in \sectref{VectorSpace:Sec}.
  
The basic routine for creating a free module is {\sf get-free-module}.
\begin{functiondef}{get-free-module}{ domain rank}
Creates a free module of dimension {\em rank} where the elements'
components are all elements of {\em domain}.  {\em domain} must be a
ring.  If {\em domain} is a field the domain returned will be a vector
space. 
\end{functiondef}
  
If one expects the coefficient domain to be a field, and thus the
affine module will actually be a vector space, then the routine
\keylisp{get-vector-space} should be used instead of
\keylisp{get-free-module}. This routine explicitly checks that the
coefficient domain is a field and signals an error if it is not a
field.
  
Once a free $R$-module has been created, it is often useful to refer
to the domain $R$ itself. This can be done using the routine
\keylisp{coefficient-domain-of}. The dimension of the free module can be
obtained using \keylisp{dimension}.
  
\begin{functiondef}{coefficient-domain-of}{domain}
Returns the domain of he coefficients of {\em domain}.
\end{functiondef}

\begin{functiondef}{dimension}{domain}
This method is defined for free modules but the value returned is not
specified. (Actually it should be infinity.)
\end{functiondef}

Elements of a free module can be created using the function
keylisp{make-element}. The routine \keylisp{make-point} calls {\sf
dimensions}, so that code that uses free modules as vector spaces can
be written more euphoniously.
  
\begin{functiondef}{make-element}{domain value \rest values}
Make an element of the module {\em domain}, whose first component is
{\em value}, etc. If value is a Lisp vector or one dimensional array,
then elements of that array are used as the components of the free
module element.
\end{functiondef}  
  
Use of these routines is illustrated below [Add something here --RZ]

  
\begin{functiondef}{ref}{vector i}
Returns the {\em i}-th element (zero based) of {\em vector}.
\end{functiondef}

\begin{functiondef}{inner-product}{u v}
Computes the inner (dot product) of the two vectors {\em u} and {\em
v}. If $u = \langle u_1, \ldots u_k \rangle$ and  $v = \langle v_1,
\ldots v_k \rangle$ then 
\[
\mbox{\sf (dot-product $u$ $v$)} = u_1 v_1 + u_2 v_2 + \cdots + u_k
v_k.
\]  
\end{functiondef}

\section{Tensor Products}
\label{Tensor:Product:Sec}

\section{Rings of Fractions}
\label{Quotient:Field:Sec}

Given an arbitrary ring $R$, we can construct a new ring, called the
\keyi{quotient ring} of $R$, whose elements are pairs of elements in
$R$ subject to the following equivalence relation. If $(a, b)$ and
$(c, d)$ are elements of a quotient ring, then they are equal if and
only if $ad = bc$. The sum and product of two elements of the quotient
ring are defined as follows
\[
\begin{eqalign}
(a, b) + (c, d) & = (ad + bc, bd), \\
(a, b) \times (c, d) & = (ac, bd).
\end{eqalign}
\]
If the ring $R$ is an integral domain then the quotient ring is
actually a field.

More generally, let $S$ be a multiplicatively closed subset of $R$. The
localization of $R$ with respect to a multiplicative subset of $R$, $S$,
written $S^{-1} R$, is the ring of pairs $(a, s)$, where $a$ is an
element of $R$ and $s$ is an element of $S$.
  
\begin{functiondef}{make-ring-of-fractions}{domain \optional
(multiplicative-set domain)}
Constructs a quotient ring from {\em domain}. If a second argument is
provided, then this {\sf make-quotient-ring} returns a ring
representing the localization of domain with respect to {\em
multiplicative-set}. Otherwise, the quotient ring of domain is
returned. If {\em domain} is an integral domain, then the ring returned will
be a field.
\end{functiondef}


\begin{functiondef}{make-quotient-field}{domain}
This generic function constructs the quotient field of {\em domain\/}.
The domain returned by this operation will be a field.  If {\em
domain} is itself a field, then it will be returned without any
modification.  If {\em domain} is a gcd domain then operations with
the elements of the resulting quotient field will reduce their answers
to lowest terms by dividing out the common gcd of the resulting
numerator and divisor.

\end{functiondef}

Two special cases are handled specially by \keylisp{make-quotient}.  If
the argument domain is either the rational integers or a polynomial
ring then special domains of the rational numbers or rational
functions are used.  This is for efficiency reasons.  Hopefully, a
more general solution can be found in the near future.

Two operations can be used to create elements of a quotient field:
\keylisp{make-quotient-element} and \keylisp{quotient-reduce}.  The operation
\keylisp{make-quotient-element} creates a quotient element from the
numerator and denominator domains.  {\sf Quotient-reduce} does the
same thing but removes the common GCD from the numerator and
denominator first.

(Ed: Need to think about things like localizations here.)

\begin{functiondef}{get-quotient-field}{ring}
Returns a field which is the quotient field of {\em ring\/}.  In some
cases, this is special cased to return return something more efficient
than the general quotient field objects.
\end{functiondef}

\begin{functiondef}{weyli::qf-ring}{qf}
Returns the ring from which the quotient field {\em qf} was built.
\end{functiondef}

\begin{functiondef}{numerator}{q}
Returns the numerator of {\em q\/}.
\end{functiondef}

\begin{functiondef}{denominator}{q}
Returns the denominator of {\em q\/}.
\end{functiondef}

\begin{functiondef}{quotient-reduce}{domain numerator \optional denominator}
{\em Numerator} and {\em denominator} are assumed to be elements of
base ring of {\em domain\/}.  {\sf quotient-reduce} creates and
quotient element in {\em domain} from {\em numerator} and {\em
denominator\/}.  If {\em denominator} is not provided, the
multiplicative unit of {\em domain} is used.
\end{functiondef}

\begin{functiondef}{with-numerator-and-denominator}{(num den) q \body
body} 
Creates a new lexical environment where the variables {\em num} and
{\em den} are bound to the numerator and denominator of {\em q\/}.
Each
\end{functiondef}


\section{Factor Domains}
\label{FactorDomains:Sec}

Let $B$ be a subgroup of $A$.  One can divide the elements of $A$ into
equivalence classes as follows:  Two elements of $A$ are in the same
equivalence class if their quotient is an element of $B$.  The
equivalence classes of $A$ with respect to $B$ form a group, called
the \keyi{factor group} of $A$ by $B$.  Using this construction one
can form factor modules of one module by a submodule.  If $A$ is a
ring and $B$ is an ideal of $A$, then the factor module of $A$ by $B$
is a ring, called the \keyi{factor ring} of $A$ by $B$.

\begin{figure}
\begin{center}
\BoxedEPSF{FactorDomains.eps scaled 500}
\end{center}
\caption{Factor Domains\label{FactorDomains:Fig}}
\end{figure}

Weyl provides four classes for dealing with factor domains, as shown
in \figref{FactorDomains:Fig}.  The factor domain (group, module or
ring) of $A$ and $B$ is written $A/B$.  The components of a factor
domain may be accessed using the following routines.

\begin{functiondef}{factor-numer-of}{factor-domain}
Returns the ``numerator'' of a factor domain.
\end{functiondef}

\begin{functiondef}{factor-denom-of}{factor-domain}
Returns the ``denominator of a factor domain.
\end{functiondef}
