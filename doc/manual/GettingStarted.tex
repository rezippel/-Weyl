% -*- Mode: LaTeX -*-
% $Id: GettingStarted.tex,v 1.3 1995/03/22 22:42:00 twilson Exp $
\chapter{Getting Started}
\label{Starting:Chap}

Weyl is implemented as a large body of functions that extend Lisp's
functionality to include non-numeric algebraic computations.  These
functions can be loaded into an existing Lisp image using the tools
described in \sectref{Creating:Weyl:Sec}.  All of Weyl's extensions
are attached to symbols in the {\sf weyl} or {\sf weyli} packages.
The philosophy behind these packages and how they are to be used is
discussed in \sectref{Weyl:Packages:Sec}.  Some issues with developing
programs using Weyl are discussed in \sectref{Weyl:Style:Sec}.  In
\sectref{FandG:Sec} we give a simple of example what is involved
when using Weyl.


\section{Creating a Weyl World}
\label{Creating:Weyl:Sec}

Common Lisp does not include a {\em defsystem} facility, which is
often used to manage the compilation dependencies between different
files in a Lisp system.\footnote{The defsystem facility plays the same
roles as {\tt make} files do in Unix systems-building.}  Although many
Lisp vendors provide a defsystem, there is no standard, and
each vendor's version differs slightly from the others.  As a
consequence, we have chosen not to use any of them and instead include
with Weyl a copy of Mark Kantrowitz's version.

The first thing one must do to get Weyl started is to load the file
{\sf sysdef.lisp}.  This file creates the {\sf weyl} package, where
all of Weyl's symbols are kept, describes which files comprise the
Weyl systems, and provides a number utility functions to help manage
the Weyl sources.  These functions are all defined in the {\sf user}
package, which is the package in which Lisp's usually start up.  Thus
you will probably not need to include the {\sf user::} prefix that we
have included in this section.

The function {\sf user::compile-weyl} will recompile any Weyl sources
that need recompilation.  To load the the recompiled files use {\sf
user::load-weyl}.  

\begin{functiondef}{user::compile-weyl}{}
Compile those Weyl source files that need recompilation.  Increment
Weyl's minor version number.
\end{functiondef}

\begin{functiondef}{user::load-weyl}{}
Load all binary files for Weyl and initialize Weyl's data structures.
If Weyl has already been loaded, this function loads any binary files
that are more recent than the ones already loaded.  This is commonly
used as a stripped down ``{\sf load-patches}.''
\end{functiondef}

\begin{functiondef}{user::dump-weyl}{}
Loads the current binary files for Weyl and dumps a world.
\end{functiondef}

\section{Packages}
\label{Weyl:Packages:Sec}

In Common Lisp, values and variables are attached to symbols.
Collections of symbols are organized into {\em packages} to control
access to the values and functions.  The symbols used by Weyl are
contained in two Common Lisp packages, {\sf weyl} and {\sf weyli}.
Those symbols that are intended for users are contained in the {\sf
weyl} package and exported.  The internal routines of Weyl are
contained in the {\sf weyli} package so that they will not interfere
with the user's symbols.

Common Lisp places all of its own symbols in the {\sf lisp} package.
Weyl extends the functionality of certain function in Common Lisp by
providing its own version of certain symbols from the {\sf lisp}
package.  Weyl's version of these symbols {\em shadow} the Common Lisp
symbols.  Among these symbols are the arithmetic functions, ($+$,
$-$, $*$, $\ldots$), most of the mathematical functions ({\sf
sqrt}, {\sf sin}, $\ldots$) and some of the sequence functions ({\sf
map}, {\sf delete}, {\sf member}, $\ldots$).  The variable {\sf
*weyli-shadowed-symbols*} contains a list of all shadowed symbols.  In
all cases, when applied to arguments admissible to the Common Lisp
version of the function, the Weyl versions of these functions perform
in exactly the same fashion as the Common Lisp version.  The only
difference is that the Weyl versions have been extended to work with
the mathematical objects defined in Weyl.

The only symbols in the {\sf weyl} package are those intended for
users.  The internal functions and values that are maintained by the
system itself and that are needed only by those extending Weyl are
contained in the {\sf weyli} package.

In some cases there are symbols with the same name in both the {\sf
weyl} and the {\sf weyli} packages.  This is usually done when a
version with more error checking is needed by users, but where such
error checking would compromise the performance of Weyl if used
internally where such error checking is not necessary.

An example of this is the {\sf make-element} function.  When
constructing vectors and matrices, {\sf weyl:make-element} makes sure
that each element of the vector or matrix to be created belongs to
the coefficient domain, and if not tries to coerce the element into
the coefficient domain.  The function {\sf weyli::make-element} does
not, but it is only used internally in the vector and matrix routines.

Most projects that use Weyl will create their own package for their
routines.  Because of the shadowing done by the {\sf weyl} package, if
one simply included {\sf weyl} in the user's new package one would
get many conflicts.  Instead one should use the function {\sf
use-weyl-package}, which will perform the necessary shadowing.

\begin{functiondef}{use-weyl-package}{package}
Does a shadowing import of the shadowed symbols in the {\sf weyl}
package and then makes {\em package} use the package {\sf weyl}.
\end{functiondef}

Thus a common idiom for creating a package for an new application that
uses Weyl is 
\begin{code}
(unless (find-package 'Application-Package)
  (make-package "Application-Package" :use '(lisp clos))
  (use-weyl-package (find-package 'Application-Package)))
\end{code}

\section{Programming Style}
\label{Weyl:Style:Sec}

In many cases the symbols in the {\sf weyl} package shadow normal
Common Lisp symbols.  When this is done, the functions in the {\sf
weyl} package are upward compatible extensions of the Common Lisp
ones.  For instance, the Common Lisp function {\sf lisp:expt} can
only be applied to Lisp numbers, while the function {\sf weyl:expt}
can be applied to polynomials, rational functions and other
mathematical objects as well as numbers.  In all cases, the {\sf weyl}
versions of these shadowed functions are upwardly compatible with the
Common Lisp versions.

We have adopted the convention that the name of all boolean predicates
end with a {\sf ?}.  This is different from the convention adopted by
Common Lisp, where most predicates end with a {\sf p}.  Thus the Weyl
function which determines if a number is positive is {\sf plus?}, where
the Common Lisp function is {\sf plusp}.  

\section{F and G Series Computation}
\label{FandG:Sec}

In this section we show how to perform a simple algebraic calculation
using Weyl.  Throughout this section we will use terms, such as {\em
domains\/}, that are not yet well defined.  These terms will be defined
in later sections.

First one needs to start a Lisp image that contains Weyl.  Using
Lucid's lisp on a Decstation we get a prompt like:
\noindent
\begin{code}
;;; Lucid Common Lisp/DECsystem
;;; Development Environment Version 4.0, 12 November 1990
;;; Copyright (C) 1985, 1986, 1987, 1988, 1989, 1990 by Lucid, Inc.
;;; All Rights Reserved
;;;
;;; This software product contains confidential and trade secret information
;;; belonging to Lucid, Inc.  It may not be copied for any reason other than
;;; for archival and backup purposes.
;;;
;;; Lucid Common Lisp is a trademark of Lucid, Inc.
;;; DECsystem is a trademark of Digital Equipment Corp

;;; Weyl Version 4.55, saved 14:38 Thursday, October 3, 1991

> 
\end{code}

The first thing that should be done at this point is to type {\sf
(user::load-weyl)} which will cause any bug fixes and improvements
that have been created since the Lisp image was dumped was loaded, as
illustrated below:
\begin{code}
> (user::load-weyl)
;;; Loading file #P"/amd/nori/a/weyl/develop/lisp-support.mbin"
;;; Done loading system USER::WEYL
;;; Weyl 4.56 loaded. 

> (in-package 'weyl)
#<Package "WEYL" 1028943E>
\end{code}
We have also switched the {\sf weyl} package so that can easily access
all the functions in Weyl.  We are now ready to begin.

\medskip
Computations using Weyl are usually involve first creating a set of
domains in which to perform a computation, creating a few elements of
these domains to initiate the computation, and then performing the
computation itself.  This is illustrated in the following routines
that computes the coefficients of the ``$F$ and $G$ series.''  These
coefficients are defined by the following recurrences.
\[
\begin{eqalign}
  f_0&= 0 \\
  f_n& = -\mu g_{n-1} - \sigma(\mu + 2\epsilon){\partial
f_{n-1} \over \partial \epsilon} + (\epsilon - 2 \sigma^2) {\partial
f_{n-1} \over \partial \sigma} - 3 \mu \sigma {\partial f_{n-1} \over
\partial \mu}\\
   g_0&=1\\ g_n&= f_{n-1} - \sigma(\mu +
2\epsilon){\partial g_{n-1} \over \partial \epsilon} + (\epsilon - 2
\sigma^2) {\partial g_{n-1} \over \partial \sigma} - 3 \mu \sigma
{\partial g_{n-1} \over \partial \mu}
\end{eqalign}
\]

The resulting sequences, $f_i$ and $g_i$, are all polynomials in $\mu$,
$\epsilon$ and $\sigma$ with integer coefficients.  Thus they lie in
the domain $\Z[\mu, \epsilon, \sigma]$.  Thus our first task in
computing them is to generate the domain $\Z[\mu, \epsilon, \sigma]$.
This can be done by the following line:
\begin{code}
> (setf R (get-polynomial-ring (get-rational-integers) '(m e s)))
Z[m, e, s]
\end{code}

\noindent
It is useful to have our hands on the variables $\mu$, $\epsilon$ and
$\sigma$.  This is done by coercing the symbols {\sf m}, {\sf e} and
{\sf s} into the domain {\sf R}.
\begin{code}
> (setf mu (coerce 'm R))
m
> (setf eps (coerce 'e R))
e
> (setf sigma (coerce 's R))
s
\end{code}

The coefficients of the first terms of the both recurrences are the
same, so rather than computing them afresh each time, we'll store them
in some global variables

\begin{code}
> (setq x1 (* (- sigma) (+ mu (* 2 eps))))
- s m + -2 s e

> (setq x2 (+ eps (* -2 (expt sigma 2))))
e + -2 s^2

> (setq x3 (* -3 mu sigma))
-3 s m

> (setq x4 (- mu))
- m
\end{code}
\noindent
Notice that we used the usual Lisp functions for manipulating the
elements of {\sf R}.  Second, there is an automatic coercion from the
integers into almost every domain we are working with, thus we could
write {\sf (* 2 eps)} above, rather than the more explicit {\sf (*
(coerce 2 R) eps)}.

Finally, we can write out the recursion formulae.

\begin{code}
(defun f (n) 
  (if (= n 0) (coerce 0 R)
    (let ((fn-1 (f (1- n))))
      (+ (* x4 (g (1- n)))
         (* x1 (partial-deriv fn-1 eps))
         (* x2 (partial-deriv fn-1 sigma))
         (* x3 (partial-deriv fn-1 mu))))))
\end{code}

\begin{code}
(defun g (n)
  (if (= n 0) (coerce 1 R)
    (let ((gn-1 (g (1- n))))
      (+ (f (1- n)) 
         (* x1 (partial-deriv gn-1 eps))
         (* x2 (partial-deriv gn-1 sigma))
         (* x3 (partial-deriv gn-1 mu))))))
\end{code}

And compute $f_5$ and $g_5$
\begin{code}
> (f 5)
- m^3 + (-24 e + 210 s^2) m^2 + (-45 e^2 + 630 s^2 e + -945 s^4) m

> (g 5)
-30 s m^2 + (-180 s e + 420 s^3) m
\end{code}

The explicit {\sf coerce}'s in the definitions of {\sf f} and {\sf g}
are included to ensure that {\sf f} and {\sf g} always return elements
of the same ring.

