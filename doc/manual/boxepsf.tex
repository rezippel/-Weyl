 %%%%%%% start BoxedEPSF.tex

  %%
 %%%%%%%%%%%%%%%%%%%%%%%%%%%%%%%%%%%%%%%%%%%%%%%%%%%%%%%%%%%%%
  %%
 %%%%%   BoxedEPSF.tex FOR FIGURE INSERTS OF EPSF NORM  %%%%%
 %%%%%   (EPSF = Encapsulated PostScript File)
  %%
 %%%%%%%%%%%%%%%%%%%%%%%%%%%%%%%%%%%%%%%%%%%%%%%%%%%%%%%%%%%%%
  %%
 %%%  AUTHOR: Laurent Siebenmann
  %%    lcs@matups.matups.fr
  %%
 %%%  VERSIONS: Feb-May 1991, 7 May 1991
  %%
 %%%  SOMMAIRE: BoxedEPSF.tex d\'efinit des macro-commandes
  %%    qui permettent d'int\'egrer dans un document TeX des
  %%    objets graphiques d\'ecrits par fichier de norme EPSF,
  %%    tout en accordant a chacun le statut d'une bo\^ite TeX ayant
  %%    les bonnes dimensions.  La (seule!) contribution unique
  %%    de ce fichier est de faire cela d'une fa{\c}con universelle.
  %%    C'est a dire de fa{\c}con \`a pouvoir commod\'ement
  %%    servir avec tout pilote d'imprimante de norme
  %%    PostScript --- malgr\'e l'absence d'une norme
  %%    pour \special.
  %%
 %%%  POSTINGS: anonymous ftp
  %%  ---  ftp 28.146.7.200 (alias shape.mps.ohio-state.edu); login:
  %%  anonymous; password: <anything>; directory pub/osutex
  %%
  %%  ---  ftp 130.84.128.100 (alias rsovax.circe.fr);
  %%  login: anonymous; password: <anything>; directory
  %%  [anonymous.siebenmann]
  %%
 %%%% DOCUMENTATION:
  %%  --- see BoxedEPSF.doc
  %%
 %%%% ACTIVATION:
  %%    by a driver-by-driver protocol
  %%    see \SetTexturesEPSFSpecial
  %%    and its companions below.
  %%

% \def\LOADED{\relax}

% \ifx\BoxedEPSFLoaded\LOADED
%  \immediate\write16{}
%  \immediate\write16{  BoxedEPSF macros already defined!}
%  \immediate\write16{}
%  \endinput    % This line confuses latex2html
% \fi

% \let\BoxedEPSFLoaded\LOADED

 \chardef\CatAt\the\catcode`\@
 \chardef\CatColon\the\catcode`\:
 \catcode`\@=11
 \catcode`\:=12

 %%temporarily suppress Plain's logging of allocations
 \let\wlog@ld\wlog
 \def\wlog#1{\relax}

 %% New for TOOLS
 \newif\ifIN@
 \newdimen\XShift@ \newdimen\YShift@
 \newtoks\Realtoks

 %%% New for Boxed EPSF
  %
 \newdimen\Wd@ \newdimen\Ht@
 \newdimen\Wd@@ \newdimen\Ht@@
 %
 \newdimen\TT@
 \newdimen\LT@
 \newdimen\BT@
 \newdimen\RT@
 %
 \newdimen\XSlide@ \newdimen\YSlide@
 %
 \newdimen\TheScale  %% secretly scale in mils: 1pt= 1mil
 \newdimen\FigScale  %% secretly scale in mils: 1pt= 1mil
 %
 \newdimen\ForcedDim@@

 \newtoks\EPSFDirectorytoks@
 \newtoks\EPSFNametoks@
 \newtoks\BdBoxtoks@

 \newif\ifNotIn@
 \newif\ifForcedDim@
 \newif\ifForcedHeight@
 \newif\ifPSOrigin

 \newread\EPSFile@


 %%%% WORD-PROCESSING MACROS
  %%
  %%% \IN@0#1@#2@ : Is 1st exp of #1 in 1st exp of #2 ??
   %% Answer in \ifIN@
 \newif\ifIN@\def\IN@{\expandafter\INN@\expandafter}
  \long\def\INN@0#1@#2@{\long\def\NI@##1#1##2##3\ENDNI@
    {\ifx\m@rker##2\IN@false\else\IN@true\fi}%
     \expandafter\NI@#2@@#1\m@rker\ENDNI@}
  \def\m@rker{\m@@rker}

  %%%  \SPLIT@0#1@#2@  :  Split 1st exp of #2 at 1st exp of #1
   %%  \Initialtoks@ , \Terminaltoks@ will contain pieces
  \newtoks\Initialtoks@  \newtoks\Terminaltoks@
  \def\SPLIT@{\expandafter\SPLITT@\expandafter}
  \def\SPLITT@0#1@#2@{\def\TTILPS@##1#1##2@{%
     \Initialtoks@{##1}\Terminaltoks@{##2}}\expandafter\TTILPS@#2@}

 %%%% MACROS TO TRIM  \ForeTrim@0#1@ and \Trim@0#1@
   %% result appears in \Trimtoks@
   %% LIMITATION: assume no multiple spaces to trim

  \newtoks\Trimtoks@

  %%% \ForeTrim@0#1@ trims initial space of first erpansion of #1
   %% #1 of form \the\toks0 or \mymacro
 \def\ForeTrim@{\expandafter\ForeTrim@@\expandafter}
 \def\ForePrim@0 #1@{\Trimtoks@{#1}}
 \def\ForeTrim@@0#1@{\IN@0\m@rker. @\m@rker.#1@%
     \ifIN@\ForePrim@0#1@%
     \else\Trimtoks@\expandafter{#1}\fi}
   %%\m@rker expands here to \m@@rker since spot initial,
   %% so no confusuion with \m@rker

  %%% \Trim@0#1@ trims init and terminal spaces
   %% Same syntax.
   %% Warns if internal spaces found.
   %%
  \def\Trim@0#1@{%
      \ForeTrim@0#1@%
      \IN@0 @\the\Trimtoks@ @%
        \ifIN@
             \SPLIT@0 @\the\Trimtoks@ @\Trimtoks@\Initialtoks@
             \IN@0\the\Terminaltoks@ @ @%
                 \ifIN@
                 \else \Trimtoks@ {FigNameWithSpace}%
                 \fi
        \fi
      }


  %%%% MATH MACROS (provisional)
    %% use dimen registers for reals; unit 1pt
    %% (numerical dimension arguments OK unless contrary noted)

  %%%% One needs the point token seq (pt with cat 12) USES dimen 0
   \newtoks\pt@ks
   \def \getpt@ks 0.0#1@{\pt@ks{#1}}
   \dimen0=0pt\expandafter\getpt@ks\the\dimen0@

   %%% Convert dimen to "decimal multiplier"% USES dimens 0,2
  \newtoks\Realtoks% the output!
  \def\Real#1{%
    \dimen2=#1%
      \SPLIT@0\the\pt@ks @\the\dimen2@%%  lop off the points
       \Realtoks=\Initialtoks@%\showthe\Realtoks
            }

   %%% Multiplication
      % USES dimens 0,2,4,6; preserves args; output \Product
   \newdimen\Product
   \def\Mult#1#2{%
     \dimen4=#1\relax
     \dimen6=#2%
     \Real{\dimen4}%
     \Product=\the\Realtoks\dimen6%
        }

   %%% Inverse
     % USES dimens 0; preserves arg; output \Inverse
 \newdimen\Inverse
 \newdimen\hmxdim@ \hmxdim@=8192pt%halfmaxdimen
 \def\Invert#1{%
  \Inverse=\hmxdim@
  \dimen0=#1%
  \divide\Inverse \dimen0%
  \multiply\Inverse 8}

 %%% \Rescale#1#2#3  % USES dimens 0,2,4,6
  %%  alters dimen register #1 by ratio #2/#3
  %%  where #2,#3 can be raw dimensions OR dimen registers
   \def\Rescale#1#2#3{% Adequate accuracy. Can improve.
              \divide #1 by 100\relax
              \dimen2=#3\divide\dimen2 by 100 \Invert{\dimen2}%
              \Mult{#1}{#2}%
              \Mult\Product\Inverse
              #1=\Product}

 %%% \Scale#1 scales dimen register #1
   %  by dimen register real \TheScale; USES dimens 0
  \def\Scale#1{\dimen0=\TheScale %
      \divide #1 by  1280 %% 1280*5120*10=1000*2^16
      \divide \dimen0 by 5120 %
      \multiply#1 by \dimen0
      \divide#1 by 10   %% max size of #1 about 32000/10 pt
     }

 %%% SCRUNCHING BOXES AND SHIFTING CONTENTS
  %% TeX has to do this in general
  %% since some drivers do not let one do it readily using Postscript
  %%

 \newbox\scrunchbox

 %%% \Scrunched#1 puts #1 in an hbox
  %%    then in effect zeros the dimensions of this box
 \def\Scrunched#1{{\setbox\scrunchbox\hbox{#1}%
   \wd\scrunchbox=0pt
   \ht\scrunchbox=0pt
   \dp\scrunchbox=0pt
   \box\scrunchbox}}

  %%% \Shifted@#1 puts #1 in \hbox
   %% then locates basepoint to bottom left corner
   %% then translates ink only by \XShift@,\YShift@
   %% with Postscript convention
   %% For simplicity use only on scrunched boxes
  %\newdimen\XShift@
  %\newdimen\YShift@
 \def\Shifted@#1{%
   \vbox {\kern-\YShift@
       \hbox {\kern\XShift@\hbox{#1}\kern-\XShift@}%
           \kern\YShift@}}

  %%% \cBoxedEPSF#1 the main macro
   %%  component macros are explained in order below
   %
 \def\cBoxedEPSF#1{{}\leavevmode %{} fixes box mirage for \Mas
   \ReadNameAndScale@{#1}%
   \SetEPSFSpec@
   \ReadEPSFile@ \ReadBdB@x
   %% Calculations
     \TrimFigDims@
     \CalculateFigScale@
     \ScaleFigDims@
     \SetInkShift@
   \hbox{$\mathsurround=0pt\relax
         \vcenter{\hbox{%
             \FrameSpider{\hskip-.4pt\vrule}%
             \vbox to \Ht@{\offinterlineskip\parindent=\z@%
                \FrameSpider{\vskip-.4pt\hrule}\vfil
                \hbox to \Wd@{\hfil}%
                \vfil
                \InkShift@{\EPSFSpecial{\EPSFSpec@}{\FigSc@leReal}}%
             \FrameSpider{\hrule\vskip-.4pt}}%
         \FrameSpider{\vrule\hskip-.4pt}}}%
     $}%
    \CleanRegisters@
    }

 \def\tBoxedEPSF#1{\setbox4\hbox{\cBoxedEPSF{#1}}%
     \setbox4\hbox{\raise -\ht4 \hbox{\box4}}%
     \box4
      }

 \def\bBoxedEPSF#1{\setbox4\hbox{\cBoxedEPSF{#1}}%
     \setbox4\hbox{\raise \dp4 \hbox{\box4}}%
     \box4
      }

  \let\BoxedEPSF\cBoxedEPSF% default setting

  %% Some compatibility with BoxedArt.tex
   %
   \let\BoxedArt\BoxedEPSF

  %% Some compatibility with Sweet-teX
   %
  \def\gLinefigure[#1scaled#2]_#3{%
        \BoxedEPSF{#3 scaled #2}}

  %% Some compatibility with Rokicki's dvips
   %
  \let\EPSFbox\bBoxedEPSF \let\EPSFfile\bBoxedEPSF

  \def\EPSFxsize{\afterassignment\ForceW@\ForcedDim@@}
      \def\ForceW@{\ForcedDim@true\ForcedHeight@false}

  \def\EPSFysize{\afterassignment\ForceH@\ForcedDim@@}
      \def\ForceH@{\ForcedDim@true\ForcedHeight@true}


 %%% \ReadNameAndScale@#1
  %
 \def\ReadNameAndScale@#1{\IN@0 scaled@#1@% DOUBLE BARRELED
   \ifIN@\ReadNameAndScale@@0#1@%
   \else \ReadNameAndScale@@0#1 scaled\DefaultMilScale @
   \fi}

 \def\ReadNameAndScale@@0#1scaled#2@{% HELPER MACRO
    \Trim@0#1@%
    \EPSFNametoks@\expandafter{\the\Trimtoks@}%
    \FigScale=#2 pt%
     }

 \def\SetDefaultEPSFScale#1{%
      \global\def\DefaultMilScale{#1}}

 \SetDefaultEPSFScale{1000}


 %%% \ReadEPSFile@
  %
 \def \SetBogusBbox@{%
     \global\BdBoxtoks@{ BoundingBox:0 0 100 100 }%
     \global\def\BdBoxLine@{ BoundingBox:0 0 100 100 }%
     }

 \def\ReadEPSFile@{%\show\EPSFSpec@%
     \openin\EPSFile@\EPSFSpec@
     \relax  %necessary to prevent precocious expansion of \ifeof
  \ifeof\EPSFile@
     \SetBogusBbox@
     \immediate\write16{}%
     \message{ *** EPS FILE  }%
     \message\expandafter{\the\EPSFNametoks@}%
     \message{ NOT FOUND!  }%
     \immediate\write16{}\relax%
  \else%\fi
   \begingroup
   \catcode`\%=12\catcode`\:=12\catcode`\\=12
   \NotIn@true
    \loop
      \ifeof\EPSFile@\NotIn@false
        \SetBogusBbox@
        \immediate\write16{}%
        \message{ *** BoundingBox not found in }%
        \message\expandafter{\the\EPSFNametoks@\space *** }%
        \immediate\write16{}%
      \else\global\read\EPSFile@ to \BdBoxLine@
      \fi
      \global\BdBoxtoks@\expandafter{\BdBoxLine@}%
      \IN@0BoundingBox:@\the\BdBoxtoks@ @%
      \ifIN@\NotIn@false\fi%
    \ifNotIn@\repeat
   \endgroup\relax
  \fi
  \closein\EPSFile@
   }


  %%% \ReadBdB@x
   % Rmk For simplicity 0 not used in syntax
   %  of \ReadBdB@x@,  \ReadBdB@x@@
  \def\ReadBdB@x{% PART 0
   \expandafter\ReadBdB@x@\the\BdBoxtoks@ @}

  \def\ReadBdB@x@#1BoundingBox:#2@{% PART 1
    \ForeTrim@0#2@%
    \expandafter\ReadBdB@x@@\the\Trimtoks@ @%
   }

  \newtoks\LLXtoks@  %% useful info for Oz
  \newtoks\LLYtoks@

  \def\ReadBdB@x@@#1 #2 #3 #4@{% PART 2
      \Wd@=#3bp\advance\Wd@ by -#1bp%
      \Ht@=#4bp\advance\Ht@ by-#2bp%
       \Wd@@=\Wd@ \Ht@@=\Ht@ %% useful info for Clark
       \LLXtoks@={#1}\LLYtoks@={#2}%% useful info for Oz
      \ifPSOrigin\XShift@=-#1bp\YShift@=-#2bp\fi
     }

  %%% \SetEPSFDirectory
   %
   \def\G@bbl@#1{}
   \bgroup%\def\\{}
     \global\edef\OtherB@ckslash{\expandafter\G@bbl@\string\\}
   \egroup

  \def\SetEPSFDirectory{%  Part 1
           \bgroup\catcode`\:=12\relax
           \SetEPSFDirectory@}

 \def\SetEPSFDirectory@#1{% Part 2
    \Trim@0#1@%  result in \Trimtoks@
    \global\EPSFDirectorytoks@\expandafter{\the\Trimtoks@ }\relax
    \egroup}

  %%% \SetEPSFSpec@
 \def\SetEPSFSpec@{%
     \bgroup
     \let\\=\OtherB@ckslash
     \global\edef\EPSFSpec@{\the\EPSFDirectorytoks@\the\EPSFNametoks@}%
     \global\edef\EPSFSpec@{\EPSFSpec@}%
     \egroup}


 %%% \TrimFigDims@
  %
 \def\TrimTop#1{\advance\TT@ by #1}
 \def\TrimLeft#1{\advance\LT@ by #1}
 \def\TrimBottom#1{\advance\BT@ by #1}
 \def\TrimRight#1{\advance\RT@ by #1}

 \def\TrimBoundingBox#1{%
   \TrimTop{#1}%
   \TrimLeft{#1}%
   \TrimBottom{#1}%
   \TrimRight{#1}%
       }

 \def\TrimFigDims@{%
    \advance\Wd@ by -\LT@
    \advance\Wd@ by -\RT@ \RT@=\z@
    \advance\Ht@ by -\TT@ \TT@=\z@
    \advance\Ht@ by -\BT@
    }


 %%% \CalculateFigScale@
  %
  \def\ForceWidth#1{\ForcedDim@true
       \ForcedDim@@#1\ForcedHeight@false}

  \def\ForceHeight#1{\ForcedDim@true
       \ForcedDim@@=#1\ForcedHeight@true}

  \def\epsfxsize{\afterassignment\ForceW@\ForcedDim@@}
      \def\ForceW@{\ForcedDim@true\ForcedHeight@false}

  \def\epsfysize{\afterassignment\ForceH@\ForcedDim@@}
      \def\ForceH@{\ForcedDim@true\ForcedHeight@true}

  \def\CalculateFigScale@{%
            %Have default \FigScale or read \FigScale
     \ifForcedDim@\FigScale=1000pt% %% start afresh
           \ifForcedHeight@
                \Rescale\FigScale\ForcedDim@@\Ht@
           \else
                \Rescale\FigScale\ForcedDim@@\Wd@
           \fi
     \fi
     \Real{\FigScale}%
     \edef\FigSc@leReal{\the\Realtoks}%
     }

  \def\ScaleFigDims@{\TheScale=\FigScale
      \ifForcedDim@
           \ifForcedHeight@ \Ht@=\ForcedDim@@  \Scale\Wd@
           \else \Wd@=\ForcedDim@@ \Scale\Ht@
           \fi
      \else \Scale\Wd@\Scale\Ht@
      \fi
      \ForcedDim@false
      \Scale\LT@\Scale\BT@  %%%\Scale\Wd@\Scale\Ht@
      \Scale\XShift@\Scale\YShift@
      }

  %%% \ShowReservedBoxes
   %%  shows (prints) corrected scaled and positioned
   %%  bounding boxes; for diagnostics
  %%% \HideReservedBoxes makes them invisible again
   %%
 \def\HideReservedBoxes{\gdef\FrameSpider##1{}}
 \def\ShowReservedBoxes{\gdef\FrameSpider##1{##1}}
 \let\HideDisplacementBoxes\HideReservedBoxes  %% some synonyms
 \let\ShowDisplacementBoxes\ShowReservedBoxes
 \let\HideFigureFrames\HideReservedBoxes
 \let\ShowFigureFrames\ShowReservedBoxes
  \ShowReservedBoxes

  %%% \hSlide#1, \vSlide#1
   %%
 \def\hSlide#1{\advance\XSlide@ by #1}
 \def\vSlide#1{\advance\YSlide@ by #1}

  %%% \SetInkShift@, \InkShift@#1
   %%
  \def\SetInkShift@{%
            \advance\XShift@ by -\LT@
            \advance\XShift@ by \XSlide@
            \advance\YShift@ by -\BT@
            \advance\YShift@ by -\YSlide@
             }
   %
  \def\InkShift@#1{\Shifted@{\Scrunched{#1}}}

  %%% \CleanRegisters@
   %
  \def\CleanRegisters@{%
      \globaldefs=1\relax
        \XShift@=\z@\YShift@=\z@\XSlide@=\z@\YSlide@=\z@
        \TT@=\z@\LT@=\z@\BT@=\z@\RT@=\z@
      \globaldefs=0\relax}

 %%% Special syntax for several drivers. The macros
  %% \SetTexturesEPSFSpecial  %% Textures
  %% \SetUnixCoopEPSFSpecial %% dvi2ps early unix
  %% \SetBetcholsheimEPSFSpecial %% dvi2ps by S.P.Betcholsheim
  %% \SetLisEPSFSpecial %% dvi2ps by Tony Lis
  %% \SetRokickiEPSFSpecial  %% dvips by Tom Rokicki
  %% \SetOzTeXEPSFSpecial  %% OzTeX by Andrew Trevorrow
  %% \SetPSprintEPSFSpecial %% PSprint by Andrew Trevorrow
  %% \SetArborEPSFSpecial  %% ArborTeX DVILASER/PS
  %% \SetClarkEPSFSpecial %% dvitops by James Clark
  %% \SetDVIPSoneEPSFSpecial %% DVIPSONE of Y&Y
  %% \SetBeebeEPSFSpecial %% DVIALW by N. Beebe
  %% \SetStandardEPSFSpecial %% Nonexistant: Placebo below
  %% These macros adapt to various drivers by (re-)defining
  %% the macro \EPSFSpecial#1#2, where
  %% #1 = EPS file pathname (use \\ for the letter backslash)
  %% #2 = scale in mils
    %% Be wary of using strange characters in pathnames!

 %% Textures, Blue Sky Research, Barry Smith
 \def\SetTexturesEPSFSpecial{\PSOriginfalse%\PSOrigintrue
  \gdef\EPSFSpecial##1##2{\relax
    \edef\specialthis{##2}%
    \SPLIT@0.@\specialthis.@\relax
    %\showthe\Initialtoks@
    \special{illustration ##1 scaled
                        \the\Initialtoks@}}}

  %% Unix : dvi2ps by:  Mark Senn, Stephan  Bechtolsheim,
   % Bob  Brown, Richard, Furuta, James Schaad,
   % Robert  Wells, Norm Hutchinson, Neal Holtz.
   % Introduced by B. Horn <bkph@ai.mit.edu>
  \def\SetUnixCoopEPSFSpecial{\PSOrigintrue % Please test!
   \gdef\EPSFSpecial##1##2{%
      \dimen4=##2pt% convert real to dimen
      \divide\dimen4 by 1000\relax
      \Real{\dimen4}%dimens 0,2 used here
      \edef\Aux@{\the\Realtoks}%
      %%convert dimen to real
      \special{psfile=##1\space
        hscale=\Aux@\space
        vscale=\Aux@}}}

  %% dvi2ps by S.P. Bechtolsheim,
   % implantations? ; dates?; availability?
   % Introduced by B. Horn <bkph@ai.mit.edu>; please test!!
  \def\SetBechtolsheimRokickiEPSFSpecial{\PSOrigintrue
   \gdef\EPSFSpecial##1##2{%
      \dimen4=##2pt% convert real to dimen
      \divide\dimen4 by 1000\relax
      \Real{\dimen4}% dimens 0,2 used here
      \edef\Aux@{\the\Realtoks}%
      %%convert dimen to real
      \special{ps: psfiginit}%
      \special{ps: literal 1 1 0 0 1 1 startTexFig
           \the\mag\space 1000 div \Aux@\space mul
           \the\mag\space 1000 div \Aux@\space mul scale}%
      \special{ps: include  ##1}%
      \special{ps: literal endTexFig}%
        }}

  %% dvi2ps by Tony Lis,
   % implantations? ; dates?; availability?
   % Introduced by B. Horn <bkph@ai.mit.edu>; please test!!
  \def\SetLisEPSFSpecial{\PSOrigintrue
   \gdef\EPSFSpecial##1##2{%
      \dimen4=##2pt% convert real to dimen
      \divide\dimen4 by 1000\relax
      \Real{\dimen4}% dimens 0,2 used here
      \edef\Aux@{\the\Realtoks}%
      %%convert dimen to real
      \special{pstext="1 1 0 0 1 1 startTexFig\space
           \the\mag\space 1000 div \Aux@\space mul
           \the\mag\space 1000 div \Aux@\space mul scale}%
      \special{psfile=##1}%
      \special{pstext=endTexFig}%
        }}

  %% dvips by Tom Rokicki; driver in portable C
   % This driver improves on dvi2ps; its Postscript
   % output is thee times as compact as that
   % produced by dvi2ps
   % Introduced by W.D. Neumann <neumann@mps.ohio-state.edu>
  \def\SetRokickiEPSFSpecial{\PSOrigintrue
   \gdef\EPSFSpecial##1##2{%
      \dimen4=##2pt% convert real to dimen
      \divide\dimen4 by 10\relax
      \Real{\dimen4}% dimens 0,2 used here
      \edef\Aux@{\the\Realtoks}%
      %%convert dimen to real
      \special{psfile="##1"\space
        hscale=\Aux@\space
        vscale=\Aux@}}}

  \def\SetInlineRokickiEPSFSpecial{\PSOrigintrue
   \gdef\EPSFSpecial##1##2{%
      \dimen4=##2pt% convert real to dimen
      \divide\dimen4 by 1000\relax
      \Real{\dimen4}% dimens 0,2 used here
      \edef\Aux@{\the\Realtoks}%
      %%convert dimen to real
      \special{ps::[begin] 1 1 0 0 1 1 startTexFig\space
           \the\mag\space 1000 div \Aux@\space mul
           \the\mag\space 1000 div \Aux@\space mul scale}%
      \special{ps: plotfile ##1}%
      \special{ps::[end] endTexFig}%
        }}

  %% OzTeX, by AndrewTrevorrow,
   %  complete public domain TeX for Macintosh
   %  Send 10 UNFORMATTED 800K disks
   %  with return postage
   %  Peter Abbott, Computing Service,
   %  Aston University, Aston Triangle, Birmingham B4 7ET
  \def\SetOzTeXEPSFSpecial{\PSOriginfalse % artifice; see below
  \gdef\EPSFSpecial##1##2{%note order
     \special{##1\space
       ##2 1000 div \the\mag\space 1000 div mul
       ##2 1000 div \the\mag\space 1000 div mul scale
       \the\LLXtoks@\space neg \the\LLYtoks@\space neg translate
             }}}

 %% PSprint,  by AndrewTrevorrow for VaX VMS
  % diagnosed and tested 2-91 by Max Calviani
  % <ISICA@ASTRPD.infn.it>
  \let\SetPSprintEPSFSpecial\SetOzTeXEPSFSpecial
  \let\SetPsprintEPSFSpecial\SetOzTeXEPSFSpecial

 %% DVILASER/PS driver originally written by David Fuchs
  % marketed and supported by ArborTeXt  535 W. William St.
  % Suite 300, Ann Arbor, MI 48103, U.S.A
  % (313) 996-3566 (313) 996-3573
  % help@arbortext.com, Andrew Dobrowolski
 \def\SetArborEPSFSpecial{\PSOriginfalse % check!
   \gdef\EPSFSpecial##1##2{%
     \edef\specialthis{##2}%
     \SPLIT@0.@\specialthis.@\relax % suppress decimals (nec!)
     \special{ps: epsfile ##1\space \the\Initialtoks@}}}

 %% dvitops, (c) James Clark <jjc@jclark.uucp>
  % public domain; distributed by UK TeX Archive
  % computers: unix, msdos, vms, primos and vm/cms,
  % (introduced by S. Ratz <spqr@uk.ac.southampton.ecs>)
 \def\SetClarkEPSFSpecial{\PSOriginfalse % please test!
   \gdef\EPSFSpecial##1##2{%
     \Rescale {\Wd@@}{##2pt}{1000pt}%
     \Rescale {\Ht@@}{##2pt}{1000pt}%
     \special{dvitops: import
           ##1\space\the\Wd@@\space\the\Ht@@}}}

 %% DVIPSONE, for PC compatibles
  % Y&Y, 106 Indian Hill, Carlisle MA 01741, USA
  % (508) 371-3286
  % (introduced by B. Horn <bkph@ai.mit.edu>)
  \let\SetDVIPSONEEPSFSpecial\SetUnixCoopEPSFSpecial
  \let\SetDVIPSoneEPSFSpecial\SetUnixCoopEPSFSpecial

 %% DVIALW by N. Beebe, public domain, charge $100
  % DVI Driver Distribution, Center for Scientific Computing,
  % Department of Mathematics, 220 South Physics Building,
  % University of Utah, Salt Lake City, UT 84112, USA
  % (introduced by B. Horn <bkph@ai.mit.edu>)
  \def\SetBeebeEPSFSpecial{%please test!
   \PSOriginfalse%
   \gdef\EPSFSpecial##1##2{\relax
    \special{language "PS"
      literal "##2 1000 div ##2 1000 div scale
      position = "bottom left",
      include "##1"}}}
  \let\SetDVIALWEPSFSpecial\SetBeebeEPSFSpecial

  \let\SetDVIALWEPSFSpecial\SetBeebeEPSFSpecial

 \def\SetStandardEPSFSpecial{%
   \gdef\EPSFSpecial##1##2{%
     \immediate\write16{}
     \immediate\write16{%
       **** Sorry! There is still no standard for \string%
       \special \space EPSF integration *****}%
     \immediate\write16{%
      --- So you will have to identify your driver using a command}%
     \immediate\write16{%
      --- of the form \string\Set...EPSFSpecial, in order to get}%
     \immediate\write16{%
      --- your graphics to print.  See BoxedEPSF.doc.}%
     \immediate\write16{}
     \KillEPSFSpecial
     }}

  \def\KillEPSFSpecial{\gdef\EPSFSpecial##1##2{}}

  \SetStandardEPSFSpecial %% currently gives warning

 \let\wlog\wlog@ld %%restore logging

 \catcode`\@=\CatAt
 \catcode`\:=\CatColon

 %% The following are specialization for Unix  --RZ
 \SetRokickiEPSFSpecial
 \SetEPSFDirectory{Pix/}

 %% Using Textures use the following  --RZ
%% \SetTexturesEPSFSpecial
%% \SetEPSFDirectory{:Pix:}

  \endinput

%%%%%%%%%%%% ASCII Character test
 %
 %       Upper case letters: ABCDEFGHIJKLMNOPQRSTUVWXYZ
 %       Lower case letters: abcdefghijklmnopqrstuvwxyz
 %                                   Digits: 0123456789
 % Square, curly, angle braces, parentheses: [] {} <> ()
 %           Backslash, slash, vertical bar: \ / |
 %                              Punctuation: . ? ! , : ;
 %          Underscore, hyphen, equals sign: _ - =
 %                Quotes--right left double: ' ` "
 %"at", "number" "dollar", "percent", "and": @ # $ % &
 %           "hat", "star", "plus", "tilde": ^ * + ~
 %
 %%%%%%%%%%%%%%%%%%%%%%%%
 %
 % Une seule erreur de transmission peut empoisoner un programme!
 %
 % A single transmission error can poison a whole program.
 %
 %%%%%%%%%%%%%%%%%%%%%%%%


 %%%%%%% end BoxedEPSF.tex
