% -*- Mode: LaTeX -*-
% $Id: Introduction.tex,v 1.2 1995/03/22 22:42:02 twilson Exp $
\chapter{Introduction}
\label{Intro:Chap}


In the last twenty years the algorithms and techniques for manipulating
symbolic mathematical quantities have improved dramatically.  These
techniques have been made available to a practitioners through a number
of algebraic manipulation systems.  Among the most widely distributed
systems are Macsyma \cite{Macsyma:Manual}, Reduce \cite{Reduce:Manual},
Maple \cite{Maple:Manual} and Mathematica \cite{Wolfram88}.
These systems are designed to be self-contained and are not intended to
be incorporated into larger, more specialized systems.  This is at odds
with the experience with numerical computation where libraries of
carefully coded routines like Linpack \cite{LINPACK:Manual} and Eispack
\cite{EISPACK:Manual} have been of the greatest value because they
could be incorporated in larger systems (like fluid dynamics simulators
or circuit analyzers).  Linear algebra systems like Matlab
\cite{Matlab:Manual}, though of significant value, where developed much
later and tend to be used more for research in linear algebra than
as part of large computations.

Another limitation of current symbolic mathematics systems is that they
generally deal with a relatively limited and fixed set of algebraic
types, which the user is not expected to extend significantly.  Thus it
may be difficult for a user to experiment with algorithms for dealing
with Poisson series if the algebra system does not already have a data
type that matches the behavior of Poisson series.  This situation is
exacerbated by the unavailability of the code that implements the
algebraic algorithms of the symbolic manipulation system.  Scratchpad
\cite{Jenks84} is a noticeable exception to this trend in that the
developers plan to make the algebra code for the system widely
available, and the internal data typing mechanisms provided by
Scratchpad are designed with the extension mechanisms just mentioned in
mind.

Weyl is an extensible algebraic manipulation substrate that has been
designed to represent all types of algebraic objects.  It deals not
only with the basic symbolic objects like polynomials, algebraic
functions and differential forms, but can also deal with higher level
objects like groups, rings, ideals and vector spaces.  Furthermore, to
encourage the use of symbolic techniques within other applications,
Weyl is implemented as an extension of Common Lisp
\cite{CommonLisp:Manual} using the Common Lisp Object Standard
\cite{CLOS:Manual} so that all of Common Lisp's facilities and
development tools can be used in concert with Weyl's symbolic tools.

It should be noted that the initial implementation of Weyl is intended
to be as clean and semantically correct as possible.  The primary goal
of Weyl is provide the tools needed to express algebraic algorithms
naturally and succinctly.  Nonetheless, we believe that algebraic
algorithms can be efficiently implemented within the Weyl framework
even though that was not a goal of the initial implementation.

\section{The Domain Concept}
\label{Domain:Concept:Sec}

One of the novel concepts in Weyl is that of a {\em domain\/}.  This
section gives three examples that illustrate the need for domains. 

Consider the problem of integrating the function $1/(x^3-2)$:
\[
\int \frac{dx}{x^3-2} = 
\frac{\sqrt[3]{2}}{12} \log\frac{(x - \sqrt[3]{2})^2}{x^2 +
\sqrt[3]{2}x + \sqrt[3]{4}} + \frac{\sqrt[3]{2} \sqrt{3}}{6} 
    \arctan\frac{- \sqrt{3} x}{2 \sqrt[3]{2} + x}.
\]
Notice that, although the original problem involved rational functions
with integer coefficients ($\Z$), the final answer requires the use of
$\sqrt{3}$ and $\sqrt[3]{2}$.  These radicals are needed to express
the zeroes of $x^3-2$.  Sometimes, one gets lucky and all the zeroes
lie in the ground field, \eg,
\[
\int\frac{dx}{x^3 -2x^2 - x +2} = \int\frac{dx}{(x-1)(x+1)(x-2)}
 = -\frac{1}{2} \log (x-1) + \frac{1}{6}\log(x+1) + \frac{1}{3} \log(x-2),
\]
but often new algebraic quantities must be added.

Now consider a routine called {\sf solve} that returns one of the
zeroes of a polynomial in one variable.  Given $x^2-5$, it should
return $\sqrt{5}$.  But how should the symbol $\sqrt{5}$ be
interpreted, \ie, how does one know that $(\sqrt{5})^2 = 5$, and is
$\sqrt{5}$ equal to $2.2361$ or $-2.2361$?  Furthermore, after {\sf
solve} has been asked to solve $x^2-5$, what is returned as the
solution of  $x^2-20$?
Either $\sqrt{20}$ or $2 \sqrt{5}$ could be correct, although the
latter is probably better than the former.

These issues are addressed in Weyl by using \keyi{domains}.  Domains are
collections of objects, usually with operations among the objects, that
possess some underlying mathematical organization.  For instance, the
set of rational integers $\Z$ is a domain, as is the set of rational
numbers $\Q$.  The polynomials $x^2-5$, $x^2-20$ and $x^2-x-1$ could
all be elements of the domain $\Q[x]$, the set of polynomials in $x$
with rational number coefficients.  Other common domains are the real
numbers, $\R$, and the complex numbers, $\C$.  On the other hand, we
rarely think of a set of numbers, \eg, $\{ 1492, 1776, 1917\}$, as
having any mathematical structure and being a domain. 

In the example above, {\sf solve} is first called on $x^2-5$,
which is an element of $\Q[x]$.  For the duration of this section, we
will use the notation $(x^2-5)_{\Q[x]}$ when it is necessary to
indicate the domain that contains an algebraic element.  When {\sf
solve} returns $\sqrt{5}$ it must also create a domain in which
$\sqrt{5}$ is an element.  This domain is $\Q[\alpha]/(\alpha^2-5)$,
the ring of polynomials in $\alpha$ with coefficients in $\Q$ reduced
modulo the relation $\alpha^2-5=0$.  Notice that $\alpha$ could
correspond to either of the zeroes of $x^2-5$, so {\sf solve} must
also set up an embedding of $\Q[\alpha]/(\alpha^2-5)$ into $\C$---that
is, a map that sends each element of $\Q[\alpha]/(\alpha^2-5)$ to a
complex number.  Since there is a natural embedding of $\Q$ into $\C$
it is only necessary to indicate the image of $\alpha$ in $\C$.

This final algebraic structure, along with the embedding into $\C$ is
abbreviated as $\Q[\sqrt{5}]$.  {\sf Solve} returns
$(\sqrt{5})_{\Q[\sqrt{5}]}$ thus conveying both the simplification
rules that $\sqrt{5}$ must obey and the embedding in $\C$.  Notice
that this information is associated with the domains, rather than
attached directly to $\sqrt{5}$.  We find this to be the
mathematically more natural approach.

The second invocation of {\sf solve} is interesting.  Is {\sf solve}
invoked on $(x^2-20)_{\Q[x]}$ or $(x^2-20)_{\Q[\sqrt{5}][x]}$?  It is
not possible to determine which is the case by just examining the
polynomial.  Some reference to the enclosing domain is needed.
If {\sf solve} is called on $(x^2-20)_{\Q[x]}$ then
$(\sqrt{20})_{\Q[\sqrt{20}]}$ is returned, just as in the previous
discussion.  When {\sf solve} is called on
$(x^2-20)_{\Q[\sqrt{5}][x]}$, however, the coefficient field does not
need to be extended, and $(2\sqrt{5})_{\Q[\sqrt{5}]}$ is returned.
This eliminates the potential future need of determining the
relationship between $\sqrt{5}$ and $\sqrt{20}$.


\medskip
Weyl provides mechanisms to support multiple distinct but isomorphic
domains in the same computation.  Why would one want to have two copies
of $\Z$, say, present during a computation?  Algebraic extensions
provide one motivation.  The three zeroes of $x^3 -2 $ are 
\[
\sqrt[3]{2},\quad \frac{1+\sqrt{-3}}{2} \sqrt[3]{2} = \zeta_3
\sqrt[3]{2}, \quad \zeta_3^2\sqrt[3]{2}.
\]
Each generates an algebraic extension of $\Q$ of degree $3$.  Since
each has a different embedding in the complex numbers, they are, in
fact, different algebraic extensions of $\Q$.  However, algebraically
each extension is a simple cubic extension of the form
$\Q[\alpha](\alpha^3 -2)$: \ie,
\[
\begin{eqalign}
\Q[\sqrt[3]{2}] & = \Q[\alpha]/(\alpha^3 - 2), \\
\Q[\zeta_3 \sqrt[3]{2}] & = \Q[\beta]/(\beta^3 - 2), \\
\Q[\zeta_3^2 \sqrt[3]{2}] & = \Q[\gamma]/(\gamma^3 - 2).
\end{eqalign}
\]
That is, the elements of these fields can be represented as univariate
polynomials modulo an irreducible univariate polynomial. Algebraically
these fields are isomorphic.  The only difference between these three
fields is their embedding in $\C$, \ie,
\[
\begin{eqalign}
\alpha & \longrightarrow 1.259921, \\
\beta & \longrightarrow -0.629960 + 1.091124 i, \\
\gamma & \longrightarrow -0.629960 - 1.091124 i.
\end{eqalign}
\]
Thus we need to be able to represent distinct fields that are
algebraically isomorphic.  Only when their embedding into the complex
numbers is specified is it possible to distinguish them. 

\medskip
In addition, having multiple isomorphic domains allows us to check
``functorial'' code for missing or incorrect conversions while still
using simple examples.  For instance, consider the representation of
polynomials over the integers, $\Z[x]$.  Internally, such polynomials
may be represented as lists of exponent-coefficient pairs:
\[
2x^3+5 \approx ((3, 2), (0, 5)).
\]
The coefficients are elements of $\Z$ as are the exponents.  For most
operations, the exponents never come in contact with the coefficients.
For instance, when multiplying polynomials, the coefficients are
multiplied with each other, and the exponents are compared and added
to each other, but there are no operations that mix
exponents and coefficients.

Many implementations use the same representation for both coefficients
and exponents.  Instead, we suggest that there should be a domain of
integers for the exponents in addition to the domain associated with
the coefficients.  More generally, $k[x]$ would have two subdomains,
$k$ for the coefficients and $\Z$ for the exponents.

The value of this approach is apparent when examining polynomial
differentiation.  In that algorithm an exponent must be multiplied
with coefficient.  By having two isomorphic, but different, copies of
$\Z$ we are forced to coerce an exponent into the domain of the
coefficients even though it may not initially appear to be necessary
for polynomials in $\Z[x]$, thus catching the need for coercion
immediately. 

