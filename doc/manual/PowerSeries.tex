% -*- Mode: LaTeX -*-
% $Id: PowerSeries.tex,v 1.1 1995/01/30 02:02:08 rz Exp $
\chapter{Truncated Power Series}
\label{Series:Chap}

{\em Formal power series} are infinite degree
polynomials, where the arithmetic operations are defined
formally using the same rules as a polynomial arithmetic.
Thus
\[
f(x) = \sum_{0\le i} a_i x^i = a_0 + a_1 x + a_2 x^2 + \cdots,
\]
is a power series, and we have
\[
\left(\sum_{0\le i} a_i x^i\right) \times 
\left(\sum_{0\le i} b_i x^i\right) =
\sum_{0 \le i} \left(\sum_{0\le j \le i} a_j b_{i-j}\right) x^i .
\]
Some operations are possible with formal power series that are not
possible with polynomials. For instance, if the leading coefficient of
a formal power series is invertible then the formal power series has a
reciprocal.
  
Rings of formal power series are denoted using a similar notation to
that used for polynomials except that the brackets are doubled. Thus,
$\Z[[x]]$ and $\Q[t][[x]]$ represent rings of formal power series in
$x$ whose coefficients are rational integers and polynomials with
rational number coefficients. The quotient field of a ring of formal
power series consists of formal power series whose leading term can
have negative order. For example, we have
\[
\left(x + x^2 + \cdots \right)^{-1} = \frac{1}{x} - 1 + \cdots.
\]
The field of formal power series is denoted using doubled
parenthesis---similar adaptation to the notation for the field of
rational functions.  Thus, $R((x))$ is the field of formal power
series with real number coefficients.

In mathematics one also deals with ``regular'' power series, which are
formal power series that are convergent in some sense.  Thus, while
\[
1 + 2x + 3! x^2 + 4!x^3 + \cdots,
\]
is a perfectly reasonable formal power series, but it is not
convergent for any value of $x$.  When one represents physical
quantities, convergent power series are usually more interesting,
since they can actually be evaluated.  However, arithmetic operations
with power series may shrink their radius of convergence, and keeping
track of the radius of convergence throughout the computation can be
quite difficult.  Instead, it is often more convenient to deal with
formal power series and leave the convergence issues to later.
  
While Weyl does not have a representation for the power series (either
convergent or formal), it does provide tools for manipulating
truncated power series. A truncated power series only contains those
monomials of a formal power series that have degree less than some
bound. Thus, while the function $\sin x$ has the infinite power series
expansion
\[
\sin x = x - \frac{x^3}{6} + \frac{x^5}{120} - \frac{x^7}{5040} +
\cdots, 
\]
its truncated power series expansion is
\begin{code}
> (setq sinx (taylor (sin qx) domain 5))
x - (1/6)x^3 + (1/120)x^5 - (1/5040)x^7 + o(x^5)
\end{code}

\noindent
The small ``o''  notation indicates that all terms
of order greater than $5$ have been ignored.  The argument
domain is some domain of truncated power series. 
  
The following sections describe the tools provided by Weyl for
computing with formal power series.
  
\section{Creating Truncated Power Series Domains}

Truncate power series rings and fields can be created using the
function \keylisp{get-tpower-series}.

\begin{functiondef}{get-tpower-series-domain}{coefficient-domain variable}
{\em Coefficient-domain} must be a ring.  If the {\em
coefficient-domain} is also a field, the constructed truncated power
series domain will be a field.  {\em variable} should be coercible
into a general expression variable.  (Notice that this is somewhat
more restricted than what is permissible for polynomials.  At some
point in the future this restriction will be lifted.)  The variable
indicated will be the generator of the truncated power series.
\end{functiondef}
  
The following form creates the ring of power series
with rational integer coefficients
\begin{code}
> (weyli::get-tpower-series-domain (get-rational-integers) 'x)
Z[[x]]
\end{code}
  
\noindent
Notice that this domain is actually a ring and not a field is
indicated by the square brackets. When a power series domain is
created over a field, round brackets are used:
 
\begin{code}
> (setq domain (get-tpower-series-domain (get-rational-numbers) 'x))
Q((x))
\end{code}
  
\section{Truncated Power Series Operators}
  
The simplest way to actually create a truncated power series is to use
the Taylor series expansion function:
  
\begin{functiondef}{taylor}{expr tps-domain order}
Returns the power expansion of the general expression
{\em expr} as a truncated power series of order order in the
domain {\em tps-domain}. The variable of expansion will be
the power series variable of {\em tps-domain}.
\end{functiondef}
  
For instance, to compute the power series expansion of $\sin x$ one
could type
\begin{code}  
> (setq sinx (taylor (sin 'x) domain 8))
x - (1/6)x^3 + (1/120)x^5 - (1/5040)x^7 + (1/362880)x^9 + O(x^10)
\end{code}

The \keylisp{taylor} function also works with more complex functions.
\begin{code}  
> (setq temp1 (taylor (cos (sin 'x)) d 5))
1 - (1/2)x^2 + (5/24)x^4 + o(x^5)
  
> (setq temp2 (taylor (sin (sin 'x)) d 5))
x - (1/3)x^3 + (1/10)x^5 + o(x^5)
\end{code}
  
Arithmetic operations can also be performed with power series
expressions:
\begin{code}
> (+ (* temp1 temp1) (expt temp2 2))
1 + o(x^5)
\end{code}
  
Although \keylisp{taylor} returns a truncated power series
approximation to the power series expansion of the expression, if the
coefficient arithmetic is exact, then every coefficient returned will
be exact. So, in this case up to order $5$, the only non-zero term in
the power series expansion is $1$.
  
Truncated power series can be manipulated with the usual arithmetic
operations. For binary operations, the order of the result may be
reduced to ensure that only the known coefficients are advertised as
known.  For instance, although {\sf temp2} has order $5$ its sum with a
truncated power series of order $3$ is only of order $3$.
\begin{code}  
> (+ temp2 (taylor (sin 'x) d 3))  
2x - (1/2)x^3 + o(x^3)
\end{code}
  
\noindent
Notice also, the even unary operations can yield results with
different orders, \viz,
\begin{code}
> (* temp2 temp2)  
x^2 - (2/3)x^4 + (14/45)x^6 + o(x^6)
  
> (expt temp2 -3)
x^(-3) + x^(-1) + (11/30)x + o(x)
\end{code}
  
\noindent
Nonetheless, when the \keylisp{taylor} function is requested to
compute the power series expansion to a given order, it computes the
internal subexpressions to a sufficiently high order, to ensure that
the correct coefficients are computed.
\begin{code}  
> (expt (- temp1 1) -3)
(-8)x^(-6) - 10x^(-4) + o(x^(-3))
  
> (taylor (expt (- (cos (sin 'x)) 1) -3) d 5)
(-8)x^(-6) - 10x^(-4) - (88/15)x^(-2) - 16783/7560 - (32267/50400)x^2
 - (1045753/6652800)x^4 + o(x^5)
\end{code}
  
The power series package can also deal with power
series with fractional exponents. For instance,
\begin{code}  
> (expt (taylor (sin qx) d 5) -1/3)
x^(1/3) - (1/18)x^(7/3) - (1/3240)x^(13/3) + o(x^(13/3))
\end{code}
  
One power series can be substituted into another using the substitute
function. In addition, derivatives of power series can also be
computed using \keylisp{deriv}.
  
\begin{functiondef}{substitute}{value var tps}
Returns the truncated power series computed by substituting
the truncated power series value for the ring variable
in {\em tps}. The general expression variable var should be
equal to the ring variable of the domain of {\em tps}.
\end{functiondef}
  
\begin{functiondef}{deriv}{tps \rest vars}
The derivative of the {\em tps} with respect to each variable in {\em
vars} is computed successively and the final truncated power series is
returned.
\end{functiondef}
  
For instance, another way to compute the first few
terms of the power series of $\sin \sin x$ is to substitute the power
series of $\sin x$ into itself.
\begin{code}
> (substitute (taylor (sin 'x) d 5) qx (taylor (sin 'x) d 5))
x - (1/3)x^3 + (1/10)x^5 + o(x^5)
\end{code}
  
As mentioned, before the order of the resulting power series expansion
may be less than the order of arguments.
\begin{code}
> (deriv * 'x)
1 - x^2 + (1/2)x^4 + o(x^4)
\end{code}
  
One of the operations that can be performed with power series that
cannot be performed with polynomials is the reversion. That is, given
a power series for $f(x)$, we can compute the power series for
$f^{-1}(x)$.  This computation is performed by the function revert
  
\begin{functiondef}{revert}{tps}
Returns the truncated power series that is the reversion
of {\em tps}.
\end{functiondef}
  
For instance, given the power series expansion for $\sin x$ 
\begin{code}
> (setq sinx (taylor (sin 'x) domain 10))
x - (1/6)x^3 + (1/120)x^5 - (1/5040)x^7 + (1/362880)x^9 + O(x^10)
\end{code}

\noindent
we can compute the reversion of $\sin x$ as follows:
\begin{code}
> (reversion sinx)
x + (1/6)x^3 + (3/40)x^5 + (5/112)x^7 + (35/1152)x^9 + O(x^10)
\end{code}

\noindent
If we substitute the reversion of {\sf sinx} obtained above
for $x$ in {\sf sinx}, we should get $x$, as expected.:
\begin{code}
> (substitute (reversion sinx) 'x sinx)
x + O(x^10)
\end{code}
  
\begin{functiondef}{solve-diff-eqn}{diff-eqn coef-ring var order init-list}
Solves the differential equation {\em diff-eqn}
given in the form of a general expression. The solution
is given as a truncated power series of order {\em order}
with coefficients in the {\em coef-ring} with {\em var} taken as
the ring variable. The {\em init-list} is the list of initial
values of length equal to the order of the {\em diff-eqn}.
\end{functiondef}
  
The following examples illustrate how to obtain power
series solutions to ordinary differential equations
in one variable:
\begin{code}
> (setq y (funct y 'x))
y(x)

> (setq y0 (deriv y 'x) y00 (deriv y 'x 'x))
y_{0}(x)
  
> (setq diff-eqn1 (+ y y0 (* -1 qx qx))
y(x) + y_{0}(x) - x^2
  
> (solve-diff-eqn diff-eqn1 (get-rational-numbers) 'x 6 '(1))
1 - x + (1/2)x^2 + (1/6)x^3 - (1/24)x^4 + (1/120)x^5 - (1/720)x^6 + o(x^6)
  
> (setq diff-eqn2 (- y00 y))
  
-1 (y(x)) + y_{00}(x)
  
> (solve-diff-eqn diff-eqn2 (get-rational-numbers) 'x 5 '(c0 c1))
  
c0 + c1x + ((c0)/(2))x^2 + ((c1)/(6))x^3 + ((c0)/(24))x^4 + ((c1)/(120))x^5 + o(x^5)
\end{code}  


\section{Truncated Power Series Internals}
  
{\em This section describes the internal representation used to
implement truncated power series and is only of interest to those who
want to extend the package.}

  
Truncated power series are currently represented by a single Lisp
class, \keylisp{weyli::tpower-series}, which is a representation in
which all of the coefficients are enumerated in an array. Generative
representations are not currently available.
  
\subsection{Enumerated Truncate Power Series}
  
This class, currently the only representation for power series,
maintains an array of known coefficients and is therefore useful only
for power series with a finite and typically small number of nonzero
coefficients.  The power series may have an infinite order (in which
case the behavior is nearly identical to that of polynomials) but in
such cases the largest nonzero term must have a finite exponent. This
class will perform best when used with dense power series (\ie, power
series in which the sequences of exponent numerators of nonzero terms
do not have large skips).
  
Truncated power series in Weyl are a little more complex than the
proceeding discussion might indicate.  In particular, we permit
truncated power series to have negative and fractional exponents in
order to allow for certain algebraic operations on power series. For
instance, we would like to represent
\[
\sqrt{x+x^2} = x^{1/2} - \frac{x^{3/2}}{2} + \frac{x^{5/2}}{8} +
\cdots,
\]
and the reciprocal of the function 
\[
\frac{1}{\sqrt{x+x^2}} = x^{-1/2} - \frac{x^{1/2}}{2} + \frac{3}{8}
x^{3/2}
- \frac{5}{16} x^{5/2} + \cdots.
\]

To deal with these issues the truncated power series representation in
Weyl has four components. The most obvious is the vector with
coefficients of the power series. The \keyi{branch order} of the power series
is the least common multiple of the denominators of the exponents that
arise in the power series. Thus in the pervious two examples the
branch order is $2$.
  
The \keyi{valence} of the power series is the degree of the term with
nonzero coefficient with least degree when multiplied by the branch
order. In the previous two examples, the valences are $1$ and $-1$
respectively.

Finally, the product of the branch order and the degree of the highest
term retained in the truncated power series is called the \keyi{order}
of the power series.  Notice that these definitions have been chosen
so that all of the parameters are integers.
  
The following routines can be used to access the parameters of a
truncated power series.
  
\begin{functiondef}{branch-order}{tps}
Returns the branching order of {\em tps} as a
lisp integer. This number is the greatest common divisor
of all of the denominators of the exponents in the truncated
power series.
\end{functiondef}
  
\begin{functiondef}{order}{tps}
Returns the order of {\em tps} as a Lisp integer.
This number is the product of the greatest exponent
in the truncated power series and the branch order of
the power series.
\end{functiondef}
  
\begin{functiondef}{valence}{tps}
Returns the valence of {\em tps} as a Lisp integer.
This number is the product of the least exponent in
the truncated power series and the branch order of the
power series.
\end{functiondef}
  
The following examples illustrate the use of these routines:
\begin{code}
> (setq temp (taylor (expt (+ 'x (* 'x 'x)) -1/2) d 2))
x^(-1/2) - (1/2)x^(1/2) + (3/8)x^(3/2) - (5/16)x^(5/2) + o(x^(5/2))
\end{code}

\noindent
{\sf temp} is the power series of the expression given above. It has
both negative and fraction exponents.

\begin{code}
> (describe temp)
x^(-1/2) - (1/2)x^(1/2) + (3/8)x^(3/2) - (5/16)x^(5/2) + o(x^(5/2))
  is an instance of the class WEYLI::TPOWER-SERIES:
The following slots have allocation :INSTANCE:
DOMAIN          Q((x))
VALENCE         -1
BRANCH-ORDER    2
ORDER           5
COEFFS          #(1 0 -1/2 0 3/8 0 -5/16)
\end{code}
  
Notice that the branch order, valence and order are all rational
integers.
  
The branch order of the power series can be modified by using setf and
the branch-order accessor. This operation increases the size of the
coefficient array of the power series and adjusts the order and
valence appropriately.  This operation is used internally to before
combining two power series with different branching orders.
\begin{code}
> (setf (weyli::branch-order temp) 4)
x^(-1/2) - (1/2)x^(1/2) + (3/8)x^(3/2) - (5/16)x^(5/2) + o(x^(5/2))
\end{code}

\noindent
Notice that the outward appearance of the power series has not
changed, even though the branching order has been increased.  However,
the internal form has changed.
\begin{code}
> (describe temp)
x^(-1/2) - (1/2)x^(1/2) + (3/8)x^(3/2) - (5/16)x^(5/2) + o(x^(5/2))
  is an instance of the class WEYLI::TPOWER-SERIES:
The following slots have allocation :INSTANCE:
DOMAIN          Q((x))
VALENCE         -2
BRANCH-ORDER    4
ORDER           10
COEFFS          #(1 0 0 0 -1/2 0 0 0 3/8 0 ...)
\end{code}

When writing low level code using truncated power series, it is more
convenient not to use the accessors for the slots in a truncated power
series, but instead to have variables bound to the various pieces.
This is done by the following special form. 

\begin{functiondef}{weyli::with-tpower-series}{((var$_1$ tps$_1$)
$\ldots$ (var$_1$ tps$_1$)) \body body}
For each truncated power series provided, this form binds four new
variables, with names based on {\em var}$_i$, one for each of the
slots in {\em tps}$_i$.  The names of these slots are 
{\sf {\em var}$_i$-bo} for the branching order,
{\sf {\em var}$_i$-val} for the valence, 
{\sf {\em var}$_i$-order} for the order and 
{\sf {\em var}$_i$-coeffs} for the coefficient vector.
\end{functiondef}

{\em Similarly, the following function need not be here but should be a
setf method on {\sf order}}

\begin{functiondef}{weyli::truncate-order}{tps integer}
Truncates the order of {\em tps} so that the largest exponent is no
greater than {\em integer}.  If {\em integer} is greater than or equal
to the current truncation order, then {\em tps} is returned unaltered.
\end{functiondef}

