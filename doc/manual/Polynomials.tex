% -*- Mode: LaTeX -*-
% $Id: Polynomials.tex,v 1.3 1995/04/18 18:36:00 rz Exp $
\chapter{Polynomial Rings}
\label{Polynomial:Chap}

Polynomial rings are domains that consist of polynomials in some
number of variables over a coefficient domain, \eg, $\Z[x, y]$,
$\R[x]$, $\Q((t))[x]$.  In these three cases the coefficient domains
are the rational integers $\Z$, the real numbers $\R$ and the ring of powerseries
in $t$, respectively.  Polynomial rings are created using the function
\keylisp{get-polynomial-ring}.

\begin{genericdef}{get-polynomial-ring}{coefficient-domain list-of-variables}
{\em Coefficient-domain} is must be a ring.  {\em List-of-variables}
is a list of general expressions or other objects that can be coerced
into general 
expressions.   The resulting polynomial ring will have as independent
variables the ``kernels'' that appear in {\em list-of-variables}.  In the
most common case, {\em list-of-variables} is a list of variables.
However, it is occaisionally useful to include in {\em
list-of-variables} and expression like $x^3+y$, in which case the
variables of the ring will include $x$ and $y$.  No
check is made to ensure that the elements of {\em list-of-variables}
are algebraically independent.
\end{genericdef}

For instance, the rings $\Z[x,y]$ and $\R[x]$ can be constructed as
follows.
\begin{code}
> (setq ring1 (get-polynomial-ring (get-rational-integers) '(x y)))
Z[x, y]

> (setq ring2 (get-polynomial-ring (get-real-numbers) '(x)))
R[x]
\end{code}
Recall that the ``{\sf get-}'' functions endeavor to return a ring
that already exists that matches the requirements of its
arguments.  Thus in the following example, despite the fact that a general
expression is returned instead of a Lisp expression the same ring is
returned as before.
\begin{code}
> (setq ring3 (get-polynomial-ring (get-real-numbers) 
                                   (list (coerce 'x *general*))))
R[x]


> (eql ring2 ring3)
T
\end{code}

Three different rings that are isomorphic to $\Z[x, y]$ can be created
using Weyl.  In addition, to {\sf ring1}, we can generate the rings
$\Z[y, x]$ and $\Z[x][y]$ and $\Z[y][x]$:
\begin{code}
> (setq ring4 (get-polynomial-ring (get-rational-integers) '(y x)))
Z[y, x]

> (setq ring5 (get-polynomial-ring 
                 (get-polynomial-ring (get-rational-integers) '(x))
                 '(y)))
Z[x][y]

> (setq ring6 (get-polynomial-ring 
                 (get-polynomial-ring (get-rational-integers) '(y))
                 '(x)))
Z[y][x]
\end{code}
Each of these rings are different and in general their elements cannot
be combined using arithmetic operations.  Notice that in the last two
examples, the coefficient domain of the polynomial ring is itself a
polynomial ring. 

Notice that following form also specifies the ring {\sf ring1}.  This
illustrates that if the list of variables to
\keylisp{get-polynomial-ring} include complex expressions, Weyl will
determine the kernels in the expressions and use them.
\begin{code}
> (setq ring7 (get-polynomial-ring (get-rational-integers) (list (+ 'x 'y))))
Z[x, y]

> (eql ring1 ring7)
T
\end{code}

As an example of creating a ring with algebraically dependedent,
independent variables we have the following.  This is really a bug and
should be fixed in the future. 
\begin{code}
> (get-polynomial-ring (get-rational-integers) (list 'x (expt 'x 1/3)))
Z[x, x^1/3]
\end{code}

\section{Information About Variables}

The variables of a polynomial ring have three different
representations: as a polynomial, as symbol, or as a {\sl variable
index\/}.  The simplest representation is as a symbol.  These are the
symbols passed to \keylisp{get-polynomial-ring} and are the only mechanism the
user has to influence the external printed representation of the
variable.  Internal to the polynomial package, there are (Lisp) integers
associated with variables.  These integers are used as indices into
tables and to indicate the ordering of variables in multivariate
polynomials.  Finally, a variable can be represented as a polynomial.

Regardless of the internal representation used for polynomials, there
is an integer associated with each variable of the polynomial.  These
numbers are used as an index into all data structures that hold
additional information about the variable.  The index corresponding to
a variable is obtained by the generic function \keylisp{variable-index}.
The variable corresponding to an order number is \keylisp{variable-symbol}.

\begin{functiondef}{variable-index}{domain variable}
Returns the variable index corresponding to {\em variable}.  {\em
Variable} can be either a symbol or a polynomial.
\end{functiondef}

\begin{functiondef}{variable-symbol}{domain variable}
Returns the variable symbol corresponding to {\em variable\/}.  {\em
Variable} can be either an integer or a polynomial.
\end{functiondef}

There is a property list associated with each variable in a polynomial
ring.  This property list is ``ring'' specific and not global.  The
ring property list is accessed using the generic function
\keylisp{get-variable-property}.  Properties can modified using
\keylisp{setf}, as with normal property lists.

\begin{functiondef}{get-variable-property}{domain variable property}
Returns a {\em property} property of {\em variable\/}.
\end{functiondef}

In contrast with normal mathematical usage, the polynomial rings of
Weyl can be modified in a limited fashion after they have been
created.  In particular, it is possible to add variables to a
polynomial ring, although removing variables is not allowed.  Variables
are added to the end of the ring's list of variables so that they
will be less ``main'' than any of the existing variables, and thus only
minimal changes to existing polynomial representations will be needed.

\begin{methoddef}{add-new-variable}{ring var}
This method applies when {\em ring} is a \keylisp{polynomial-ring}, or when its
class inherits from \keylisp{polynomial-ring}.
The second argument, {\em var} is a variable in the general
representation, or something that can be coerced into a general
expression.  This function 
modifies {\em ring} to have an addition variable.
\end{methoddef}

The behavior of this routine is illustrated by the following examples.
First, we create a polynomial ring with two variables, and then add a
third variable to it.
\begin{code}  
> (setq R (get-polynomial-ring (get-rational-numbers) '(x y)))
Q[x, y]
  
> (add-new-variable R 'z)
Q[x, y, z]
\end{code}

\noindent
Now try adding a more complex expression to the ring.  In this case,
\keylisp{add-new-variable} determines that {\sf w} and {\sf sin(z+x)}
are the only new kernels and adds them.
\begin{code} 
> (add-new-variable R (expt (+ 'x 'w (sin (+ 'x 'z))) 3))
Q[x, y, z, w, sin(z + x)]
\end{code}

\section{Polynomial Arithmetic}
\label{Poly:Arith:Sec}
  
The simplest way to create a polynomial is to coerce a general
expression into a polynomial ring. For instance, if {\sf R} is the
ring $\Q[x, y]$,  as defined above, we could proceed as follows.  First,
we create a polynomial in $x$ and $y$ as general expressions.
\begin{code}
> (expt (+ 'x 'y) 3)
(y + x)^3
\end{code}
  
\noindent
Notice that the expressions is not expanded. Next, we coerce the
expression into the polynomial ring {\sf r}.
\begin{code}
> (coerce (expt (+ 'x 'y) 3) R)
x^3 + 3 y x^2 + 3 y^2 x + y^3
\end{code}
  
This time the expression is expanded. When represented as an element
of a polynomial ring, as opposed to being general expressions,
polynomials are represented using a canonical representation (as
described in \sectref{Poly:Structure:Sec}). This canonical
representation expresses polynomials as a sum of terms. This is easily
seen in the following example.
\begin{code}
> (setq p (+ (* (- 'x 'y) (+ 'x 'y)) (expt 'y 2)))
(-1 y + x) (y + x) + y^2
\end{code}

General expressions can represent polynomials as products of sums, and
thus do not have a canonical representation for the polynomial. When
the expression is coerced into a polynomial ring, the products are
expanded to produce a simple form for the expression.
\begin{code}  
> (coerce p R)
x^2
\end{code}


\section{Polynomial Operators}
\label{Operate:Poly:Sec}

\begin{functiondef}{scalar?}{polynomial}
The argument of this function is expected to be an element of a
polynomial domain.  If the argument is an element of the coefficient
field this function returns {\sf T} otherwise it returns {\sf nil}
\end{functiondef}

\begin{functiondef}{degree}{polynomial var}
Returns the degree of {\em polynomial} in the variable {\em var} as a
lisp integer.
\end{functiondef}

The usual arithmetic operations including \keylisp{plus},
\keylisp{minus}, \keylisp{difference}, \keylisp{times},
\keylisp{quotient}, \keylisp{recip} and \keylisp{expt}, can be applied
to polynomials.  For polynomial rings, \keylisp{expt} is restricted to
integer exponents.

\begin{functiondef}{remainder}{x y}
Computes the polynomial pseudo-remainder of {\em x} and {\em y}.
\end{functiondef}

\begin{functiondef}{gcd}{x y}
Computes the polynomial greatest common divisor of {\em x} and {\em
y}.  The variable \keylisp{weyli::poly-gcd-algorithm} is bound to the
particular GCD algorithm to be used.
\end{functiondef}

\begin{functiondef}{partial-deriv}{polynomial var}
This function takes the partial derivative of {\em polynomial} with
respect to {\em var\/}.  {\em var} is actually a polynomial, not the
lisp object which is the variable
\end{functiondef}

\begin{functiondef}{deriv}{polynomial var \rest vars}
Computes the derivative of {\em polynomial} with respect to {\em var}.
This is done for {\em var} and each element of {\em var\/}.  The
variables can either be elements of \keylisp{*general*} or of a
polynomial domain.
\end{functiondef}

\begin{functiondef}{coefficient}{polynomial var \optional{} degree}
Compute the coefficient of the monomial in {\em var} of order {\em
degree} in {\em polynomial}.  {\em degree} defaults to $1$.
\end{functiondef}

\begin{functiondef}{substitute}{value var polynomial}
Substitutes {\em value} for each occurrence of {\em var} in {\em
polynomial\/}.  If {\em value} is a list, it is interpreted as a set
of values to be substituted in parallel for the variables in {\em
var\/}.  The values being substituted must be either elements of the
domain of {\em polynomial} or its coefficient domain.
\end{functiondef}

\begin{functiondef}{list-of-variables}{polynomial \optional{} vars}
Returns a list of the variables that actually appear in {\em polynomial}.
If provided {\em vars} must be a list.  The variables that appear in
{\em polynomial} are added to {\em vars}.
\end{functiondef}

\begin{functiondef}{interpolate}{vars points values \optional degrees}

Uses the information provided to produce a polynomial
whose values at each of the specified points is the
corresponding value.  The variables of the polynomials
are indicated by {\em vars}.  For a univariate interpolation,
{\em vars} should be a single variable and {\em points} and {\em values}
should be simple lists.  In this case the degree of the
interpolated polynomial will be less than the length
of {\em points}. For multivariate polynomials, {\em points}
and {\em values} are lists of lists, where each of the sub-lists
has the same length as the number of {\em variables}. The
degree bounds for the multivariate polynomial can be
specified by the {\em degrees} argument. It can be either
a list of the maximum degrees in each of the variables,
or the symbols {\sf :total} or {\sf :maximum}.  In either of these
two cases, {\sf interpolate} uses the maximum degree bound
that yields a polynomial with fewer terms than the number
of points supplied. When {\sf :maximum} is specified it is
assumed that each variable can attain the maximum degree
independently. So for two variables with a maximum of
10 terms, the possible terms of the returned polynomial
are:
\[
1, x, y, xy, x^2, x^2 y , x y^2.
\]
When {\sf :total} is specified, with the same parameters, the possible
terms are: 
\[
1, x, y, x^2, xy, y^2, x^3, x^2y, xy^2, y^3.
\]
\end{functiondef}

\begin{functiondef}{interpolate}{domain Bp degree-bounds}

The {\em domain} should be \keylisp{multivariate-polynomial-ring}.
The {\em Bp }is the name of the function representing
the black box, that is, the function should return the
value of the polynomial at the requested point. The
{\em degree-bounds} is the list of degree bounds for
the multivariate polynomial in each of the ring variables
of the domain. The function will return the multivariate
polynomial computed using the probabilistic sparse multivariate
interpolation algorithm given in the book by Zippel.
\end{functiondef}

\section{Differential Rings}
\label{DifferentialRings:Sec}

Differential rings are polynomial rings that have {\em derivations\/}.  A
derivation of a differential ring $R\langle x\rangle$ is a map $\delta$ from
$R\langle x\rangle$ to $R\langle x\rangle$ which has the following properties:
\[
\begin{eqalign}
  \delta (ap+bq) &= a \,\delta p + b \,\delta q\\ \delta a &= 0\\ \delta
(p q) &= p\,\delta q + q \,\delta p
\end{eqalign}
\]
where $a, b \in R$ and $p$ and $q$ are not elements of $R$.

\begin{functiondef}{get-differential-ring}{domain variables}
Return a differential ring whose coefficients lie in {\em domain} and
which contains the variables {\em variables}.  For example:
\begin{code}
> (setq R (get-differential-ring (get-rational-integers) '(x y)))
Z<x, y>
\end{code}
\end{functiondef}

\begin{functiondef}{derivation}{p}
Returns the derivative of the differential polynomial {\em p\/}.  For
example:
\begin{code}
> (setq p (expt (+ (coerce 'x r) (coerce 'y r)) 2))
x^2 + 2 y x + y^2

> (deriv p)
(2 D{x, 1} + 2 D{y, 1}) x + ((2 D{x, 1} + 2 D{y, 1}) y)
\end{code}
\end{functiondef}

\begin{functiondef}{variable-derivation}{domain variable}
Returns the derivative of {\em variable\/}. The derivative can be
either an element of {\em domain}, or {\em :generate} which indicates
that the derivative will be a new variable which is yet to be created.

The derivative of a variable can be set using {\em setf}.
\end{functiondef}

\section{Structure Types for Polynomials}
\label{Poly:Structure:Sec}

{\em This section discusses the internal structures used represent
polynomials. This material is only of value for those problems that
require especially efficient access to the low level polynomial
primitives in Weyl. }

\begin{figure}
\begin{center}
\BoxedEPSF{PolyClasses.eps scaled 500}
\end{center}
\caption{Polynomial Class Structure\label{PolyElt:Fig}}
\end{figure}

Polynomials are represented using one of three different
structure types. The class structure of these types is given in
\figref{PolyElt:Fig}.  The simplest structure is only used by elements of
univariate polynomial rings and is called \keylisp{weyli::upolynomial}s.
Two different representations are provided for multivariate
polynomials.  The \keylisp{weyli::mpolynomial} structure type uses a
recursive structure, so polynomials can be views as univariate
polynomials with polynomial coefficients.  This is the classical
representation used by systems like Macsyma.  The
\keylisp{weyli::epolynomial} structure is represents polynomials as a
set of pairs of exponent vectors and coefficients.

In all three cases, instances of the the polynomial class include a
slot called form, in which the data representing the polynomial is
kept. These lower level data structures are what is actually passed
between low level polynomial routines. For univariate polynomials the
form slot contains a vector of the polynomial's coefficients.
Instances of \keylisp{weyli::mpolynomial} contain a recursive list
structure while instances of \keylisp{weyli::epolynomial} contain a
sorted list of the polynomial's monomials with non-zero
coefficients. For efficiency the internal code used by the more
complex algorithms uses these internal structures, not instances of
the polynomial classes. The algorithms themselves however, accept
their arguments and return values that are instances of the polynomial
classes. This approach couples maximum system flexibility with
efficient representations when needed.
  
Throughout the remainder of this section we will refer to the classes
of these polynomials without the ``{\sf weyli::}'' prefix for succinctness.

\subsection{Multivariate Polynomials}
\label{MultPoly:Rep:Sec}

There are two basic representations of polynomials, ``recursive'' and
``expanded.''  The difference between these two is illustrated by the
polynomial
\[
x^3 + x^2(y+3z^3) + xy^3 + y^4,
\]
which is given in recursive form, and by
\[
x^3 + x^2y + 3x^2 y^3 + xy^3 + y^4,
\]
in expanded form. Both of these forms express the polynomial as a sum
of terms. The expanded polynomial is a somewhat simpler representation
since it consists of a sum of monomials in all of the variables.
\sectref{ExpPoly:Rep:Sec} is devoted to this representation.
  
The recursive representation uses terms that are products of $x$ to
some power times a polynomial in the remaining variables. This
coefficient is then represented recursively as a sum of terms in $y$
with coefficients in the remaining variables. The details of the
representation also depend on the variable order chosen. In the
example given above, we have chosen the variable ordering $x$, $y$,
$z$.  If we had chosen $z$, $y$, $x$, the polynomial would have the
following form:
\[
z^3(3x^2) + (y^4 + y^3(x) + y(x^2) + x^3).
\]
  
Multivariate polynomials are implemented using three different levels
of structure. First, there is the \keylisp{mpolynomial} structure
type. 


\subsection{Expanded Polynomials}
\label{ExpPoly:Rep:Sec}

For some important algorithms, especial those of commutative algebra
that are based on the Gr\"{o}bner basis algorithm, it is convenient to
represent a polynomial as a sorted list of monomials.
  
As in the previous section we assume that we are
representing polynomials in the ring $k[x_1, \ldots, x_n]$, $k$ is
assumed to be a ring.

The monomials of an \keylisp{epolynomial} are represented by a
(simple) vector of $n+1$ elements. The first component of the vector
contains the coefficient of the monomial. The remaining components
contain the exponents of the variables, represented as Lisp fixnums.
  
Associated with each \keylisp{epolynomial} is a function that orders
the monomials. These functions can be provided by the user, but it is
usually preferable to let Weyl create them since Weyl's version is
usually the most efficient. A selection of ordering function can be
produced by the following function

\begin{functiondef}{weyli::make-comparison-fun}{n list-of-vars \key
(total? nil) (new? nil) (reverse? nil)} 
Returns a function that can be used to order the monomial structures
used by epolynomials.  {\em n} is the number of variables in the ring, or
equivalently, one more then the length of the vectors used to
represent the monomials.  {\em list-of-vars} is a list of the indices from $1$
to {\em n} in the order in which the corresponding exponents should be
examined. If {\em total?} is specified, then the total degree of the
monomial will be tested before the individual exponents.  {\sf
weyli::make-comparison-fun} tries to reuse functions that  were
previously created. If {\em new?} is specified, this is not done and a
new function is generated. This is usually only necessary for
debugging purposes. 
\end{functiondef}

When special orderings of the variables are not needed and one of the
standard variable orderings is used, the following function is often
more convenient.

\begin{functiondef}{weyli::get-comparison-fun}{n type}
Returns a function that can be used as to order the monomial
structures used by \keylisp{epolynomial}s.  $n$ is the number of
variables ion the ring, or equivalently, one more than the length of
the vectors used to represent the monomials.  {\em type} is one of the
keywords given the table below.
\begin{center}
\begin{tabular}{lp{4in}}
{\tt :lexical} & Lexical monomial ordering \\
{\tt :revlex} & Reverse lexical monomial ordering \\
{\tt :total-lexical} & Monomials are ordered by their total degree.
If they have the same total degree then the lexical ordering is used
to break ties. \\
{\tt :total-revlex} & Monomials are ordered by their total degree.
If they have the same total degree then the reverse  lexical ordering
is used to break ties. 
\end{tabular}
\end{center}
\end{functiondef}

Expanded polynomials are created using the following function.  Notice
that its argument pattern is slightly different from that used to
create univariate and recursive multivariate polynomials.  It is also
necessary to provide a term ordering function, such as one returned by
\keylisp{weyli::get-comparison-fun}.

\begin{functiondef}{weyli::make-epolynomial}{domain greater-fun
poly-form}
Generates an instance of an {\sf epolynomial} in {\em domain}.  The
terms of the resulting polynomials are sorted using {\em greater-fun}.
The argument {\em poly-form} can be any object that can be coerced
into the ring {\em domain}.
\end{functiondef}


\subsection{Univariate Polynomials}
\label{UniPoly:Rep:Sec}

Univariate polynomials are the simplest representation of polynomials
used by Weyl and are only intended for special, performance intensive
reasons.  Their existence should not be visible to the user.
Nonetheless, for certain algorithms significant performance
improvements can be achieved through their use.

