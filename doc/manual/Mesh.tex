% -*- Mode: LaTeX -*-
% $Id: Mesh.tex,v 1.11 1995/05/12 20:43:05 chew Exp $

\chapter{Meshing}
\label{Mesh:Chap}

Weyl provides for the creation and manipulation of unstructured
(triangular) meshes.  Currently, these facilities are limited to
2-dimensional (flat) meshes, but we plan to add facilities for meshing
curved surfaces (embedded in 3-dimensions) in the near future.  A
desirable feature of the meshing technique used by Weyl is that the
resulting meshes are mathematically guaranteed to exhibit the
following properties: 
\begin{itemize}

\item 
Boundaries, both external and internal, are respected.  

\item 
Element shapes are guaranteed.  All elements are triangles with angles
between 30 and 120 degrees (with the exception of badly shaped
elements that may be required by the specified boundary).

\item
Element density can be controlled, producing small elements in
``interesting'' areas and large elements elsewhere.
\end{itemize}

A mesh is implemented as a simplicial-complex with some additional
information attached.  This additional information includes a
subsidiary simplicial-complex representing the boundary.  Both
external boundaries and internal boundaries are supported.  Internal
boundaries are used, for instance, when analyzing an object made out
of more than one material or when analyzing cracks.

The desired shapes for the triangles of the mesh are specified by
giving angle-bounds for the minimum and maximum angles that are
allowed to appear in the mesh.

The mesh density is controlled by specifying a size-list.  Basically,
this is a simple (flattened) table associating names with sizes where
a size can be either a simple number (used as an upper bound on the
size) or a function of two numerical arguments that returns a number.
Both boundaries and subregions can have names that appear in a
size-list.

\section{Creating and Storing Meshes}
\label{CreateMesh:Sec}

Meshes are created by using \keylisp{make-mesh}, by using
\keylisp{make-mesh-from-file}, or by using individual meshing
functions as 
described below in \sectref{MeshingFunctions:Sec}.  Meshes can be
stored in a file by using \keylisp{write-mesh}.  The format of the
resulting file is described in \sectref{MeshFF:Sec}.

The two simplest ways to create a mesh are to use either
\keylisp{make-mesh} or \keylisp{make-mesh-from-file}.  {\sf Make-mesh}
provides syntactic sugar to make it easy to specify a mesh from within
a program.  {\sf Make-mesh-from-file} controls mesh creation via a
data file; the file's format is specified in \sectref{MeshFF:Sec}.
{\sf Make-mesh} is somewhat more versatile since it allows size bounds
that are functions while the data-file format allows only size bounds
that are numeric constants.  Note though that size bounds in the data
file can be overridden by size bounds specified in the call to
\keylisp{make-mesh-from-file}; thus, size bounds that are functions
can overrride the constant bounds in the file.

\begin{macrodef}{make-mesh}{\rm ({\em space} \key {\em angle-bounds
size-list}) . {\em boundary-and-region-specs}} 
Use the boundary and region specifcations to create a mesh in which the
elements satisfy the the requirements given by {\em angle-bounds} and
{\em size-list\/}.  
\end{macrodef}

The {\em angle-bounds} represent the minimum and maximum angles
allowed in the triangles of the mesh (note though that smaller angles
can occur in the final mesh if the specified boundary includes small
angles).  If {\em angle-bounds} are unspecified then the default
bounds are used; the default bounds lead to meshes in which all angles
are between $30\degree$ and $120\degree$.  If given, the {\em
angle-bounds} can be either a 2, 1, or 0 element list.  Each of the
elements in {\em angle-bounds} is an integer specifying an angle in
degrees.  If 2 elements are specified then the first is the lower
bound and the second is the upper bound.  If just one element is given
then this is the lower bound (note that this implies an upper bound of
$180\degree$ minus twice the lower bound).  If an empty list is given
then triangles are not checked for shape.  Looser bounds produce
meshes with fewer triangles.  If tighter bounds are used (tighter than
the default bounds) then there are 3 possible outcomes depending on
the problem:
\begin{enumerate}
\item the result can be a high-quality mesh satisfying the given
bounds, 
\item the result can be a mesh satisfying the given bounds, but it includes some small elements
in arbitrary places, or 
\item the meshing algorithm can fail to halt.
\end{enumerate}
An upper bound below about $100\degree$ is likely to lead to a failure
to halt.

The {\em size-list}, if provided, is an alternating list of names and
{\em size-bounds}.  Each name should correspond to the name of either a
boundary-edge or a region.  A {\em size-bound} is either a number or a
function of two numeric arguments that will produce a number.  The
size-bound following a name determines the maximum edge length that
will be allowed either along the boundary with that name or within the
region of that name.  In other words, a triangle (or a boundary-edge)
is replaced with smaller triangles (boundary-edges) if its size is
greater than the corresponding size-bound.  When the size-bound is a
function it is evaluated at the mean of the triangle's
(boundary-edge's) vertices.

The boundary and region specs are a list of specialized elements each
of which is a list starting with an identifier.  Currently, there are
three types of elements that can appear in the boundary and region
specs; the corresponding identifiers for these elements are: \keylisp{point},
\keylisp{region}, and \keylisp{boundary}.  The formats for these
elements are given below.  They have the following meanings only within
the scope of a \keylisp{make-mesh} form.

\begin{specialdef}{point}{name coord$_1$ \ldots\ coord$_k$}
Creates a named point that can be used (by name) in later \keylisp{boundary}
elements.  The number of coordinates must match the dimension of the
embedding {\em space} specified as the first argument to \keylisp{make-mesh}.
\end{specialdef}

\begin{specialdef}{region}{name coord$_1$ \ldots\ coord$_k$}
Specify a region to be meshed and give it a name.  The boundaries
(given by \keylisp{boundary} elements) divide the embedding {\em
space} (i.e., the first argument to \keylisp{make-mesh}) into one or
more regions.  A region-to-be-meshed is specified (and named) here by
giving a point within that region.  During the meshing process, all
triangles within this region have this same name.
\end{specialdef}

\begin{specialdef}{boundary}{name {
\rm (\key ({\em closed?} {\sf nil}) ({\em split} 0))} type . description}
Insert a boundary into the mesh.  Each segment of the boundary is
given the specified name.  At the moment, there are just two types
that can appear: {\sf LINE} and {\sf ARC}.  The description formats
for these two types are given below.  If {\sf :closed?} is non-null
then, in effect, the description is made to end at the same place it
started.  If {\sf :split} is an integer greater than zero then each
segment of the boundary is recursively split that number of times
before it is inserted in the mesh.  This can be used, for instance, to
make the initial region have a shape closer to the actual region.
\end{specialdef}

Care must be taken when an object (a hole for instance) is near a
curved boundary.  This is because all segments, even curved ones, are
represented as straight lines during the meshing process.  (Note
though that when a curved segment is split, the splitting point is
correctly placed on the curved path of the segment.)  Curved segments
that are not split into enough pieces can misbehave in the sense that
a nearby object can appear to intersect the segment or even to be on
the wrong side of the segment.

For {\sf LINE}, an arbitrary number of points can follow {\sf LINE} where
each point is specified as either a name (that appeared earlier in a
{\sf point} element) or a list of the form {\sf (PT coord$_1$ $\ldots$ coord$_k$)}.  The
boundary is formed by straight lines between adjacent points.

As an example, meshing of a simple square can be accomplished by the
following use of \keylisp{make-mesh}.  It produces a mesh with small
triangles in the interior of the square with slightly smaller
triangles along the boundary.

\begin{code}
(make-mesh ((get-euclidean-space 2) :size-list '(inside .2 edges .1))
                                    :angle-bounds '(10 120)) 
  (point origin 0 0) 
  (region inside 0.5 0.5)
  (boundary edges (:closed? T) line
    origin (pt 1 0) (pt 1 1) (pt 0 1)))
\end{code}

For {\sf ARC}, the description is an alternating list of points and
arc options.  The points are specified as described for {\sf LINE}.
The arc options are lists of keyword arguments.  The valid keywords
are {\sf :center}, {\sf :thru}, {\sf :radius}, {\sf :clockwise} (or
{\sf :cw}), and {\sf :counterclockwise} (or {\sf :ccw}).  Only one of
{\sf :center}, {\sf :thru}, or {\sf :radius} should be specified.
Also, at most one of {\sf :cw} or {\sf :ccw} should be given.  The arc
options and the two adjacent points (call them $a$ and $b$) determine
a circle where $a$ and $b$ are both on the circle.  For each such
circle there are two arcs between $a$ and $b$.  The one chosen is the
one that has the specified direction (either clockwise or
counterclockwise) going from $a$ to $b$; the default direction is
counterclockwise.  The different kinds of specifications are described
below.  In each instance that calls for a point, the point is either a
name or a list of the form {\sf (PT $\ldots$)} as described for {\sf
LINE}
\begin{itemize}
\item {\sf ()}.
If an empty list is used for the arc options this results in a
straight line between the two adjacent points.
\item {\sf (:center point)}.
The circle chosen is the one through the two adjacent points that has
the given center.  Note that it is possible, although very
undesirable, to specify an invalid center --- one that is not
equidistant from $a$ and $b$.
\item {\sf (:thru point)}.
The circle chosen is the one through the two adjacent points ($a$ and
$b$) and through the specified thru-point.
\item {\sf (:radius number)}.
The circle chosen is one of the two circles through $a$ and $b$ that
have the given radius.  If {\sf number} is positive then the circle
chosen is the one to the left of the vector from $a$ to $b$.  A
negatvie {\sf number} indicates the other circle.  If the radius is
too small (e.g., zero, for instance) then the circle with $a$ and $b$
on the diameter is used.  In this case, a warning message is printed
except when {\sf number} is exactly zero.
\end{itemize}

As an example, a circle can be meshed by using the following code.

\begin{code}
(make-mesh ((get-euclidean-space 2) :size-list '(inside .4 outside .2))
  (point origin 0 0)
  (region inside 0 0)
  (boundary outside (:closed? t :split 3) arc
    (pt 1 0) (:center origin) (pt -1 0) (:center origin)))
\end{code}


\begin{functiondef}{make-mesh-from-file}{stream
\key angle-bounds size-list}
Create a mesh using data from a Lisp input stream.  
The file's format is specified
in \sectref{MeshFF:Sec}.  The angle-bounds and size-list can be
specified in the file, but the file versions are overridden if the
keywords are used.  See the text following the description of {\sf
make-mesh} for how to use the {\em angle-bounds} and {\em size-list} keywords.
\end{functiondef}

\begin{functiondef}{write-mesh}{mesh stream}
Write the given mesh to a Lisp output stream.  
The format of
the resulting file is described in \sectref{MeshFF:Sec}.
\end{functiondef}

Assuming that the file meshRequest contains data in the proper format,
the following instructions would create a mesh and write it out to the
the file newMesh.

\begin{code}
(let ((mesh nil))
  (with-open-file (in-file "meshRequest" :direction :input)
    (setf mesh (make-mesh-from-file in-file)))
  (with-open-file (out-file "newMesh" :direction :output)
    (write-mesh mesh out-file)))
\end{code}

\section{Access to Portions of a Mesh}

The following functions are useful for examining an existing mesh.

\begin{functiondef}{boundary-complex-of}{mesh}
Return the simplicial-complex that holds the boundaries (internal and
external) of the mesh.
\end{functiondef}

\begin{functiondef}{home-of}{mesh}
Return the embedding {\em space} of the mesh (\ie, the first argument
to \keylisp{make-mesh}).
\end{functiondef}

\begin{functiondef}{all-names}{simplicial-complex}
Return a list of the different names that appear in the given
simplicial complex.
(Boundaries and regions of a mesh can have names; the names are used
in size-lists to allow control of mesh density.)
\end{functiondef}

In the example below, the first use of {\keylisp all-names} reports
all the names for boundaries of the mesh.  The second use reports all
the names for subregions of the mesh.

\begin{code}
(all-names (boundary-complex-of mesh))
(all-names mesh)
\end{code}

Another routine that is often useful when working with meshes is
\keylisp{map-over-cells}.  One common use of \keylisp{map-over-cells}
is to map over all triangles of a mesh or over all boundaries of a
mesh.
\begin{code}
(map-over-cells (triangle 2) mesh
   (print triangle))
\end{code}

\begin{code}
(map-over-cells (b-segment 1) (boundary-complex-of mesh)
   (print b-segment))
\end{code}

{\sf Map-over-cells} can also be used to map over all edges of a mesh
or to map over all vertices of a mesh.  Note that {\sf (first
(vertices-of vertex))} is used in the second example since {\sf
vertex} is a $0$-dimensional simplex.
\begin{code}
(map-over-cells (edge 1) mesh
   (print edge))

(map-over-cells (vertex 0) mesh
   (print (first (vertices-of vertex))))
\end{code}


\subsection{Individual Components of a Mesh}

The following functions are useful for examining a single component of
a mesh (\eg, a single triangle or a single boundary-edge).  Note that
several of these functions apply to simplices; thus they can be used
on both triangles (2-simplices) and boundary edges (1-simplices).

\begin{functiondef}{locate}{point mesh}
Return a triangle of the given mesh that has the given point in its
interior or on its boundary.  The point must be a point in the
embedding space of the mesh.
\end{functiondef}

As an example, the following code will find the triangle of mesh {\sf
my-mesh} that contains the origin.

\begin{code}
(locate (make-point (home-of my-mesh) 0 0) my-mesh)
\end{code}

\begin{functiondef}{name}{simplex simplicial-complex}
Return the name of the given simplex.  Used to determine names of
triangles and boundary-edges.
\end{functiondef}

\begin{methoddef}{vertices-of}{simplex}
Return a list of the vertices of the given simplex.  Each vertex is a
point in the embedding space of the mesh.  
Used to find the vertices of a triangle
or to find the endpoints of a boundary-edge.
\end{methoddef}

\begin{functiondef}{angles}{triangle}
Return a list of the given triangle's angles in degrees.
\end{functiondef}

\begin{functiondef}{simplex-size}{simplex}
Return the length of the longest side of the given simplex.  For
example, this gives the length of the longest side of a mesh-triangle
or the length of a boundary-edge.
\end{functiondef}


\section{Individual Meshing Functions}
\label{MeshingFunctions:Sec}

This section describes the direct programming interface to the mesher.
In particular, it describes the functions \keylisp{read-mesh} and
\keylisp{refine-mesh} that allow a user to refine an existing mesh.
It is 
unlikely that a casual user will need the remaining functions unless
the user plans to control the meshing process from within a program.
(One might wish to do this for use with adaptive remeshing, for
instance.)

\begin{functiondef}{read-mesh}{stream}
Read an existing mesh from a Lisp input stream.  The file format is
described in \sectref{MeshFF:Sec}.  This function can read a mesh that
was written to a file by \keylisp{write-mesh}.
\end{functiondef}

The following function can refine an existing mesh.  See the text
following the description of \keylisp{make-mesh} for how to use the
keywords angle-bounds and size-list.

\begin{functiondef}{refine-mesh}{mesh \key angle-bounds size-list}
Refine an existing mesh using the specified {\em angle-bounds} and
{\em size-list}.
\end{functiondef}

As an example, the following code will read an existing mesh from a
file, refine it (presumabley the size requested is smaller than when
the initial mesh was created), and write the mesh to another file.

\begin{code}
(let ((mesh nil))
  (with-open-file (in-file "oldMesh" :direction :input)
    (setf mesh (read-mesh in-file)))
  (refine-mesh mesh :size-list '(inside .1))
  (with-open-file (out-file "newMesh" :direction :output)
    (write-mesh mesh out-file)))
\end{code}

An empty mesh can be created using the following function.

\begin{functiondef}{create-mesh}{space}
Create an empty mesh in the given space.  Typically, the space is
Euclidean $2$-space.
\end{functiondef}

\begin{methoddef}{insert-boundary}{simplex mesh \key name}
Insert a 0-dimensional or a 1-dimensional simplex into a mesh as a
boundary.  A 0-dimensional simplex corresponds to a vertex that cannot
be removed from the mesh.  A 1-dimensional simplex (a boundary
segment) also cannot be removed, but can be split during the meshing
process.  The name, if specified, can be used within a size-table in a
keyword argument to {\sf refine-mesh}.
\end{methoddef}

\begin{functiondef}{name-region}{name point mesh}
Give a name to the enclosed (by boundaries) region of the mesh that
contains the given point.  Regions that are not named are eliminated
from the mesh (in \keylisp{refine-mesh}) before mesh refinement begins.
\end{functiondef}

Note that the initial triangulation of the mesh is accomplished during
the first call to \keylisp{name-region}.  Thus, this is when crossed
boundary-edges are detected.

The function \keylisp{make-curved-segment} can be used to create a
curved (parametric) boundary that can be inserted into a mesh using
\keylisp{insert-boundary}.  Note that straight-line boundary segments
are created using \keylisp{make-simplex} (see
\sectref{AlgebraicTopology:Sec}).

\begin{functiondef}{make-curved-segment} {space param-a end-a param-b end-b generator} 
Return 1-simplex that is a parametric curve.  The {\em generator} must
be a function that takes a number (a parameter value) and returns an
element of {\em space}.  {\em End-a} and {\em end-b} are the endpoints
of the segment; {\em param-a} and {\em param-b} represent the
corresponding parameter values.  Knowing the parameter values for the
endpoints and knowing the generator, we can, of course, generate
endpoints, but these would be new endpoints and would thus not link
properly with adjacent curves.  Requiring the specifiation of {\em
end-a} and {\em end-b} alleviates this problem.  The user is reponsibe
for ensuring that the parameter values and the endpoints agree.
\end{functiondef}

