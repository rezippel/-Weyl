% -*- Mode: LaTeX -*-
% $Id: FileFormats.tex,v 1.7 1995/04/06 20:48:14 chew Exp $

\chapter{File Formats}
\label{FileFormats:Chap}

It is often useful to store Weyl objects in disk files---as
intermediate back ups in long computations, to transport mathematical
objects between sites, or as a communication mechanisms with external
systems that want to use objects created by Weyl.  This chapter
discusses the file formats used by Weyl for a variety of different
mathematical objects.

\section{Topology and Geometry}
\label{TopologyFF:Sec}

Curently, there are two file formats related to topology and geometry,
one for a vertex set and one for a simplicial complex.

The \verb+<VertexSet>+ format is

\begin{code}
VertexSet <SpaceDimension> <NumVerts>
<vertex_0>
<vertex_1>
...
<vertex_{NumVerts - 1}>
\end{code}
\noindent
where \verb+<SpaceDimension>+ is the dimension of the embedding space
and \verb+<NumVerts>+ is the number of vertices.  Each \verb+<vertex>+
is a list of floating point numbers; the length of the list must match
the \verb+<SpaceDimension>+.  

In effect, the VertexSet creates a set of vertices numbered (in the
order given) from 0.  These vertex numbers are used in the file
formats for other structures such as {\em SimplicialComplex}.

The \verb+<SimplicialComplex>+ format is

\begin{code}
SimplicialComplex <Dim> <NumSimps>
[name]
<simplex_0>
[name]
<simplex_1> 
...
[name]
<simplex_{NumSimps - 1}> 
\end{code}
\noindent
where \verb+<Dim>+ is the maximum dimension of the simplicial complex
and \verb+<NumSimps>+ is the number of maximum dimensional simplices.
Note that this format is somewhat limited in that all simplices given
here must have the same dimension (e.g., they must all be triangles or
must all be segments).  A simplex is just a list of vertex numbers
(the length of each such list is one more than \verb+<Dim>+); the
vertex numbers are those defined by a preceding {\em VertexSet}.
Names are optional.  If a name is given then each of the simplices
between this name and the next one is assigned that name.  Currently,
the names are useful only when the SimplicialComplex is part of a
Mesh.  For a Mesh, the names are used in the specification of
MeshQuality.

\section{Meshes}
\label{MeshFF:Sec}

There are two file formats, one to request the meshing of a region and
the other to read/write meshes from/to files.

\subsection{MeshRequest File Format}

The {\em MeshRequest} file format is:

\begin{code}
MeshRequest
<VertexSet>
<SimplicialComplex>
<Regions>
<MeshQuality>
\end{code}
\noindent
\verb+<VertexSet>+ and \verb+<SimplicialComplex>+ 
describe the mesh boundaries and use file formats as described above.
The \verb+<SimplicialComplex>+ describes a simplicial-complex that
makes use of the vertices given in the preceding VertexSet.  The
\verb+<Regions>+ section determines (and names) the subregion(s) to be
meshed.  The \verb+<MeshQuality>+ section gives the angle bounds and
the element sizes.

To mesh a planar region, the VertexSet would have
\verb+<SpaceDimension>+ $= 2$ and the SimplicialComplex would have
\verb+<Dim>+ $= 1$ (since each boundary is a segment---a simplex of
dimension 1).

The boundary given by the SimplicialComplex divides the embedding
space into one or more subregions.  The \verb+<Regions>+ section is
used to determine which of those subregions are to be meshed.  A
subregion is meshed if and only if there is a point given in the
Regions section that lies witin that subregion.

The format for the \verb+<Regions>+ section is

\begin{code}
Regions <NumPoints>
<name_0> <region_pt_0>
<name_1> <region_pt_1>
...
<name_{NumPoints - 1}> <region_pt_{NumPoints - 1}>
\end{code}
\noindent
where \verb+<NumPoints>+ is the number of region points to be defined.
Each \verb+<region_pt>+ is a list of floating point numbers, where the
length of the list matches the dimension of the embedding space (the
\verb+<SpaceDimension>+) given in the preceding \verb+<VertexSet>+.
The names are used to specify the desired element sizes for the mesh
in the \verb+<MeshQuality>+ section.  During meshing, each simplex of
a region has a name based on the name assigned here to the region.

The format for the \verb+<MeshQuality>+ section is

\begin{code}
[AngleBounds <real> <real>] 
[SizeTable <NumEntries>
<name_0> <real>
<name_1> <real>
...
<name_{NumEntries - 1}> <real>]
\end{code}
\noindent
If AngleBounds are given then the first number is a lower bound on
element angles and the second number is an upper bound.  If no angle
bounds are given then the resulting elements angles are between 30 and
120 degrees.  The SizeTable controls the density of the mesh by
specifying the maximum size (the length of the longest edge) allowed
for each named region or boundary.  Note that sizes in this file
format are simply numbers.  This is more restrictive than for meshes
created from within Weyl where it is possible to specify sizes that
are functions.

As an example, a request to mesh a square could look like:

\begin{code}
MeshRequest
VertexSet 2 4
0.0 0.0
1.0 0.0
1.0 1.0
0.0 1.0
SimplicialComplex 1 4
name
0 1 
1 2 
otherName
2 3  
3 0
Regions 1
interior
0.5 0.5
AngleBounds 10 150
SizeTable 3
name 0.2
otherName 0.1
interior 0.4
\end{code}

The function \keylisp{make-mesh-from-file} is used to read a
MeshRequest and to create the corresponding mesh.  See
\sectref{CreateMesh:Sec}.

\subsection{Mesh File Format}

The Mesh file format is

\begin{code}
Mesh
<VertexSet>
<SimplicialComplex> 
<SimplicialComplex>
\end{code}
\noindent
where the first SimplicialComplex represents the boundaries of the
mesh, and the second represents the elements of the mesh.  Names can
appear within each SimplicialComplex.

For example, a very simple mesh might be written to a file as

\begin{code}
Mesh
VertexSet 2 4
0.0 0.0
1.0 0.0
1.0 1.0
0.0 1.0
SimplicialComplex 1 4
boundary
0 1
1 2
2 3
3 0
SimplicialComplex 2 2
interior
0 1 2
0 2 3
\end{code}

The functions \keylisp{write-mesh} and \keylisp{read-mesh} are used
for writing and reading meshes.  Function \keylisp{write-mesh} is
described in \sectref{CreateMesh:Sec}.  Function \keylisp{read-mesh}
is described in \sectref{MeshingFunctions:Sec}.

\subsection{Regions with Curved Boundaries}

The current release contains only limited facilities for representing
curved boundaries in files.  We plan to extend this facility for the
next release.  Note that there are more powerful techniques available
for representing curved boundaries when working from within Weyl (see
\keylisp{make-curved-segment} in \sectref{MeshingFunctions:Sec}).
Within a file, curved boundaries are currently limited to circular
arcs.  File representations of arcs correspond roughly to the
available arc representations in \keylisp{make-mesh} (see
\sectref{CreateMesh:Sec}).  

At present, arcs can only be represented in MeshRequest files.  When
the resulting mesh is written out to a Mesh file, the curve
information is lost.  (Note though that boundaries retain their names,
so it is often possible to reconstruct curve information without
severe difficulty.)

Within a MeshRequest file, boundary arcs are indicated via the
inclusion of some additional information in the SimplicialComplex.
Recall that a line-segment boundary is represented in a file by a pair
of vertex numbers.  In contrast, an arc is represented by a vertex
number, a parenthesized list of arc arguments, and another vertex
number.  The two vertex numbers determine the endpoints of the arc,
$e_1$ and $e_2$.  The possible arc arguments are

\begin{itemize}

\item {\sf (c \verb+<number>+)}
The counterclockwise arc from $e_1$ to $e_2$ that is part of the
circle through $e_1$ and $e_2$ and centered at the vertex with vertex
number \verb+<number>+.  It is possible (but not a good a idea) to
specify a center that is not equidistant from $e_1$ and $e_2$.

\item {\sf (t \verb+<number>+)}
The counterclockwise arc from $e_1$ to $e_2$ that is part of the
circle through $e_1$, $e_2$, and the vertex with vertex number
\verb+<number>+.

\item {\sf (r \verb+<number>+)}
The counterclockwise arc from $e_1$ to $e_2$ that is part of the
circle with radius \verb+<number>+ through $e_1$ and $e_2$ and centered
to the left of the ray from $e_1$ to $e_2$.  If the radius is negative
then the circle is centered to the right of the ray from $e_1$ to
$e_2$.  A warning is issued if the absolute value of the radius is
less than half the distance from $e_1$ to $e_2$; the arc that results
in this case is a half-circle.

\end{itemize}

It is also possible to specify clockwise arcs by using {\sf c-}, {\sf
t-}, or {\sf r-} in place of {\sf c}, {\sf t}, or {\sf r},
respectively (although equivalent effects can be achieved by reversing
the order of the two endpoints).

As an example, the following MeshRequest file describes a circle of
radius one centered at the origin.  Each of the three possible arc
arguments have been used for part of the circle.

\begin{code}
MeshRequest
VertexSet 2 5
0 0
1 0
0 1
-1 0
0 -1
SimplicialComplex 1 4
boundary
1 (c 0) 2
2 (r 1) 3
3 (t 2) 4
1 (c- 0) 4
Regions 1
interior 0 0
AngleBounds 10 120
SizeTable 2
interior 0.4
boundary 0.2
\end{code}

