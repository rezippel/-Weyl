% -*- Mode: LaTeX -*-
% $Id: Topology.tex,v 1.9 1995/05/12 20:41:36 chew Exp $
\chapter{Spaces and Topology}
\label{Topology:Chap}

Spaces are sets of mathematical objects ({\em domains} in Weyl's
terminology) that we endow with some topological structure. Examples
of spaces include $n$-dimensional Euclidean space, the space of square
integrable functions and the spectrum of a ring. In addition, we often
provide a metric to provide the geometric structure of a space.  This
chapter describes the tools Weyl provides for creating and
manipulating these spaces.
  
\begin{figure}
\begin{center}
\BoxedEPSF{SpaceDomains.eps scaled 500}
\end{center}
\caption{Basic Domains for Spaces\label{SpaceDomains:Fig}}
\end{figure}

The basic domains Weyl provides for spaces are shown in
\figref{SpaceDomains:Fig}.  All spaces are subclasses of
\keylisp{abstract-space}. It would be very useful to be able to
indicate that a space is Hausdorff, connected, compact or has some
similar topological property. This is not generally possible at the
moment, but will be added at some future date.
  
One property that can be attached to a space is its {\em
dimension}. Intuitively, the dimension of a space can be thought of as
the number of orthogonal directions that can be taken in the
space. However, this does not apply to spaces that are not embedded in
some Euclidean space. Mathematically, the dimension of a topological
space is either a non-negative integer or infinity.\footnote{By using
the Hausdorff dimension for the dimension of a space, we can construct
spaces that have non-integral dimension.} For Weyl's purposes, it is
useful to, in addition, have spaces without a specific dimension. A
\keylisp{dimensional-space} is a space that has a specific, possibly
infinite, dimension.
  
A very useful, specific type of dimensional space is a \keyi{Euclidean
space}. A Euclidean space is $\R^n$ endowed with the Euclidean metric
for measuring distances between points. The open sets of a Euclidean
space can be constructed via set union from the open balls $B_d(p)$,
the set of points of distance less than $d$ from the point $p$. The
open sets are closed under union and finite intersection.  The closed
sets are the complements of the open sets.

  
Closely related to Euclidean spaces are general vector spaces and
modules. Recall from \sectref{ModuleDomain:Sec}, that $M$ is an
$R$-module if $M$ is an abelian group and if multiplication by
elements of $R$ is closed in $M$. A \keyi{free module} is a module
that is freely generated.  As can be seen from
\figref{SpaceDomains:Fig}, Weyl only permits free modules that have
specified dimensions.  A \keyi{vector space} is a free $R$-module
where $R$ is a field. From \figref{SpaceDomains:Fig} we see that Weyl
treats a Euclidean space as a vector space, where the coefficient
domain is the real numbers, $R$.
%For consistency with other areas of
%mathematics, we also call a vector space an \keyi{affine space}. 
The routines for dealing with free modules are described in
\sectref{FreeModules:Sec}.
  
Finally, Weyl provides a mechanism for dealing with projective spaces.
A projective space is an $n+1$ dimensional space together with an
equivalence relation that equates points that are nonzero scalar
multiples of each other, i.e., $u \cong v$ if $u = \alpha v$, for
$\alpha \not= 0$. Due to the equivalence relation, the projective
space has dimension $n$.
  
The actual dimensions of a space can be determined by the function
{\sf dimensions}.
  
\begin{functiondef}{dimensions}{space} 
Returns a list of the dimensions of the space. For
free-modules, vector spaces, projective spaces, and
so on there is only one element in this list. However
for spaces like $\R^3 \times \Z^5$ there may be more elements.
\end{functiondef}
  
\section{Point Set Topology}
\label{PSTopology:Sec}

Objects in topologies are subsets of {\em spaces}.  The root class of
all spaces is abstract-space. Abstract spaces do not have a dimension
and consist only of points.  To create an abstract space one must
explicitly call {\sf make-instance} on the class
\keylisp{abstract-space}. The most commonly used spaces, the Euclidean
spaces, can be created using the function
\keylisp{get-euclidean-space}:
  
\begin{functiondef}{get-euclidean-space}{dimension \optional (domain
{\sf *general*})} Creates, if needed, an instance of the Euclidean
space with the indicated dimension. Elements of the Euclidean space
will be represented as $n$-tuples. The components of these $n$-tuples
will be elements of {\em domain}.
\end{functiondef}  

\begin{figure}
\begin{center}
\BoxedEPSF{PointClasses.eps scaled 500}
\end{center}
\caption{Basic Classes for Points\label{PointClasses:Fig}}
\end{figure}

To create points in a space we use the function \keylisp{make-point}:
  
\begin{functiondef}{make-point}{space coord1 \rest coords}
Creates a point in {\em space}. If {\em space} is an abstract space,
then only one argument is expected and will be treated as a unique
identifier for the point in the space. If the first coordinate is nil
then a new {\em anonymous} point will be created. For vector spaces,
the values are expected to be the coordinates of the point.
\end{functiondef}
   
All points in a space are identified by a unique integer
identifier. This is currently implemented (using the class {\sf
weyli::has-id-number}) by including a slot in each point containing an
integer unique to that point.  When anonymous points are generated in
an abstract space, the printed representation is distinguished by this
number. For some spaces, \eg, $\R$, it may be more appropriate to use
a different ordering.

\begin{code}  
> (setq abs-space (make-instance 'abstract-space))
#<Domain: ABSTRACT-SPACE>

> (progn
     (setq a (make-point abs-space nil)
           b (make-point abs-space nil)
           c (make-point abs-space 'c))
     (list a b c))
(<1> <2> <C>)
\end{code}
  
\noindent
Notice that the third point was created with a name and, unlike the
other two points, its printed form includes that name.
  
Similarly, we can create points in Euclidean domains.  In this case,
the printed representation of a point includes the point's
coordinates.
  
\begin{code}
> (setq ee (get-euclidean-space 3))
E^3
  
> (make-point ee 1 2 3)
<1, 2, 3>
\end{code}  
  
\section{Affine Spaces}
\label{AffineSpace:Sec}
  
Affine spaces are created using the function \keylisp{make-affine-space}:
  
\begin{functiondef}{make-affine-space}{field dimension}  
Create an affine space of dimensions {\em dimension} where the
components are elements of the field {\em field}. (Ed: I think this
should be weakened to work for arbitrary rings.)
\end{functiondef}

\begin{code}
>(setq R (get-real-numbers))
R
>(setq space (make-affine-space R 2)
\end{code}

Elements of spaces, both affine and projective, are created using the
generic function {\sf make-point}:
  
\begin{functiondef}{make-element}{space \rest elements}  
Creates an element of {\em space}. {\em Elements} is a list of $n$
elements which can be coerced into the coefficient domain of space.
\end{functiondef}
  
There is also an internal function \keylisp{weyli::make-element} that
does not do any checking of its arguments and can lead to rather
subtle problems if used incorrectly.  On the other hand it is
noticeably faster.
  
\begin{functiondef}{cross-product}{u v}
This function is only defined for elements of three dimensional vector
spaces. If $u = \langle u_1, u_2, u_3\rangle$ and $v = \langle v_1,
v_2, v_3\rangle$ then
\[
(\mbox{\sf cross-product} u v) =
(v_2 u_3 - u_3 v_2, u_3 u_1 - u_1 v_3, u_1 v_2 - u_2 v_1).
\]
\end{functiondef}

\section{Projective Spaces}
\label{ProjectiveSpaces:Sec}
  
\begin{functiondef}{make-projective-space}{field dimension}
Create a projective space of dimensions {\em dimension} where the
components are elements of the field {\em field}.
\end{functiondef}

  
As in the case of regular affine spaces, elements of a projective
space are created using the generic function \keylisp{make-point}.

\begin{functiondef}{make-point}{space \rest elements}
Creates a point which an element of {\em space}.  For projective
spaces, {\em elements} can be a list of either $n$ or $n+1$ elements
of the coefficient domain of space.  If $n+1$ elements are provided
then these are the full set of elements of the point.  If only $n$
elements are provided, the final missing element is filled out by a 1
from the coefficient domain of space.
\end{functiondef}
  
\begin{code}
(setq p (make-projective-space r 2))
(make-point p 1 1 1)
\end{code}
  
Affine spaces can be embedded in projective spaces.  For projective
spaces of dimension $n$, there are $n+1$ canonical embeddings. The
function \keylisp{make-affine-projection} is passed a projective space
and creates an affine space with an attached homomorphism into the
projective space.
  
\begin{functiondef}{make-affine-projection}{space \optional dimension}  
This function returns an affine space that is the projection of {\em
space} where we hold the component {\em dimension} fixed.  For
instance, let $p = (u, v, w)$ is an element of a two dimension
projective space $P^2$. The image of $p$ in the affine projection of
$P^2$ with dimension $0$ held fixed is $(v/u, w/u)$.  With dimension
$1$ is held fixed then $p$ maps to $(u/v, w/v)$. Finally, if dimension
2 is held fixed, $p$ maps to $(u/w, v/w)$. If {\em dimension} is not
provided then we produce an affine space where the last dimension is
held fixed.
\end{functiondef}
  
\section{Algebraic Topology}
\label{AlgebraicTopology:Sec}

\subsection{Cells and Simplices}

A \keyi{$k$-cell} is a region (subset) of a space that is homeomorphic
to $B^k$, the unit ball in $R^k$: $B^k = \{p \in R^k : ||p|| <= 1\}$.
That is, a $k$-cell is a set that is topologically equivalent to an
$k$-dimensional ball---it is connected, without any holes, etc.
  
A \keyi{$k$-simplex} is a $k$-cell defined by the convex hull of $k+1$
points, called vertices.  For example, a $1$-simplex is a line segment
defined by two vertices, while a $2$-simplex is a triangle, defined by
three. 
%An {\em oriented} $k$-simplex is defined by an ordered list of
%vertices, and the lists are partitioned into two {\em orientations} --
%those that are even permutations of some reference ordering, and those
%that are odd permutations.

Unlike most objects in Weyl, (oriented) simplices are not domain
elements. They should be viewed as sorted lists of sets of points.
  
\begin{functiondef}{make-simplex}{vertex$_0$ $\ldots$ vertex$_k$}
Creates a $k$-simplex whose components are $\mbox{\sf vertex}_0,
\ldots, \mbox{\sf vertex}_k$. This routine ensures that all of the
points are from the same space.  Points are always reordered to match
the order of their id-numbers.  Simplices are immutable, \ie, once
created they can not be modified.
\end{functiondef}
  
Assuming that the variables {\sf a}, {\sf b} and {\sf c} are points in
some abstract space, a triangle could be created as follows:
 
\begin{code}
> (setq triangle (make-simplex a b c))
[<A>, <B>, <C>]
\end{code}

\noindent
Two simplices are equal if they have the same vertex set.
  
\begin{code}
> (= triangle (make-simplex a b c))
t
\end{code}
  
\begin{functiondef}{vertices-of}{simplex}
Returns a list of the vertices of {\em simplex}.  For instance,
\begin{code} 
> (vertices-of triangle)
(<A> <B> <C>)
\end{code}
\end{functiondef}
  
Notice that there aren't any commas between the vertices. This is a
LISP list as opposed to a simplex.
  
\begin{methoddef}{face?}{cell1 cell2}
A predicate that returns {\sf T} if {\em cell1} is a face of {\em
cell2}.  Cells are defined to be faces of themselves.
\end{methoddef}

\begin{methoddef}{face?}{vertex-list cell}
A predicate that returns {\sf T} if the simplex represented by the
{\em vertex-list} is a face of {\em cell}.
\end{methoddef}

%This function does not appear within the code. -lpc
%\begin{functiondef}{same-cell?}{cell1 cell2}
%A predicate that returns {\sf T} if {\em cell1} and {\em cell2} are
%the same cell, independent of orientation.
%\end{functiondef}
  
\begin{methoddef}{dimension-of}{cell}
Returns a Lisp integer that is the dimension of {\em cell}.
Currently, Weyl provides a dimension function only for simplices.
\end{methoddef}

\begin{methoddef}{opposite}{face simplex}
Returns a list of vertices that are opposite the given {\em face} of
{\em simplex}.  The {\em face} can be a simplex, a list of vertices,
or a single point.  Useful for navigation within a \key{cell complex}.
\end{methoddef}

\subsection{Complexes}
  
A \keyi{cell complex} $K$ is a set of cells of the same
space with the property that if $c \in K$ then $\mbox{faces}(c)
\subset K$, that is if a cell $c$ is in the complex then all of the
faces of $c$ must also be in $K$.  It follows that there is a set of
maximal cells (those that are not the face of any other cell in
$K$) that provide a unique minimal representation for $K$. Although
the maximal cells in a complex often have the same dimension, this is
not required.

A \keyi{simplicial complex} is a cell complex that can contain only
simplices.
  
\begin{functiondef}{make-simplicial-complex}{simplices}
Creates a simplicial complex containing each of the simplices in the
list {\em simplices}.
\begin{code}
> (setq complex (make-simplicial-complex (list triangle)))
#<Simplicial-Complex>
\end{code}
\end{functiondef}

\begin{macrodef}{map-over-cells}{(var \optional $n$) cell-complex
\body body}
The forms in {\em body} are evaluated with {\em var} bound to each
$n$-dimensional cell of {\em cell-complex}.  If $n$ is {\sf nil} or
unspecified then the body is evaluated for all cells of {\em complex},
regardless of dimension.  {\em Cell-complex} is a \key{cell complex}.
\end{macrodef}

\begin{macrodef}{map-over-maximal-cells}{(var) cell-complex \body body}
The forms in {\em body} are evaluated with {\em var} bound to each
maximal cell of {\em cell-complex}.
\end{macrodef}

Cells constructed at various times may be equivalent, but not identical
as Lisp objects.  The function \keylisp{get-cell} can be used to
ensure that cells are identical (\ie, Lisp {\sf EQ}).  Note that most
functions that act on some cell within a cell complex (such as
\keylisp{member}, \keylisp{facets}, \keylisp{cofacets},
\keylisp{maximal-cell?}, and \keylisp{delete-maximal-cell}) expect the
input cell to be identical (as a Lisp object) to the cell within the
complex.  In other words, unless the input cell came from the specific
cell complex due to some other operation (\eg, {\sf map-over-cells}),
you should probably use {\sf get-cell}.

\begin{methoddef}{get-cell}{cell cell-complex}
Returns a cell of {\em cell-complex} that is equivalent to {\em cell}.
{\sf Nil} is returned if there is no such cell.
\end{methoddef}

\begin{methoddef}{get-cell}{vertex-list cell-complex}
Returns a simplex of {\em cell-complex} that has the vertices
specified in {\em vertex-list}.  {\sf Nil} is returned if there is no
such simplex.
\end{methoddef}

\begin{methoddef}{member}{cell cell-complex}
Returns {\sf T} if {\em cell} is a member of {\em cell-complex}.
\end{methoddef}

The {\em facets} of a cell are the cell's faces that are one dimension
lower than the cell itself.  For instance, for a tetrahedron, the
facets are the triangles that make up the sides of the tetrahedron.

\begin{methoddef}{facets}{cell cell-complex}
Returns a list of the facets of {\em cell}.  If {\em cell-complex} is a
cell complex then the reported facets are all (pre-existing) members
of the cell complex.  If {\em cell-complex} is {\sf nil} then the
facets are newly constructed.
\end{methoddef}

\begin{methoddef}{facets}{cell-list cell-complex}
Returns a list containing each cell that is a facet of some cell in 
{\em cell-list}.  No duplicates appear in the resulting list.  This is
useful for finding facets of facets; for instance, one can find the
edges of a tetrahedron.
\end{methoddef}

Within a \key{cell complex}, the {\em cofacets} of a cell $c$ are all
the cells of the complex that have $c$ as a facet.  {\sf Facets} and
{\sf cofacets} together provide a means for ``navigating'' within a
cell complex.

\begin{methoddef}{cofacets}{cell cell-complex}
Returns a list containing the cells of {\em cell-complex} that are
cofacets of {\em cell}.
\end{methoddef}

\begin{methoddef}{cofacets}{cell-list cell-complex}
Returns a list containing each cell of {\em cell-complex} that is a
cofacet of some cell in {\em cell-list}.  No duplicates appear in the
resulting list.  This is useful for finding cofacets of cofacets; for
instance, one can find all triangles adjacent to a given vertex.
\end{methoddef}

\begin{methoddef}{maximal-cell?}{cell cell-complex}
Returns {\sf T} iff {\em cell} is a \key{maximal cell} of {\em
cell-complex}.
\end{methoddef}

\begin{methoddef}{vertex-set}{cell-complex}
Returns a list of all the vertices that appear in the {\em
cell-complex}.
\end{methoddef}

Cell complexes are, in general, immutable. New cell complexes can be
created from old cell complexes using boolean operations like union
and intersection.
  
\begin{functiondef}{union}{complex1 complex2}
Returns a new \key{cell complex}, whose cells are those that appeared
in either {\em complex1} or {\em complex2}.
\end{functiondef}

\begin{functiondef}{intersection}{complex1 complex2}
Returns a new cell complex, whose cells are those that appear in both
{\em complex1} and {\em complex2}.
\end{functiondef}
  
The following operations violate the immutability of cell complexes.
Thus they are only intended for use in situations that require
additional performance or to implement higher level operations which
do preserve the immutability of their arguments.
  
New cells can be added and deleted from cell complexes using {\sf
insert} and {\sf delete-maximal-cell}.
  
\begin{methoddef}{insert}{cell cell-complex}
Destructively modifies {\em cell-complex} by adding {\em cell} and
each of its subcells to {\em cell-complex}.  (Each cell and subcell is
checked to see if it is already a member, thus avoiding insertion of
duplicate cells.)  This operation is provided for use by internal
routines.
\end{methoddef} 
 
\begin{methoddef}{delete-maximal-cell}{cell cell-complex}
Destructively modifies {\em cell-complex} by deleting {\em cell}
(which must be a \key{maximal cell}) from {\em cell-complex}.  Any
subcell of {\em cell} that is not contained within some other cell of
{\em cell-complex} is also deleted.  This operation is provided for
use by internal routines.
\end{methoddef}


\subsection{Chains}
\label{Chains:Sec}
  
If $K$ is an oriented simplicial complex, and $G$ an abelian group, a
$p$-chain $c_p \in C_p(K,G)$ is a map $c_p:p-\mbox{simplices}(K)
\rightarrow G$ that assigns an element of $G$ to each $p$-simplex in
$K$.  Equivalently, we may say that a $p$-chain is a {\it formal sum}
of the $p$-simplices of $K$ with coefficients in $G$.  The group
operation in $G$ is then used to define the $+$ operator for
$C_p(K,G)$, yielding the abelian group of $p$-chains over $K$ and $G$.



\begin{functiondef}{get-chain-module}{complex n \optional group}
Creates the group of $n$-chains of {\em complex} with coefficients in
{\em group\/}.  If {\em group} is not provided then $\Z$ is used
instead.  If provided, {\em group} must be an abelian group.
\end{functiondef}

Individual chains may be created by coercing a simplex into the chain
module.
\begin{code}
> (setq 1-chains (get-chain-module complex 1))
C[1](#<COMPLEX>)

> (coerce (make-simplex a b) 1-chains)
[<A>, <B>]

> (+ (coerce (make-simplex a b) 1-chains) 
     (* 2 (coerce (make-simplex b c) 1-chains)))
[<A>, <B>] + 2[<B>, <C>]
\end{code}

If $\sigma = [v_0, \ldots, v_p]$ is a $p$-simplex, then its boundary
is the following $p-1$ chain:
\[
\partial \sigma = \partial [v_0, \ldots, v_p]
 = \sum_{i=0}^p (-1)^i [v_0, \ldots, \hat{v_i}, \ldots, v_p],
\]
where $\hat{v_i}$ indicates that the $v_i$ vertex is missing.
The boundary operation can be extended to chains by linearity.

\begin{functiondef}{boundary}{chain}
{\em chain} can be either chain or a simplex.  It returns the chain
representing the boundary of {\em chain\/}.
\end{functiondef}

This is illustrated by computing the boundary of {\sf triangle}.
\begin{code}
> (setq tri-bound (boundary triangle))
[<A>, <B>] + [<B>, <C>] - [<A>, <C>]

> (boundary tri-boundary)
0
\end{code}
It is generally true that $\partial^2 \sigma = 0$ for all simplices
$\sigma$. 

\begin{functiondef}{boundary-domain}{chain}
Returns the chain-module that is the domain for the boundary of {\em
chain}. 
\end{functiondef}
  
\begin{functiondef}{boundary-domain}{simplex}
Returns a chain whose coefficients are the derivatives of the given
chain coefficients.  It is assumed that the new chain is in the same
chain module as the old chain.
\end{functiondef}

 
\begin{functiondef}{deriv}{chain \rest params}
Returns a chain whose coefficients are the derivatives of the given
chain coefficients.  It is assumed that the new chain is in the same
chain module as the old chain.
\end{functiondef}
  
Chains may be added and multiplied by elements of their coefficient domain.  
  
\begin{functiondef}{make-chain}{chain-module simplex-coefficient-pairs}
Creates a chain
\end{functiondef}

%\section{Charts}  

