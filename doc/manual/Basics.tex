% -*- Mode: LaTeX -*-
% $Id: Basics.tex,v 1.5 1995/03/31 15:31:14 rz Exp $
\chapter{Basics}
\label{Basics:Chap}

This chapter describes the basic tools used by Weyl to model the
algebraic structures of mathematics.  The fundamental difference
between Weyl and most other systems is that in addition to providing a
representation for domain elements like polynomials, Weyl also
provides representations for the algebraic domains of which the
polynomials are elements.  In \sectref{Domains:Sec} we discuss {\em
domains} and {\em domain elements}. \sectref{Morphisms:Sec} discusses
morphisms, which are used to define the relationships between the
elements of two domains.  \sectref{Coercions:Sec} illustrates how to
make use of these relationships.  \sectref{Domain:Hierarchy:Sec}
presents the hierarchy of domains being used in Weyl. Weyl uses the
Common Lisp Object System quite heavily. We begin this chapter with a
brief introduction to CLOS, focusing on those points that are
relevant to Weyl.

\section{Common Lisp Object System}
\label{CLOS:Sec}

Weyl is based on the Common Lisp Object System (CLOS) \cite{CLOS:Manual}
programming model.  Since this model of object-oriented programming
differs significantly from the more common, message-based paradigm
used in Smalltalk and \Cp+, we give a short introduction here.

CLOS is an object-oriented extension to Common Lisp that has been
adopted as part of ANSI standard Common Lisp during the
standardization process.  A portable implementation of CLOS is
available\footnote{It can be transferred from {\sf parcftp.xerox.com}
in the {\sf pub/pcl} directory.} and most commercial Common Lisp
vendors provide efficient implementations.  This introduction hits
only the highlights needed for Weyl.  Many other features are present
and are described in the standard.

CLOS extends Lisp with two pairs of new concepts: {\em instances} and
{\em classes\/}, and {\em methods} and {\em generic functions\/}.
Instances and classes are an extension of the ``structure'' concept
that is already present in Common Lisp (and most other modern
languages).  Methods and generic functions are completely new to
Common Lisp and provide most of the new functionality.

Instances are structures that contain zero or more {\em slots} that
can each hold arbitrary Lisp values. Each instance is derived from a
{\em class} of similar objects. For example, we might want to
construct the class to represent complex numbers of the form $a + bi$,
where $i$ is $\sqrt{-1}$. Each instance of this class contains two
slots, which we indicate by the names {\sf a} and {\sf b}, in which
the real and imaginary parts of the complex number are stored.  The
class of such instances might be called \keylisp{complex-number} and
defined as:
\begin{code}
(defclass complex-number ()
  ((a :initarg :real :accessor real-part)
   (b :initarg :imag :accessor imag-part)))
\end{code}

A particular instance of the class \keylisp{complex-number} can be created by
calling the function \keylisp{make-instance}.  The keywords provided with
the {\sf :initarg} options in the class definition are used to
indicate the initializing values for each slot.  A particular {\sf
point} can be created as:
\begin{code}
(setq z (make-instance 'complex-number :real 1.0 :imag 2.0))
\end{code}

\noindent
The keywords provided with the {\sf :initarg} options in the class
definition are used to indicate the initializing values for each
slot. Their relationship to the slots is given in the {\sf defclass} form
that defines \keylisp{complex-number}.


The {\sf :accessor} options used in the class definition define forms
that can be used to read the values in each slot.  The forms can also
be used as arguments to {\sf setf}.\footnote{Recall that {\sf setf} is
the extensible assignment operator used by Common Lisp.  Its first
argument is an accessor and its second argument is the new value that
is stored in the location referred to by the accessor.  It can be used
to update the value of variables and modify lists, arrays and
structures.  Since arbitrary functions can be used to update these
structures, nontrivial computations can be caused by the use of {\sf
setf}.}  For example, we can modify {\sf z} to be its complex
conjugate by:
\begin{code}
(setf (imag-part z) (- (imag-part z)))
\end{code}

One can define a class so that each of its instances contain the slots
associated with another class.  For example, if we wished to associate
another property with a complex number, \eg, the units, we would
define a class like
\begin{code}
(defclass dimensioned-complex-number (complex-number)
  ((units :initarg :units :accessor units)))
\end{code}

\noindent
An instance of a {\sf dimensioned-complex-number} would contain three
slots, {\sf a} and {\sf b} slots that are {\em inherited} from
\keylisp{complex-number} and a {\sf units} slot that is unique to
instances of {\sf dimensioned-complex-number}.  We say that the class
{\sf dimensioned-complex-number} {\em inherits} from
\keylisp{complex-number}, and that {\sf complex-number} is a {\em
superclass} of {\sf dimensioned-complex-number}.

Classes can inherit from more than one superclass, in which case their
instances' set of slots contains the union of the slots of the
superclasses.  This introduces an potential problem when the same slot
name is used in more than one superclass.  How this dealt is with is
discussed in detail in the CLOS specification \cite{CLOS:Manual}.  For
our purposes, we will assume that this situation never occurs.  If it
occurs, it is to be treated as an error.

The dimensioning example given above is a good example of where
multiple inheritance would be naturally used.  There are many
different types of numerical objects for which we would like to
associate units.  This could be done using the
following definition: 
  
\begin{code}
(defclass dimensioned-floating-number ()
  ((float :initarg :value :accessor float-value)
   (units :initarg :units :accessor units)))
\end{code}

\noindent
However, the definition of the {\sf units} slots is now repeated in
two different class definitions. Thus modifications to one slot may
require modifications at several places in the source code.

To simplify our definitions and to ensure that consistent names are
used for the initialization keyword and for the accessing routines, we
create a second superclass for units, \viz,
\begin{code}
(defclass dimensioned-number ()
  ((units :initarg :units :accessor units)))
\end{code}

\noindent
Now the definition of {\sf dimensioned-complex-number} becomes just
\begin{code}
(defclass dimensioned-complex-number (complex-number dimensioned-number)
  ())
\end{code}

Similarly, the definition of {\sf dimensioned-float} should be
\begin{code}
(defclass floating-number (dimensioned-number)
  ((float :initarg :value :accessor float-value)))

(defclass dimensioned-floating-number 
    (floating-number dimensioned-number)
  ())
\end{code}
  
Notice that we split off a {\sf floating-number} class so that {\sf
dimensioned-floating-number} is the combination of two classes that
provide orthogonal functionality, in accordance with good
software-engineering practice. 
  
This is all that needs to be known about classes and instances in
CLOS.

\medskip
The basic building block of Lisp is function invocation.  In CLOS,
this mechanism is extended via {\em polymorphism}.  This is allows the
same function name to be used for arguments of different types and
with different pieces of code invoked.  For example, assume we want to
use the function \keylisp{plus} to add two numbers.  In particular, we
want {\sf plus} to work for two complex numbers.  This can be
accomplished by defining the following {\em method\/}:
\begin{code}
(defmethod plus ((z1 complex-number) (z2 complex-number))
  (make-instance 'complex-number 
                 :real (+ (real-part z1) (real-part z2))
                 :imag (+ (imag-part z1) (imag-part z2))))
\end{code}

\noindent
The {\sf +} function used above is the addition function of Common
Lisp.   Similarly, we should create a {\sf plus} method for combining
two {\sf floating-number}s.
\begin{code}
(defmethod plus ((z1 floating-number) (z2 floating-number))
  (make-instance 'floating-number  
                 :value (+ (float-value z1) (float-value z2))))
\end{code}
  
Behind the scenes, CLOS creates a {\em generic function}, which it
places in the function cell of the symbol {\sf plus}.  When invoked,
the generic function chooses the most specific method for the classes
of its arguments and invokes it. The code for choosing the most
appropriate {\sf plus} method is generated at run time, after all the
{\sf plus} methods are known. In addition, the process of invoking the
selected method is usually done using a mechanism that is more
efficient than a standard function call. This allows the code to be
quite efficient.
  
As with multiple inheritance of classes, it is possible for
ambiguities to arise when deciding which method is most appropriate
for a given set of arguments. The CLOS specification provides a
precise mechanism for eliminating this ambiguity.
  
Notice that methods are not associated with particular classes like
complex-number or floating-number as is done in languages like
Smalltalk, but with generic functions.  An example of where this is of
value is defining a method for adding a \keylisp{complex-number} and
\keylisp{floating-number}:
\begin{code}
(defmethod plus ((z1 complex-number) (z2 floating-number))
  (make-instance 'complex-number 
                 :real (+ (real-part z1) (float-value z2)) 
                 :imag (imag-part z1)))
\end{code}
  
This ``generic function oriented'' object-oriented programming style
is the major difference between CLOS and most other objected-oriented
programming environments.  It allows CLOS to treat the arguments to
generic functions symmetrically, without bias towards one or the
other.

If {\sf plus} is applied to two {\sf dimensioned-complex-numbers}, the
value returned will be a {\sf complex-number}, not a {\sf
dimensioned-complex-number}.  Furthermore, nothing will check to see
if the units of the two complex numbers match. Thus we really could
add apples and oranges. This problem can be addressed by adding the
following method:
\begin{code}  
(defmethod plus ((z1 dimensioned-complex-number) (z2 dimensioned-complex-number))
  (if (eql (units z1) (units z2)) 
      (make-instance 'dimensioned-complex-number 
                     :real (+ (real-part z1) (real-part z2)) 
                     :imag (+ (imag-part z1) (imag-part z2)) 
                     :units (units z1))
      (error "Combining dimensioned numbers with different units.")))
\end{code}
  
\noindent
Notice that when the dimensions of two {\sf complex-numbers} differ
this method will signal an error. A more sophisticated method might
perform a unit conversion.
  
In order to distinguish the different {\sf plus} methods
(there are now three), we will qualify the names of the
methods with the classes of their arguments. So this
last {\sf plus} method is the {\sf (plus dimensioned-complex-number
dimensioned-complex-number)} method.
  
The way in which we coded this last {\sf plus} method has a problem.
Each time a new subclass of {\sf dimensioned-objects} is defined, we
will have to define a new, specialized version of the {\sf plus}
method. Furthermore, the central code of the complex-number {\sf plus}
method is repeated.  This problem can be avoided by using a special
type of method, called an {\sf :around} method, which ``wraps itself
around all other methods.''  When {\sf :around} methods are provided,
each of the applicable ``around'' methods are collected into a list
with the regular or primary methods placed at the end of the list. The
generic function then invokes the first method in the list. The value
returned by this method is the value of the generic function. However,
each {\sf :around} method can pass control to the next method on the
list by invoking {\sf call-next-method}. The arguments to {\sf
call-next-method} are the arguments to the next method on the list, 
so {\sf :around} methods can be used to modify, or canonicalize, the
arguments to other methods.
  
Using this mechanism, we can replace the {\sf (plus dimensioned-complex-number
dimensioned-complex-number)} method by the following
{\sf :around} method for {\sf dimensioned-object}.
\begin{code}
(defmethod plus :around ((z1 dimensioned-number) (z2 dimensioned-number)) 
  (if (eql (units z1) (units z2))
      (let ((sum (call-next-method z1 z2)))
        (setf (units sum) (units z1))
        sum)
      (error "Combining dimensioned numbers with different units.")))
\end{code}
  
This method checks to see if its two arguments have the same
dimensions. If so, it has the remaining applicable methods perform the
addition. It then updates the {\sf units} slot of the value to be
returned.
  
Unfortunately, this is not enough.  The basic {\sf plus} methods,
those for adding two {\sf complex-number}s or two {\sf
floating-number}s, will not return objects of the right class.  In
particular, they will not create instances of classes that include the
{\sf dimensioned-object} class.  However, this is easily fixed with a
slight change to the basic methods.  For instance, the {\sf
complex-number} {\sf plus} method should be modified to:
\begin{code}
(defmethod plus ((z1 complex-number) (z2 complex-number))
  (let ((z1-class (class-of z1))
        (new-class (if (eql z1-class (class-of z2)) z1-class
                       'complex-number)))}
    (make-instance new-class
                   :real (+ (real-part z1) (float-value z2)) 
                   :imag (imag-part z1)))
\end{code}
  
With these techniques, any class that admits addition can be extended
to a dimensioned class that also admits addition, and where addition
is properly coded. This use of ``around'' methods allows us to divide
the code into more orthogonal components than is possible with more
conventional functional programming approaches.
  
The process of building up methods from pieces, as was done in this
example, is called {\em method combination} and is a very powerful
tool. However, it can lead to extremely opaque and hard to follow code
if not used judiciously. There are many other features in CLOS,
including more sophisticated forms of method combination and the
metaobject protocol, that we have not touched on. However, we hope
that this quick introduction will make Weyl's internal organization
easier to digest.
  

\section{Domains}
\label{Domains:Sec}

The objects that are usually manipulated in algebraic computation
systems---integers, polynomials, algebraic functions, etc.---are
called ``domain elements'' in Weyl.  They are elements of algebraic
objects called {\em domains\/}.  Examples of domains are the ring of
(rational) integers $\Z$, the ring of polynomials in $x$, $y$ and $z$
with integer coefficients, $\Z[x, y, z]$, and the field of Gaussian
numbers, $\Q[i]$.

Both domains and domain elements are implemented as instances of CLOS
classes.  These classes are part of a conventional,
multiple-inheritance class hierarchy.  The class hierarchy used when
constructing domains includes classes corresponding to groups, rings,
fields and many other familiar types of algebraic domains.

This approach allows us to provide information about domains that is
often difficult to indicate if we could only specify information about
the type of the domain's elements.  For instance, in addition to the
class \keylisp{ring} we also provide classes like
\keylisp{integral-domain}.\footnote{An integral domain $R$ is ring
that does not have any zero divisors.  That is, for all $a, b \in R$,
if $ab = 0$ then either $a=0$ or $b=0$.} Whether a domain is an
integral domain or just a ring does not change how the domain's
elements are represented or the operations that can be performed on
them.  What it may affect is the algorithms used to implement the
operations.  As an example in a less mathematical context, consider
the difference between binary trees and balanced binary trees. 

In addition we provide information about the permissible operations
involving elements of the domain.  For instance, an
\keylisp{abelian-group} must have a \keylisp{plus} operation that can be
applied to pairs of its elements.  The result will be an element of
the abelian group.  This parallels the mathematical definition of a
group where a group is a pair $(G, \times)$ consisting of a set of
elements ($G$) and a binary operation ($\times$) that maps pairs of
elements of $G$ into $G$.  Contrast this to the typical definition of
a type as a set of objects that obeys a predicate.

We attach information about the permissible operations on the elements
of a domain to the domain itself.  This ensures that when an instance
of a domain is instantiated, the appropriate operations have been
provided.  This is especially valuable in an incremental development
environment like those common for Lisp.

Domain elements are also implemented as instances of CLOS classes, but
in addition they are associated with a particular domain.  This
association is implemented by a reference in the domain element to the
containing domain. The CLOS class structure provides the ``structural
type'' referred to earlier.

\begin{figure}
\begin{center}
\BoxedEPSF{Domains.eps scaled 812}
\end{center}
\caption{Domain Example\label{Domain:Ex:Fig}}
\end{figure}

\figref{Domain:Ex:Fig} illustrates the relationship between domains,
domain elements and CLOS classes for a simple example.  The polynomial
$x+1_{\Z[x]}$ is an instance of the CLOS class \keylisp{polynomial}.  The
domain $\Z[x]$ is an instance of the class \keylisp{polynomial-ring},
which inherits from \keylisp{Ring}, \keylisp{monoid} and
\keylisp{abelian-group}.  The polynomial $x+1_{\Z[x]}$ is an element
of the domain $\Z[x]$.  Notice that there is a complete type hierarchy
sitting above the class \keylisp{polynomial-ring}.

There are always two functions for obtaining a domain.  The one
that begins ``{\sf weyli::make-}'' always creates a new domain.  The one
beginning with ``{\sf get-}'' tries to find an isomorphic domain that
already exists.  Thus {\sf (get-rational-integers)} always returns the
same rational integer domain, while {\sf (make-rational-integers)}
always creates a new instance of the rational integers.

\subsection{Subdomains and Superdomains}

In some circumstances it is useful to be able to create a domain that
is a subdomain of a larger domain.  For instance, the positive
integers are viewed as a subdomain of the rational integers. The elements
of an ideal of ring form an additive group that is a subdomain of the
original ring.

\begin{genericdef}{super-domains-of}{domain}
Returns the superdomains that contain {\em domain}.
\end{genericdef}

At the moment, nothing uses this mechanism. The superdomain concept
is intended for dealing with deductive questions like: Are there any
zero divisors in the set of positive integers? Although the set of
positive integers is not an integral domain, it is a subsemigroup of an
integral domain and thus none of the positive integers is a zero
divisor.  Similarly, it is intended that an ideal of a ring $R$ will be
implemented as an $R$-module that is also a subdomain of $R$.

\section{Morphisms}
\label{Morphisms:Sec}

Morphisms are maps between domains.  They are first-class objects in
Weyl and can manipulated like domains and domain elements.  In
particular, two morphisms can be composed using the operation
\keylisp{compose}.  Morphisms are created using the function
\keylisp{make-morphism}, which takes three arguments. 

\begin{genericdef}{make-morphism}{domain mapping range \key (replace?
{\sf T})}
This function creates a morphism from {\em domain} to {\em range}.
{\em Mapping} is a function of one argument that takes an element of
{\em domain} and returns an element of {\em range\/}.  If {\em
replace?} is true, then any existing morphisms between {\em domain} and
{\em range} are deleted before the new morphism is created.
\end{genericdef}

All morphisms created are remembered in the \keylisp{*morphisms*}
data structure.  (Initially, this is just a list.)  To find any
existing homomorphisms we can use the function \keylisp{get-morphisms}. 

\begin{functiondef}{get-morphisms}{\key{} domain range}
When given no arguments this function returns a list of all the
morphisms Weyl currently knows about in this context.  When the {\em
domain} is provided, it returns all morphisms from {\em domain} to
anywhere.  Similarly, when {\em range} is provided, only those
morphisms that map into {\em range} are returned.  When both {\em
range} and {\em domain} are given, then only those morphisms from {\em
domain} to {\em range} are returned.
\end{functiondef}

Classes have been provided to indicate that more is known about the
map than that is a simple morphism.  In particular, a morphism can be
a homomorphism, an injection, a surjection, an isomorphism and an
automorphism.  At the moment nothing takes advantage of this information.

\section{Coercions}
\label{Coercions:Sec}

Explicit coercions are performed by the function \keylisp{coerce}:

\begin{genericdef}{coerce}{element domain}
Coerce finds an element of {\em domain} that corresponds with {\em
element}.  This is done using one of two methods.  First, there may be
a ``canonical'' coercion, which is one that is defined via explicit
``{\sf coerce}'' methods.  These methods take care of mapping Lisp
expressions, like numbers and atoms, into Weyl domains.  If there are
no canonical coercion methods then coerce checks to see if there is a
unique morphism between {\em element}'s domain and {\em domain}. If
so, this morphism is used to map {\em element} to {\em domain}. If
there is more than one morphism then an error is signaled.

If the the switch \keylisp{*allow-coercions*} is set to false ({\sf
nil}) then the canonical maps are the only ones that will be used by
\keylisp{coerce}.  However, if the user sets
\keylisp{*allow-coercions*} to {\sf T} then by creating a {\em
homomorphism} between two domains, the set of canonical maps between
domains can be extended.  If \keylisp{coerce} cannot find any other
predefined mapping between the domain of {\em element} and {\em
domain\/} it then searches the set of all defined homomorphisms.  If
there exists a {\em canonical homomorphism} between the two domains
then it is used to map {\em element} into {\em domain\/}. If there
does not exist a canonical homomorphism but there is exactly one
homomorphism between the two domains, then it is used.  
\end{genericdef}


As a general rule, Weyl does not provide for implicit coercions of
arguments to functions. Thus we assume that in the expression {\sf (+
a b)} the domains of {\sf a} and {\sf b} are the same. If this is
not the case, the user must explicitly insert a coercion.  The reason
for this is to deal with problems such as would arise if, for example,
{\sf a} were $1/2$ and {\sf b} were $x$. The domain of {\sf a} is
$\Q$, while the domain of {\sf b} is $\Z[x]$. The sum cannot be
represented in either domain, and in fact we have the choice of
embedding it in either $\Q[x]$ or $\Z(x)$. The wrong choice could be
lead to very inefficient algorithms. However, we do make one
exception. There is assumed to be a canonical homomorphism of the
rational integers into every domain.  If only one of the arguments to
one of the four basic arithmetic operations ({\sf +}, {\sf -}, {\sf *}
and {\sf /}) is an element of $\Z$, then it is automatically coerced
into the domain of the other argument.


\section{Hierarchy of Domains}
\label{Domain:Hierarchy:Sec}

\begin{figure}
\begin{center}
\BoxedEPSF{DomainHier.eps scaled 700}
\end{center}
\caption{Algebraic Domains\label{Algebraic:Domains:Fig}}
\end{figure}

The root of the domain hierarchy is the class \keylisp{domain}.  It
provides a number of utility routines for managing the set of
operations and axioms associated with a domain.  However, all
algebraic domains are built on the higher level class \keylisp{set}.

The basic algebraic domains are given in
\figref{Algebraic:Domains:Fig}.  The elements of a domain of class
\keylisp{set} can be compared using the binary operator
\keylisp{weyli::binary=}.  However, users will find it more convenient
to use the macro \keylisp{=}, which converts several arguments into a
sequence of calls to \keylisp{weyli::binary=}.  That is
\[
\mbox{\sf (= a b c)} \Longrightarrow
 \mbox{\sf (and (weyli::binary= a b)(weyli::binary= b c))}.
\]

\begin{genericdef}{weyli::binary=}{x y}
Test to see if {\em x} and {\em y} are equal.  Equality is meant in
the mathematical sense, so two vectors are {\sf =} if their
components are {\sf =}.
\end{genericdef}

\subsection{Semigroups, Monoids and Groups}

This section discusses domains that have one operation.  That is, they
consist of a set $S$ and a binary operation $\oplus$ such that for two
elements $a$ and $b$ in $S$, $a \oplus b$ is an element of $S$.  The
simplest interesting class of such domains is a \keyi{semigroup},
where $\oplus$ is assumed to be \keyi{associative}.  That is, for $a$,
$b$ and $c$ elements of $S$,
\[
(a \oplus b) \oplus c = a \oplus (b \oplus c).
\]

In Weyl, domains that are semigroups include the class
\keylisp{semigroup}.  In addition, the operation of a {\sf semigroup}
is assumed to be {\sf weyli::times}.\footnote{This is a limitation of
the current implementation of Weyl and will be fixed shortly.}  As
with {\sf weyli::binary=}, there is a macro {\sf *} that simplifies its
use.  For instance,
\[
\mbox{\sf (* a b c)} = 
\mbox{\sf (weyli::times (weyli::times a b) c)}.
\]
In addition, repeated multiplication can be indicated by
\keylisp{expt}.  That is,
\[
\mbox{\sf (expt a 3)} = 
\mbox{\sf (weyli::times (weyli::times a a) a)},
\]
although it may be computed more efficiently for elements of some domains.

\noindent\functdef{*}{\rest args}
\begin{genericdef}{weyli::times}{x y}
Returns the product of two elements of a \keylisp{semigroup}.
\end{genericdef}

If one wants to compute the product of all the elements of a lists, a
convenient idiom is to ``{\sf apply}'' the ``{\sf *}'' function to the
list.  This is not possible with the Weyl ``{\sf *}'' operator since
it is a macro, not a function.  As a work around for this problem,
Weyl also provides a function for multiplication caled ``{\sf
\%times}.''  Semantically, this function is the same as the ``{\sf
*}'' macro, but the code that would be generated if ``{\sf \%times}''
were used in place of ``{\sf *}'' is not as efficient.\footnote{The
need for this function can be eliminated by compiler macros, but this
solution cannot currently be implemented in a portable manner.}

\functdef{{\percent}times}{\rest args}

\medskip
\keyi{Monoid}'s are semigroups that contain a multiplicative
identity.  This element is called \keylisp{one} and may be accessed
by applying the function \keylisp{one} to the domain. 

\begin{genericdef}{one}{semigroup}
Returns the multiplicative identity of {\sf monoid}.
\end{genericdef}

\begin{genericdef}{1?}{elt}
Returns true if its argument is the multiplicative identity and false
otherwise.
\end{genericdef}

In addition, the \keylisp{expt} function is extended so if its second
argument is 0, it returns the multiplicative identity.

The elements of a \keylisp{group} have multiplicative inverses which can
be determined using {\sf recip}

\begin{genericdef}{recip}{elt}
Returns the multiplicative inverse of {\em elt\/}.
\end{genericdef}

For efficiency reasons it often valuable to have a binary operation
that multiplies an element by the multiplicative inverse of the
second argument.  This operation is called \keylisp{weyli::quotient}.
The macro {\sf /} can be used to simplify its use:
\[
\begin{eqalign}
\mbox{\sf (/ a)}& =  \mbox{\sf (recip a)} \\
\mbox{\sf (/ a b)}& =  \mbox{\sf (weyli::quotient a b)} \\
\mbox{\sf (/ a b c)}& 
   =  \mbox{\sf (weyli::quotient (weyli::quotient a b) c)}
\end{eqalign}
\]

In a group, the \keylisp{expt} operation is extended to apply to
negative second arguments also.  In this case
\[
\mbox{\sf (expt x $-n$)} = \mbox{\sf (expt (recip x) $n$)}
\]

\begin{genericdef}{expt}{a n}
Multiplies $a$ by itself $n$ times.  If the domain of $a$ just a
semigroup, then $n$ must be a positive Lisp integer.  When the domain
of $a$ is a monoid, then $n$ can be equal to zero.  If the domain of
$a$ is a group then $n$ can be negative. 
\end{genericdef}

A binary operation $\oplus$ on a semigroup $S$ is \keyi{commutative}
if for every pair of elements in $a$ and $b$ in $S$, $a \oplus b = b
\oplus a$.  The domains \keylisp{abelian-semigroup}, \keylisp{abelian-monoid}
and \keylisp{abelian-group} are similar to the domains
\keylisp{semigroup}, {\sf monoid} and \keylisp{group} except the
binary operation is \keylisp{plus} instead of \keylisp{times}.  This
is the biggest distinction between the domain structure of Weyl and
the domains of modern algebra.\footnote{It would be better if any
operation could be declared to be commutative.}

\noindent\functdef{+}{\rest args}\\
\functdef{{\percent}plus}{\rest args}
\begin{functiondef}{weyli::plus}{x y}
Returns the sum of {\em x} and {\em y\/}, which are elements of an
\keylisp{abelian-semigroup}.  {\sf weyli::plus} is a commutative binary
operation.
\end{functiondef}
The multiple argument version of \keylisp{weyli::plus} is \keylisp{+},
which is what should be used in all applications.

\begin{functiondef}{zero}{abelian-semigroup}
Returns the additive identity of an abelian semigroup.
\end{functiondef}

\noindent\functdef{zero?}{elt}
\begin{functiondef}{0?}{elt}
Returns true if its argument is the additive identity and false
otherwise. 
\end{functiondef}

The additive inverse of an element of an \keylisp{abelian-group} can
be obtained using \keylisp{weyli::minus}.  As in the multiplicative
case, there is a binary operation, \keylisp{weyli::difference} and a
macro {\sf -} that makes these operations easier to use.
\[
\begin{eqalign}
\mbox{\sf (- a)}& =  \mbox{\sf (weyli::minus a)} \\
\mbox{\sf (- a b)}& =  \mbox{\sf (weyli::difference a b)} \\
\mbox{\sf (- a b c)}& 
   =  \mbox{\sf (weyli::difference (weyli::difference a b) c)}
\end{eqalign}
\]

The function ``{\sf \%difference}'' is provided for use when a
multi-argument version of the ``{\sf -}'' should be passed to another
function.  The same caveats about efficiency apply here.

Notice that algebraically we consider all of these domains to possess
only one operation {\sf weyli::times}, or {\sf weyli::plus} in the
abelian case.  The other operations are either abbreviations, \eg,
{\sf expt}, or are constructive versions of existential axioms.  For
instance, a monoid $M$ obeys the axiom: There exists an element $e$ in
$M$ such that all for elements $a$ of $M$, $e\cdot a = a\cdot e = a$.
The function {\sf one} just returns the element $e$.


\subsection{Rings}
A ring is a triple $(R, +, \times)$ such that $(R, +)$
is an abelian grop and $(R, \times)$ is a monoid.  In addition,
$\times$ distributes over $+$.  That is, for all $a$, $b$ and $c$ in
$R$
\[
a\times (b + c) = (a\times b) + (a \times c)
\quad\mbox{and}\quad
(a + b) \times c = (a \times c) + (b \times c).
\]
A \keyi{commutative ring} is a ring where the operation $\times$ is
commutative.  In many parts of algebra and geometry the only rings
that occur are commutative rings, and more often than not,
introductory books contain the phrase ``throughout, by ring we mean a
commutative ring with identity.''  In fact, this is what Weyl means by
the class \keylisp{ring}.  Instances of the class \keylisp{rng}
satisfy only the stricter (broader) definition of a ring.

\begin{figure}
\begin{center}
\begin{tabular}{|l|p{4in}|} \multicolumn{2}{c}{}\\ \hline 
$\Z, \Q, \R, \C$ & The ring of rational integers, the field of rational
numbers, the field of real numbers and the algebraically closed field
of complex numbers. \\ \hline
$R[x, y, z]$ & The ring of polynomials in $x$, $y$ and $z$ whose
coefficients lie in $R$. \\ \hline
$R(x, y, z)$ & The ring of rational functions (quotients of two
polynomials) in $x$, $y$ and $z$ whose coefficients lie in the ring
$R$. \\ \hline
$R[[t]]$ & The ring of univariate power series with non-negative
integral exponents in $t$ whose coefficients lie in the ring $R$. \\
\hline 
$R((t))$ & The ring of univariate power series with integral exponents
in $t$ whose coefficients lie in the ring $R$. \\  \hline 
\end{tabular}
\end{center}
\caption{Different Ring Notations\label{RingNotation:Fig}}
\end{figure}

Examples of rings occur throughout algebra.  The table in
\figref{RingNotation:Fig} gives some examples of rings and fixes the
notation that is commonly used.  Additional ring structures are
discussed in \chapref{Quotients:Chap} and
\sectref{DifferentialRings:Sec}.  More baroque structures often arise
in commutative algebra.  For instance, if $S$ is a subring of the ring
$R$, then the set of polynomials whose constant term lies in $S$ is a
ring, $R[S)$, called the ring of {\em half open polynomials}.

For some rings, when the multiplicative identity element is added to
itself a number of times we are left with zero.  That is,
\[
\overbrace{1+ 1+ \cdots + 1}^{p {\rm ~times}} = 0.
\]
In this case the smallest positive $p$ which has this property is
called the \keyi{characteristic} of the ring.  If the ring is an
integral domain, then it is not hard to prove that $p$ must be a prime
number.  If there is no such $p$, then we say the characteristic of the
ring is zero.  The function \keylisp{characteristic} returns the
characteristic of a ring. 

If all of the non-zero elements of a \keylisp{ring} have
multiplicative inverses, then the ring is a \keylisp{field}.

\subsection{Modules}
\label{ModuleDomain:Sec}

A module is an abelian group whose elements can be acted upon by
elements of a ring.  That is, if $M$ is an $A$-module then $M$ is an
abelian group, and if $x$ is an element of $M$ and $a$ is an element of
$A$, then $a\cdot x$ is also an element of $M$.  Furthermore, for all $x$
and $y$ in $M$ and $a$ and $b$ in $A$ we have
\begin{eqaligntwo*}
a \cdot (x + y) &= a \cdot x + a \cdot y,
& a \cdot x + b \cdot x &= (a + b) \cdot x, \\
a \cdot (b \cdot x) &= (ab) \cdot x,
& 1\cdot x &= x.
\end{eqaligntwo*}

If this is the case, $M$ is said to be a (left) $A$-module.  Similarly,
we can define right $A$-modules.  A ring that is both a left and a
right $A$-module is simply called an $A$-module.  If $G$ is an
abelian group then it is trivially a $\Z$-module by the action
\[
n\cdot a \mapsto \overbrace{a + a + \cdots + a}^{n {\rm ~times}},
\]
and $(-n) \cdot a \rightarrow n \cdot (- a)$ is used to define the
action for negative $n$.  The integer $0$ sends all elements of $M$ to
the additive identity of $M$. 

There are numerous examples of modules in mathematics.  A vector space
with with real coefficients is a $\R$-module.  The elements of an
ideal of a ring $R$ form an $R$-module.  In Weyl, all domains that are
instances of the class \keyi{module} represent modules.\footnote{At
this time we do not distinguish between left and right modules.}  If
$M$ is an $A$ module, then we call $A$ the \keyi{coefficient domain}
of $M$.

\begin{functiondef}{coefficient-domain-of}{module}
Returns the coefficient domain of {\em module}. 
\end{functiondef}

Note that classes are also defined for \keylisp{free-module} and
\keylisp{vector-space}.\footnote{A {\em vector space} is a free module
over a field.}  These classes and the operations available for
manipulating their elements are discussed in more detail in
\sectref{FreeModules:Sec} and \sectref{VectorSpace:Sec}.


\subsection{Properties}

There are a variety of different properties that algebraic objects can
possess.  Many of these properties are important in deciding which
algorithms should be used in a computation.  However, these properties
can sometimes be rather difficult to determine.  For instance, an
algebraic number ring may or may not be a unique factorization domain.
Determining whether it is is quite difficult and we would like to have
the option of delaying this determination until it is absolutely
necessary.

Although we could have indicated these properties by the class of the
domains, and once did, we currently use a different approach.  These
properties are attached to domains (via their property lists), so that
it is not necessary to determine the state of the properties when the
domain is created.\footnote{In fact all the mathematical properties
should probably be managed in this way.  Taking this approach would
compromise some of the modularity in creating some domains, so we
haven not taken this step yet.}

The properties that are currently defined are given in
\figref{MathProperties:Fig}.  We have given the predicates that can be
used to determine if the property holds of a particular domain.


\begin{figure}
\begin{center}
\begin{tabular}{|l|p{4in}|} \multicolumn{2}{c}{}\\ \hline
\keylisp{complete-set?}& Does the limit of all Cauchy sequences exist
in this domain?  In practice this means that floating point numbers
can be coerced into this domain.\\ \hline 
\keylisp{ordered-domain?}& Can $>$ be applied to the elements of this
domain \\ \hline 
\keylisp{integral-domain?} & If true, this domain does not contain any
zero divisors. \\ \hline 
\keylisp{euclidean-domain?} & If true, greatest common divisors can be
defined in this domain. \\ \hline
\keylisp{gcd-domain?} & If true, not only is this domain euclidean,
but a \keylisp{gcd} function is actually defined for its elements. \\ \hline
\keylisp{unique-factorization-domain?} & If true, the elements of this
domain can be factored uniquely.  \\ \hline
\end{tabular}
\end{center}
\caption{Mathematical Properties\label{MathProperties:Fig}}
\end{figure}

 
%\section{Hierarchy of Conditions}

%What are the conditions that are raised by the algebra package.  All
%error conditions should be replaced by conditions.  Part of the
%conditions should be able to answer the question ``what does this
%error mean?''  Need to work out the hierarchy of conditions.  In the
%current system there will be very few conditions, since most of what
%we have is fairly low level.

