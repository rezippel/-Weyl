\chapter{Algebraic Structures}
  
This chapter begins with a discussion
of Weyl's representation of ideals and spectrums of
rings. Taking the quotient of these structures and their
rings gives leads to domains that can be used to represent
algebraic numbers and algebraic functions.

\section{Ideals}
  
Recall that an ideal of a commutative ring $R$ is an additive
subgroup of $R$ such that the product of any element of the ideal by
an element of $R$ is an element of the ideal. Thus the ideal is an $R$
module.
  
Ideals are a slightly unusual objects in Weyl since they are treated
like domain elements (arithmetic operations can be applied to ideals),
but are in fact domains.  One might think that they should be both
domains and domains elements, and perhaps they will be in the future,
but doing so now would be problematic. If they were domain elements,
they would have to be elements of some domain and there is no natural
domain that would suffice.  (The spectrum of the ring only includes
the prime ideals, and additional possesses a topology.)
  
\begin{figure}
\begin{center}
\BoxedEPSF{IdealClass.eps scaled 500}
\end{center}
\caption{Ideal Class Structure\label{Ideal:Classes:Fig}}
\end{figure}

The set of classes used to represent ideals is given in
\figref{Ideal:Classes:Fig}.  All ideals are subclasses of the class
ideal. Every ideal is represented by a (not necessarily minimal) list
of elements that generate the ideal. Currently, Weyl is only capable
of representing ideals that are finitely generated. Ideals can be
created using the following function:
  
\begin{functiondef}{make-ideal}{domain \rest elements}
  
Creates the ideal of the ring domain, which is generated
by the polynomials {\em polys}. The polynomials must
all come from the same ring. Initially, we have only
implemented the Gr\"{o} bner basis algorithm for polynomials
over fields, the domain of the polynomials must be a
polynomial ring over a field.

\end{functiondef}
  
There are two subclasses of the class ideal: \keylisp{PID-ideal} and
\keylisp{grobner-basis}. If the ring of the ideal is a principal ideal
domain, then \keylisp{make-ideal} will create an instance of
\keylisp{PID-ideal}. The ring  of the ideal is a polynomial ring then
a \keylisp{grobner-basis} ideal will be created.\footnote{The name of
the class {\sf grobner-basis} is very poor and should be changed so
that it doesn't refer to a particular algorithm.}

Two basic arithmetic operations are provided for manipulating ideals:
addition and multiplication.  Given the ideal ideals $A = (a_1,
\ldots, a_m)$ and $B = (b_1, \ldots, b_n)$, their sum is defined to be
\[
A+B = (a_1, \ldots, a_m, b_1, \ldots, b_n)
\]
and their product is defined to be 
\[
A \cdot B  =
 (a_1 b_1, a_1, b_2, \ldots, a_1 b_n, a_2 b_1,\ldots, a_m b_{n-1}, a_m b_n).
\]
  
The standard {\sf +} and {\sf *} operators in Weyl are thus overloaded
to work with ideals also. This is illustrated in the following simple
examples:
  
\begin{code}
> (setq a (make-ideal (get-rational-integers) 12))
#Id(12)
  
> (setq b (make-ideal (get-rational-integers) 15))
#Id(15)
\end{code}
  
An ideal is printed as a parenthesized list of its generators. Since
we know that $\Z$ is a principal ideal domain, the list of generators
will only have one element.\footnote{There are a lot of very delicate
issues here.  The test for whether or not the ring is a PID is
generally quite difficult.  In the current implementation, Weyl
explicitly checks for $\Z$.  Even if one knows the ring is a PID, it
can be difficult to determine the ideal's generator.  The current
implementation assumes that in addition to being a PID there is an
explicit GCD algorithm available.}
 
\begin{code}
> (+ a b)
#Id(3)

> (* a b)
#Id(180)
\end{code}
  
For a more sophisticated example consider:
\begin{code}  
> (setq r (get-polynomial-ring (get-rational-numbers) '(x y)))
Q[x, y]
  
> (setq a (make-ideal r (+ (* 'x 'x 'x) (* 'y 'y)) (* 'x 'y)))
#Id( x y,  x^3 +  y^2)

> (setq b (+ a (* a a)))
#Id(x^2 y^2, x^4 y + x y^3, x^4 y +  x y^3, x^6 + 2 x^3 y^2 + y^4, x^3 + y^2, x y)
\end{code}

Although the last two results do exhibit bases for a ideals, they are
not the minimal bases. A reduced basis can be obtained by applying the
following routine.
  
\begin{functiondef}{reduce-basis}{ideal}
Reduces the basis provided for the ideal using the
techniques available.  For polynomials ideals this means
computing the Gr\"{o}bner basis assuming a monomial ordering
has been provided. 
\end{functiondef}  

When this routine is applied to the previous example we have:
\begin{code}
> (reduce-basis a)
#Id( x y,  x^3 +  y^2,  y^3)
  
> (reduce-basis b)}
#Id( x y,  x^3 +  y^2,  y^3)
\end{code}
  
So we see that {\sf a} is equal to {\sf (+ a (* a a))}. Further notice
that the reduce basis has more elements that the original basis.
  
  
\section{The Spectrum of a Ring}
  
This section needs to discuss the topological structure of the
spectrum as well.
  
  
\begin{functiondef}{spectrum}{Ring}
  This function returns a new domain which is the spectrum
of {\em ring}. The elements of this domain are the ideals
of {\em ring}. (Need to figure out how the topology
of Spec {\em R} fits into all this.) 
\end{functiondef}
  
\section{Algebraic Extensions}

\begin{functiondef}{algebraic-degree}{domain1 domain2}
This function checks to ensure that {\em domain2} is a subring of {\em
domain1} and, if so, returns the algebraic degree of {\em domain2}
over {\em domain1}.
\end{functiondef}

\section{Algebraic Closures}

Need to special case the algebraic closure of the real numbers.  This
is the only algebraic closure that is of finite degree.

