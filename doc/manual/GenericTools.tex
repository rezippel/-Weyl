% -*- Mode: LaTeX -*-
% $Id: GenericTools.tex,v 1.4 1995/03/31 15:31:15 rz Exp $

\chapter{Generic Tools}
\label{GenericTools:Chap}
  
Some facilities needed by symbolic computing routines are more
generally useful. For instance, it is occasionally necessary to
perform some function for every permutation of the elements of a set.
To capture this idiom, Weyl provides a new control structure, called
{\sf permute}. This control structure, and other combinatorial control
structures, are described in \sectref{CombinTools:Sec}. Other data and
control structures that are not directly tied to mathematical
computation are given in the other sections.

\section{Combinatorial Tools}
\label{CombinTools:Sec}
  
Weyl provides control abstractions that are often useful for
computations that involve combinatorial enumeration and
searching. These control structures are {\sf permute} and {\sf choose},
which enumerate all permutations of a sequence and all subsets of a
fixed size of a sequence.
  
\begin{specialdef}{permute}{sequence (var . options) \body body}
{\em sequence} is a sequence of elements.\footnote{At the moment this
control structure is only implemented for lists.}  The variable {\em
var} is repeatedly bound to the different permutations of {\em
sequence}, and {\em body} is evaluated each time. The options are
provided to specify even and odd permutations, but are not at this
point implemented.
\end{specialdef}
  
For example, the following code will print all permutations of the
list {\sf (a b c)} and count their number:
\begin{code}
> (let ((count 0))
    (permute ' (a b c) (p)
      (print p)
      (incf count)) 
    (format t "~%~D permutations total. ~%" count))
  
(C B A)
(B C A)
(C A B)
(A C B)
(B A C)
(A B C)
6 permutations total.
NIL
\end{code}
  
\begin{specialdef}{choose}{set (var n . options) \body body}
While the first argument to {\sf permute} must be ordered, the first
argument to {\sf choose} need only be a set (but again, only lists are
currently implemented). The variable {\em var} is bound to each subset
of {\em set} that has precisely {\em n} elements and {\em body }is
evaluated in each case. At the moment no options are permitted.
\end{specialdef}
  
A \keyi{partition} of a positive integer $n$ is a representation of $n$
as a sum of positive integers. The following control structure is used
to enumerate partitions.
  
\begin{specialdef}{partition}{(var n . options) \body body}
The argument {\em n} is assumed to be an integer.
This control structure repeatedly binds {\em var} to
additive partitions of {\em n} and then evaluates the
{\em body}.  The options allow one to control which partitions
of {\em n} are produced. The options are

\begin{center}
\begin{tabular}{lp{4in}}
{\tt :number-of-parts} & The number of parts each partition is
allowed to contain. \\
{\tt :minimum-part} & The minimum value each for each of the components
of the partition. \\
{\tt :maximum-part} & The maximum value each for each of the components
of the partition. \\
{\tt :distinct?} & If specified as {\tt T} then each of the
components of the partition must be distinct. 
\end{tabular}
\end{center}

\noindent
{\sf Partition} returns no values.
\end{specialdef}
  
The following examples illustrate the use of the partition control
structure. First, compute all of the partitions of 6:
\begin{code}
> (partition (l 6) (print l))
(1 1 1 1 1 1)
(2 1 1 1 1)
(3 1 1 1)
(2 2 1 1)
(4 1 1)
(3 2 1)
(5 1)
(2 2 2)
(4 2)
(3 3)
(6)
\end{code}
  
\noindent
Now restrict the partitions to those that do not contain $1$'s or
those that consist of precisely $3$ components:
\begin{code}  
> (partition (l 6 :minimum-part 2)
    (print l))
(2 2 2)
(4 2)
(3 3)
(6)
  
> (partition (l 6 :number-of-parts 3)
    (print l))
(4 1 1)
(3 2 1)
(2 2 2)
\end{code}
  
We can further restrict the partitions to only include components that
contain components no larger than $3$ and to those partitions that
consist of distinct components.
\begin{code}  
> (partition (l 6 :number-of-parts 3 :maximum-part 3)
    (print l))
(3 2 1)
(2 2 2)
  
> (partition (l 6 :number-of-parts 3 :maximum-part 3 :distinct? t)
    (print l)) 
(3 2 1)
  
> (partition (l 6 :distinct? t)
    (print l))
(3 2 1)
(5 1)
(4 2)
(6)
\end{code}
  
It is well known, and easy to prove, that the number of partitions of
an integer $n$ into $m$ parts is equal to the number of partitions of
$n$ where the largest component of the partition is $m$. We can check
this numerically with the following functions. The first function
counts the number of partitions of $n$ with exactly $m$ parts. For
this the {\sf :number-of-parts} option suffices.
\begin{code}
(defun partition-count-fixed-parts (n m)
  (let ((cnt 0))
    (partition (part n :number-of-parts m)
      (declare (ignore part))
      (incf cnt))
      cnt))
\end{code}
  
The function {\sf partition-count-exact-max} computes the
number of partitions where the maximum component is
exactly {\sf m}. In this case, the {\sf :maximum-part} option helps
filter the partition, but then an additional test needs
to be applied to ensure that each partition actually
has an element of size {\sf m}.
\begin{code}
(defun partition-count-exact-max (n m)
  (let ((cnt 0))
    (partition (part n :maximum-part m)
      (when (= m (apply #'cl::max part))
        (incf cnt)))
    cnt))
\end{code}
  
Finally we provide a routine for testing the functions given above.
\begin{code}
(defun partition-test-1 (n m)
  (let ((part-count1 (partition-count-fixed-parts n m))
        (part-count2 (partition-count-exact-max n m)))
    (list (= part-count1 part-count2)
          part-count1 part-count2)))
  
> (partition-test-1 10 3)
(T 8 8)

> (partition-test-1 15 3)
(T 19 19)
  
> (partition-test-1 15 4)
(T 27 27)
\end{code}
  
\section{Memoization}
\label{Memoization:Sec}
  
It often occurs that the straightforward way of expressing an
algorithm is quite inefficient because it performs a great deal of
recomputation of values that were previously computed.  The simplest
example along these lines is the Fibonacci function
\begin{code}
(defun fib (n)
  (if (< n 2) 1
      (+ (fib (- n 1)) (fib (- n 2)))))
\end{code}
  
Due to the double recursion, this implementation can take time
exponential in {\sf n}. By caching values of {\sf (fib n)} to avoid
recomputation, it is possible to reduce this to a computation that is
linear in {\sf n}. This is easily done using the control abstraction
{\sf memoize}. For instance, we have the following efficient version
of the Fibonacci function:
 
\begin{code}
(defun fib-memo (n)
  (memoize `(fib ,n) 
    (if (< n 2) 1 
        (+ (fib-memo (- n 1)) (fib-memo (- n 2))))))
\end{code}

The \keylisp{memoize} form should be understood as follows.  The first
argument is an expression that identifies a particular computation.
The above example, the form {\sf `(fib ,n)} represents the computation
of the $n$-th Fibonacci number.  The body of the \keylisp{memoize}
form contains the code that actually performs the computation.  The
\keylisp{memoize} form checks a cache to see if the computation
indicated by the first argument has been previously performed.  If so,
the value returned by the earlier computation is returned immediately.
If not the body is executed and the value cached for future use.
  
The cache used by the \keylisp{memoize} control structure is handled
by the class \keylisp{weyli::has-memoization}. The domain of general
expressions (discussed in \chapref{General:Chap}) is always a subclass
of \keylisp{weyli::has-memoization} and is the domain with which
memoization is usually associated.  The internal function
\keylisp{weyli::{\percent}memoize} allows one to associate a memoization with
any subclass of \keylisp{weyli::has-memoization}.

\begin{methoddef}{weyli::{\percent}memoize}{domain expression \body body}
Each time this form is executed, it checks to see if {\em expression}
is cached in {\em domain}'s memoization cache.  If so, the value in
the cache is returned without executing the body. If {\em expression}
is not found in the cache, then the forms in body are evaluated, and
the value of the last form is both returned and saved in {\em
domain}'s memoization cache.
\end{methoddef}
  
It is usually much more convenient to use the {\sf memoize} control
structure:
  
\begin{specialdef}{memoize}{expression \body body}
Performs the same functions as {\sf weyli::\%memoize} except
that the domain used is {\sf *general*}. 
\end{specialdef}
  
One should note that the expression which is used as the index of the
cache cannot be a constant.  It should contain all of the information
that can influence the value returned by the body.  Also, at this
writing, only the first value returned by the expression being
memoized is actually cached. The other values are ignored.
  

\section{Tuples}
  
Instances of the class \keylisp{weyli::tuple} are simply
one-dimensional vectors. Generally, one doesn't create instances of
bare tuples, but rather includes the \keylisp{weyli::tuple} class with
some other classes to create more interesting objects. For instance,
Weyl vectors are instances of a class that than includes tuple and
domain-element.  When creating an instance of a \keylisp{weyli::tuple}
one needs to initialize a slot that contains the values of the tuple.
This can be done by using the {\sf :values} initialization keyword.
For instance,
\begin{code}
> (setq tup (make-instance 'weyli::tuple :values '(1 2 3)))
<1, 2, 3>
\end{code}
  
\noindent
The initialization value can be either a list or a Lisp vector.
  
Since tuples are instances of a class, various methods can be
overloaded to work with them. The simplest such function is {\sf
length}, which computes the number elements in the tuple. Another
useful function is {\sf ref}, which accesses different elements of the
tuple by their index.  The generic function {\sf ref} is used in place
of the Common Lisp functions {\sf aref} or {\sf svref}.
  
\begin{macrodef}{ref}{sequence \rest indices}
Accesses the indicated elements of sequence. Tuples
are one-dimensional arrays so only one index is allowed.
The indexing scheme is zero based, so the first element
of the tuple has index $0$, the second $1$ and so on.
  
For example,
\begin{code}
> (list (ref tup 0) (ref tup 2) (ref tup 1))
(1 3 2)
\end{code}
\end{macrodef}  

It is sometimes useful to be able to convert a tuple into a list of
its elements. This can be done with the following function:
  
\begin{methoddef}{list-of-elements}{tuple}
Returns a list of the elements of {\em tuple}.  For example,
\begin{code}
> (list-of-elements tup)
(1 2 3)
\end{code}
\end{methoddef}
  
The following functions extend the Common Lisp mapping functions to
work with tuples as well as the usual sequences of Common Lisp.
  
\begin{methoddef}{map}{type function tuple \rest sequences}
The number of arguments of {\em function} is
expected to be one more than the number of elements
in {\em sequences}.  {\em function} is applied to each element of
{\em tuple} and the corresponding elements of each of the
elements of {\em sequences}. For instance,
\begin{code}
> (map 'tuple #'cons tup tup)
<(1 . 1), (2 . 2), (3 . 3)>
\end{code}
\end{methoddef}

Algebraic objects in Weyl have a slot that indicates the algebraic
domain of which the object is an element.  When creating an instance
of such an object it is necessary to indicate this domain.  If the
sequence returned by {\sf map} requires this information then the
domain will be extracted from the tuple.  If it is necessary to
explicitly specify the domain of the resulting tuple, the following
function may be used.
  
\begin{methoddef}{map-with-domain}{type domain function tuple \rest sequences}
Similar to {\sf map} but the domain of the resulting tuple will be
{\em domain}.
\end{methoddef}

\section{Arithmetic with Lists}

In Lisp, lists are extremely convenient data structures, especially
for temporary results in computations.  To make them even more
useful, Weyl provides a couple of functions for performing arithmetic
operations with lists.  These routines use the Weyl arithmetic
operations, and thus can be used both for arrays and lists that contain
Lisp numbers and those those that contain Weyl's mathematical objects
[{\em This section needs to be fleshed out.  I believe that there are
many similar operations that should be added.  Perhaps we should just
overload the standard arithmetic operations?  --RZ}] \quad
[{\em Or we could do something like the {\tt Listable} attribute
in Mathematica. --TW}]


\begin{methoddef}{list-inner-product}{list1 list2}
Computes the sum of the pairwise product the elements of {\em list1} and
{\em list2}. 
\end{methoddef}

\begin{methoddef}{list-expt}{base-list expt-list}
Returns a list of consisting of the elements of {\em base-list} raised
to the power of the corresponding element of {\em expt-list}.
\end{methoddef}

\begin{methoddef}{array-times}{array1 array1}
Checks to make sure the arguments of the of the right dimensions and
then computes a matrix product storing the result in a new,
appropriately sized matrix.
\end{methoddef}


\section{AVL Trees}
  
  

