% -*- Mode: LaTeX -*-
% $Id: Scalars.tex,v 1.3 1995/03/31 15:31:17 rz Exp $
\chapter{Scalar Domains}
\label{Scalar:Chap}

Almost all domains that occur in mathematics can be constructed from
the rational integers, $\Z$.  For instance, the the rational numbers
($\Q$) are the quotient field of $\Z$, the real numbers are completion
of $\Q$ using the valuations at infinity and so on.  For performance
reasons provides direct implementations of four basic scalar
domains, the rational integers ($\Z$), the rational numbers ($\Q$), the
real numbers ($\R$) and the complex numbers ($\C$).  In addition, a
similar direct implementation of finite fields is also provided.
These scalar domains are called {\em numeric domains}, and the
corresponding domain hierarchy is given in \figref{NumberDomains:Fig}.
Instances of the {\sf GFp} class are finite fields of $p$ elements.
(These are parameterized domains.)  The {\sf GFq} class is used to
implement fields of $q= p^r$ elements.  The class {\sf GFm} is
discussed in more detail in \sectref{Finite:Fields:Sec}.

\begin{figure}
\begin{center}
\BoxedEPSF{NumberDomains.eps scaled 500}
\end{center}
\caption{Mathematical Scalar Domains\label{NumberDomains:Fig}}
\end{figure}

Instances of the characteristic zero domains may be created using the
functions \keylisp{get-rational-integers},
\keylisp{get-rational-numbers}, \keylisp{get-real-numbers} and
\keylisp{get-complex-numbers}.  To create finite fields one can use
the function \keylisp{get-finite-field}.  Depending on its argument,
which must be a power of a single prime, this function will return a
domain of type {\sf GFp} or {\sf GFq}.

Elements of these domains are implemented using the hierarchy of
structure types shown in \figref{NumberElements:Fig}.  The elements of
a rational integer domain are all of structure type
\keylisp{rational-integer}.  However, some numeric domains may contain
elements of different structure types.  For instance, objects of
structure type \keylisp{rational-integer} as well as
\keylisp{rational-number} can be elements of the rational number
domains ($\Q$).  This approach allows us to represent exactly integers
and rational numbers in $\R$ and $\C$.  In the future, a similar
approach will allow us to represent algebraic and transcendental
numbers exactly.

\begin{figure}
\begin{center}
\BoxedEPSF{Scalars.eps scaled 500}
\end{center}
\caption{Mathematical Scalars\label{NumberElements:Fig}}
\end{figure}

One of the complications Weyl must deal with is that Lisp has its own
model of the numbers, which must co-exist with Weyl's model.  The type
structure used by Lisp is given in \figref{LispNumbers:Fig}.  To
simplify use of Weyl, we allow users to create Weyl numbers from Lisp
numbers, and to incorporate Lisp numbers into their code by providing
some automatic coercions.  Thus adding a Lisp number to an element of
$\R$ causes the Lisp number to be coerced to an element of $\R$ of
structure {\sf rational-integer}.

\begin{figure}
\begin{center}
\BoxedEPSF{LispNumbers.eps scaled 500}
\end{center}
\caption{Lisp Numbers\label{LispNumbers:Fig}}
\end{figure}

There are several different ways to determine if an object is a
scalar.  

\begin{center}
\begin{tabular}{ll}
{\sf (typep {\em obj} 'cl:number)}\index{number@{\sf cl:number}, a
Lisp type}& a Lisp number, \\ 
{\sf (typep {\em obj} 'numeric)} \index{numeric@{\sf numeric}, a
Lisp type}& a Weyl number, \\ 
{\sf (number? {\em obj})}\index{number?@{\sf number?}}& 
    either a Lisp number or a Weyl number.
\end{tabular}
\end{center}

The following sections are organized by the different structure types.
Sections~\ref{Rational:Integer:Sec} through \ref{Finite:Fields:Sec}
deal with rational integers, rational numbers, real numbers, complex
numbers and elements of finite fields respectively.  In each section
we discuss the different domains these structure elements can be used
in followed by a discussion of the operations that can be performed
with each structure type.

\section{Rational Integers}
\label{Rational:Integer:Sec}

The rational integers are the integers of elementary arithmetic:
\[
\Z = \{\ldots, -3, -2, -1, 0, 1, 2, 3, \ldots\}.
\]
Other than limitations on the memory of the host computer, there is no
limitation on the size of the elements of $\Z$.  The term {\em
rational} integer is used to distinguish this domain from other
domains of {\em algebraic} integers, \eg, $\Z[(1+\sqrt{5})/2]$.

A domain of rational integers can be created using the following function.

\begin{functiondef}{get-rational-integers}{}
Returns a domain that is isomorphic to the rational integers, $\Z$.
When called repeatedly, it always returns the same value until
\keylisp{reset-domains} is called. 
\end{functiondef}

\noindent
Most of the time, there only needs to be one rational integer domain.
The domain of rational integers is a euclidean domain.

Elements of structure type \keylisp{rational-integer} can be elements
of domains of type \keylisp{rational-numbers}, \keylisp{real-numbers}
and \keylisp{complex-numbers}.  Equivalently, if {\em domain} is a
domain that admits elements of structure type
\keylisp{rational-integer}, then one invoke \keylisp{make-element}
with {\em domain} and a Lisp integer.

\medskip
Rational integers are most easily created by coercing a Lisp integer
to a rational integer domain using the function \keylisp{coerce}.
Furthermore, the usual arithmetic routines ({\tt +}, {\tt -}, {\tt *},
{\tt /} and {\tt expt}) work with rational integers that are 
elements of the same domain.  If the domain is a field then {\tt /}
may return a \keylisp{rational-number}. 

For instance, the following routine could be used to compute
factorials.
\begin{code}
(setq *Z* (get-rational-integers)) 

(defun factorial (n)
  (if (< n 2) (one *Z*)
      (* (coerce n *Z*) (factorial (- n 1)))))
\end{code}

\noindent
Notice that the unit element of {\sf *Z*} was created by using the
function {\sf one}, rather than {\sf (coerce 1 *Z*)}.  In general,
this is more efficient. 

One of the more commonly used control structures is that used to
construct exponentiation from multiplication by repeated squaring.
This control structure is captured by the internal function
\keylisp{weyli::repeated-squaring}. 

\begin{functiondef}{weyli::repeated-squaring}{mult one}
Returns a function of two arguments that is effectively
\begin{code}
 (lambda (base exp) 
   (declare (integer exp))
   (expt base exp))
\end{code}
except that the body does the exponentiating by repeated squaring using
the operation {\em mult}.  If {\sf exp} is $1$, then {\em one} is
returned.
\end{functiondef}

Using this function, one could have defined exponentiation as
\begin{code}
(defun expt (x n)
  (funcall (weyli::repeated-squaring #'times (coerce 1 (domain-of x)))
           x n))
\end{code}
However, this routine can be used for operations other than
exponentiation.  For instance, if one wanted a routine that replicates
a sequence {\tt n} times, one could use the following:
\begin{code}
(defun replicate-sequence (x n)
  (funcall (weyli::repeated-squaring #'append ())
           x n))
\end{code}

\begin{functiondef}{isqrt}{n}
Returns the integer part of the square root of {\em n}.
\end{functiondef}

\begin{functiondef}{integer-nth-root}{m n}
Computes the largest integer not greater than the $n$-th root of $m$.
\end{functiondef}

\begin{functiondef}{power-of?}{number \optional{} base}
Returns {\em base} and $k$ if $\mbox{\em number} = \mbox{\em base}^k$,
otherwise it returns {\sf nil}.  If {\em base} is not provided returns
the smallest integer of which {\em number} is a perfect power.
\end{functiondef}


\begin{functiondef}{factor}{n}
Factors {\em n} into irreducible factors.  The value returned is a
list of dotted pairs.  The first component of the dotted pair is the
divisor and the second is the number of times the divisor divides {\em
n}.  The type of factorization method used, can be controlled by
setting the variable \keylisp{*factor-method*}.  The allowable values are 
\keylisp{simple-integer-factor} and \keylisp{fermat-integer-factor}.
\end{functiondef}


\begin{functiondef}{prime?}{n}
Returns true if $n$ is a prime number. (For other domains,
if $n$ has no factors that are not units.)
\end{functiondef}  

\begin{functiondef}{totient}{n}
Returns the Euler totient function of {\em n}, the number of positive
integers less than $n$ that are relatively prime to {\em n}, \ie:
\[
\mbox{\sf totient}(n) = n \prod_p \left(1 - \frac{1}{p}\right),
\]
where product is over all prime divisors of {\em n}.
\end{functiondef}

\begin{functiondef}{factorial}{n}
Computes $n!$
\end{functiondef}
  
\begin{functiondef}{pochhammer}{n k}
Computes the Pochhammer function of {\em n} and {\em k}, which is
closely related to the factorial:
\[
\mbox{\sf pochhammer}(n, k) = (n)_k = n \cdot (n+1) \cdot (n+2) \cdots
(n+k-1).
\]
\end{functiondef}


\begin{functiondef}{combinations}{n m}
Computes the number of combinations of {\em n} things taken {\em m} at
a time. 
\[
\mbox{\sf combinations}(n, m) = {n \choose m} = \frac{n!}{m! \,
(n-m)!}.
\]
\end{functiondef}
  
\begin{functiondef}{newprime}{n}
Returns the largest prime less than its argument.
\end{functiondef}

(Need to point out that the elements of the second rational integer
domain created are totally different from that those that are elements
of the first instance of the rational integers.)

\section{Rational Numbers}
\label{Rational:Numbers:Sec}

The domain rational numbers, $\Q$, is the quotient field of the ring
of rational integers.  The elements of a rational number domain can
have structure type either \keylisp{rational-integer} or
\keylisp{rational-number}.  Elements

As in Common Lisp there is a set of four functions for truncating
numbers and ratios to integers.  If the second argument is not
provided then it defaults to 1.  If only the first argument is
provided and it is a rational integer, then all four functions return
the same values.

A domain of rational integers is created by the following function.

\begin{functiondef}{get-rational-numbers}{}
Returns a domain that is isomorphic to the rational numbers, $\Q$.  When
called repeatedly, it always returns the same value until
\keylisp{reset-domains} is called. 
\end{functiondef}


\medskip
\begin{flushleft}
 \functdef{floor}{number \optional divisor} \\
 \functdef{ceiling}{number \optional divisor}\\
 \functdef{truncate}{number \optional divisor}\\
 \functdef{round}{number \optional divisor}
\end{flushleft}


\section{Real Numbers}
\label{Real:Numbers:Sec}

{\em The entire real number situation is somewhat confused.  In
particular, the relationship between floating point numbers and real
numbers is jumbled.  These issues will be fixed at a later date. }

\begin{functiondef}{get-real-numbers}{\optional{}precision}
This returns a domain whose elements are floating point numbers.  If
{\em precision} is not specified, then the machines default double
precision floating point numbers will be used.  If {\em precision} is
specified, then a special arbitrary precision floating point package
will be used.  Operations with these numbers will be somewhat slower
(and will cause more garbage collection) than when using the machine's
floating point data types.
\end{functiondef}

\begin{functiondef}{floor}{number \optional {} divisor}
Computes the floor of {\em number}.
\end{functiondef}

\begin{functiondef}{ceiling}{number \optional {} divisor}
Computes the ceiling of {\em number}.
\end{functiondef}

\begin{functiondef}{truncate}{number \optional {} divisor}
Computes the truncate of {\em number}.
\end{functiondef}

\begin{functiondef}{round}{number \optional {} divisor}
Computes the round of {\em number}.
\end{functiondef}


\begin{functiondef}{sqrt}{n}
For positive $n$ returns positive $\sqrt{n}$ with the same precision
as $n$.
\end{functiondef}


The following standard trigonometric and hyperbolic routines are provided
\begin{center}
\begin{tabular}{llll}
$\mbox{\sf sin}~n$ & $\mbox{\sf asin}~n$ & $\mbox{\sf sinh}~n$ & $\mbox{\sf asinh}~n$ \\
$\mbox{\sf cos}~n$ & $\mbox{\sf acos}~n$ & $\mbox{\sf cosh}~n$ & $\mbox{\sf acosh}~n$ \\
$\mbox{\sf tan}~n$ & $\mbox{\sf atan}~n$ & $\mbox{\sf tanh}~n$ &
$\mbox{\sf atanh}~n$ 
\end{tabular}
\end{center}

\begin{functiondef}{exp}{n}
Returns $e^n$.
\end{functiondef}

\begin{functiondef}{log}{n \optional b}
For positive $n$ returns the principal part of $\log_b n$.  If $b$ is
not supplied then $e$, the base of natural logarithms, is used for $b$.
\end{functiondef}

\section{Complex Numbers}
\label{Complex:Numbers:Sec}

Not yet implemented.

\begin{functiondef}{realpart}{z}
If $z = x + i y$ returns $x$.
\end{functiondef}

\begin{functiondef}{imagpart}{z}
If $z = x + i y$ returns $y$.
\end{functiondef}

\begin{functiondef}{conjugate}{z}
If $z = x + i y$ returns $x - i y$.
\end{functiondef}

\begin{functiondef}{abs}{z}
If $z = x + i y$ returns $|z| = \sqrt{x^2 + y^2}$.
\end{functiondef}

\begin{functiondef}{phase}{z}
If $z = r e^{i\theta}$ returns $\theta$.  $r$ is the absolute value of $z$.
\end{functiondef}

\section{Quaternions}
\label{Quaternion:Sec}
  
Quaternions are a non-commutative algebra over a field, usually the
reals, that are often used to represent three dimensional
rotations. Weyl can construct a quaternion algebra over any field,
$F$. This algebra is a four dimensional vector space over $F$ with the
following relations. The element $\langle 1, 0, 0, 0 \rangle$ is the
multiplicative identity. If we denote $i = \langle 0, 1, 0, 0 \rangle$,
$j = \langle 0, 0, 1, 0 \rangle$ and $k = \langle 0, 0, 0, 1 \rangle$,
then 
\[
i^2 = j^2 = k^2 = -1,\quad ij = - ji,\quad jk = -kj 
\quad\mbox{and}\quad ik = -ki.
\]
  
\begin{functiondef}{get-quaternion-domain}{field}
Gets a quaternion algebra over {\em field}, which must be a field.
\end{functiondef}
  
Quaternions can be created using \keylisp{make-element}. 
  
\begin{functiondef}{make-element}{quaternion-algebra}{$v_1$ $v_2$ $v_3$ $v_4$}
Creates an element of quaternion-algebra from its arguments. The value
returned will be $v_1 + i \cdot v_2 + j \cdot v_3 + k \cdot v_4$.  As
with other versions of {\sf make-element}, the function
{\sf weyli::make-element} assumes the arguments are all elements of the
coefficient domain and is intended only for internal use.
\end{functiondef}
  
As an algebraic extension of the real numbers, the quaternions are a
little strange. The subfield of quaternions generated by $1$ and $i$,
is isomorphic to the complex numbers.  Adding $j$ and $k$ makes the
algebra non-commutative and causes it to violate some basic
intuitions. For instance, $-1$ has at least three square roots!
  
We illustrate some of these issues computationally.
First we create a quaternion algebra in which to work.
\begin{code}  
> (setq q (get-quaternion-domain (get-real-numbers)))
Quat(R)
\end{code}
  
  Next, we can create some elements of the quaternions and do some
simple calculations with them.
\begin{code}
> (setq a (make-element q 1 1 1 1))
<1, 1, 1, 1>

> (setq b (/ a 2))
<1/2, 1/2, 1/2, 1/2>
  
> (* b b b)
<-1, 0, 0, 0>
\end{code}
  
\noindent
As expected, one can multiply quaternions by other quaternions and by
elements of the coefficient field (or objects that can be coerced into
the coefficient field).
  
\begin{functiondef}{conjugate}{quaternion}
This is an extension of the concept of complex conjugation.  It
negates the coefficients of $i$, $j$ and $k$. This is illustrated by
the following example.
\begin{code}
> (setq c (make-element q 1 2 3 4))
<1, 2, 3, 4>
  
> (conjugate c)
<1, -2, -3, -4>

> (* c (conjugate c)
<30, 0, 0, 0>
\end{code}
\end{functiondef}
  
Notice that the components of the product of a quaternion with its
conjugate are all zero except for the very first component. This
matches what happens when one multiplies a complex number with its
complex conjugate.
  


\section{Finite Fields}
\label{Finite:Fields:Sec}

The usual finite fields are provided in Weyl, $\F_p$ and algebraic
extensions of $\F_q$.  Such domains are called {\em GFp} domains.
Since all finite fields with the same number of elements are
isomorphic, fields are created by specifying the elements in the
field.

\begin{functiondef}{get-finite-field}{size}
{\em Size} is expected to be a a power of a prime number.  This
function returns a finite field with the indicated number of elements.
If {\em size} is {\sf nil} then a {\em GFm} field is returned. 
\end{functiondef}

\begin{functiondef}{number-of-elements}{finite-field}
Returns the number of elements in {\em finite-field}.
\end{functiondef}

At the moment Weyl can only deal with the fields $\F_{2^k}$ and $F_p$.
For instance,
\begin{code}
> (setq F256 (get-finite-field 256))
GF(2^8)

> (characteristic F256)
2

> (number-of-elements F256)
256
\end{code}


Elements of a GFp are created by coercing a rational integer into a
GFp domain.  For finite fields with characteristic greater than $2$,
coercing an integer into $F_p$ maps $n$ into $n \pmod p$.  For
$F_{2^k}$, the image of an integer is a bit more complicated.  Let the
binary representation of $n$ be 
\[
n = {n_\ell n_{\ell-1} \cdots n_2 n_1 n_0}_2,
\]
and let $\alpha$ be the primitive element of $F_{2^k}$ over $F_2$.
Then
\[
n \longrightarrow n_{k-1} \alpha^{k-1} + n_{n-2}\alpha^{k-2} + \cdots
 + n_1 \alpha + n_0.
\]
This mapping is particularly appropriate for problems in coding
theory.

In addition, elements of finite fields can be created using
\keylisp{make-element}.  

\begin{functiondef}{make-element}{finite-field integer \optional rest}
Creates an element of {\em finite-field} from {\em integer\/}.  This
is the only way to create elements of $\F_{p^k}$.  (As with all {\sf
make-element} methods, the argument list includes \&rest arguments,
but for finite fields any additional arguments are ignored.)
\end{functiondef}

As an example of the use of finite fields, consider the following
function, which determines the order of an element of a finite field
(the hard way).
\begin{code}
(defun element-order (n)
  (let* ((domain (domain-of n))
         (one (coerce 1 domain)))
    (loop for i upfrom 1 below (number-of-elements domain)
          for power = n then (* n power)
     	  do (when (= power one) 
               (return i)))))
\end{code}

A more efficient routine is provided by Weyl as {\sf multiplicative-order}. 

\begin{functiondef}{multiplicative-order}{elt}
{\em Elt} must be an element of a finite field.  This routine computes
multiplicative order of {\em elt}.  This routine requires factoring
the size of the multiplicative group of the finite field and thus is
appropriate for very large finite fields.
\end{functiondef}

The following illustrates use of these routines.
\begin{code}
> (element-order (coerce 5 (get-finite-field 41)))
20

> (multiplicative-order (coerce 5 (get-finite-field 41)))
\end{code}

\medskip
Consider what is involved when implementing an algorithm using the
Chinese remainder theorem. The computation is done in a number of
domains like $\Z/(p_1)$, $\Z/(p_2)$ and $\Z/(p_3)$.  The results are
then combined to produce results in the domains $\Z/(p_1p_2)$ and
$\Z/(p_1p_2p_3)$.  Rather than working in several different domains
and explicitly coercing the elements from one to another, it is easier
to assume we are working in a single domain that is the union of
$\Z/(m)$ for all integers $m$ and marking the elements of this domain
with their moduli.  We call this domain a \keylisp{GFm}.

\keylisp{GFm} domains are also created using the \keylisp{get-finite-field}
but by providing {\tt nil} as the number of elements in the field.

Elements of \keylisp{GFm} are printed by indicating their modulus in a
subscript surrounded by parentheses.  Thus $2_{(5)}$ means $2$ modulo
$5$.  Combining two elements $a_{(m)}$ and $b_{(m)}$ that have the
same moduli is the same as if they were both elements of $\Z/(m)$.  To
combine elements of two different rings, we find a ring that contains
both as subrings and perform the calculation there.  Thus combining
$a_{(m)}$ and $b_{(n)}$ we combine the images of $a$ and $b$ as
elements of $\Z/(\gcd(m,n))$.



FIXTHIS: Need works something out for dealing with completions of the
integers at primes, and how we are going to compute with elements.

